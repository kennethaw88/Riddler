\documentclass{article}


\usepackage{amsmath} % math stuff
\usepackage{amssymb} % math stuff
\usepackage{array} % equations and stuff
\usepackage{bm} % bold math
%\usepackage{caption} % suppressed table numbering; incompatible with revtex, and longtable, I think
\usepackage{comment} % comment environment
%\usepackage{enumitem} % customization of enumeration, itemize, and description
\usepackage[T1]{fontenc} % font encoding for special characters, must also use scalable font package
\usepackage[margin=0.8in]{geometry} % paper sizes and margins (but be careful not to mess up pre-defined pages)
\usepackage{graphicx} % for graphics
%\usepackage{helvet} % default font is the helvetica postscript font
\usepackage{lipsum} % lorem ipsum filler text
\usepackage{lmodern} % scalable font?
\usepackage{longtable} % multi-page tables
\usepackage{mathrsfs} % math script font
\usepackage{mhchem} % easier chemical formula
\usepackage{microtype} % allows disabling of ligatures
%\usepackage{newcent} % new century schoolbook font
\usepackage{nicefrac}
\usepackage{parskip} % removes paragraph indentation, and adjusts paragraph skip, as well as list items
%\usepackage{setspace} % adjust text spacing and indents
\usepackage{siunitx} % decimal alignment
\usepackage{subfigure} % divided figures
%\usepackage{tabu} % extra table options
\usepackage{textcomp} % symbols
\usepackage{threeparttablex} % better footnotes with longtable
\usepackage{titling} % title placement
\usepackage{ulem} % strikethrough text
%\usepackage{url} % superceded by hyperref
\usepackage{verbatim} % verbatim environment
\usepackage{xcolor} % colors and color boxes
\usepackage{xspace} % commands that don't eat up white space
\usepackage{hyperref} % links and page setup; should always come last

\hypersetup{
	bookmarks=true,
	colorlinks=true,
	citecolor=blue,
	linkcolor=blue,
	urlcolor=blue,
	pdfstartview={XYZ null null 1.0} % default open view is 100%
}

\DisableLigatures[f]{encoding = *, family = * } % disable ff, fi, fl ligatures, without f option, it also disables -- = endash
\renewcommand{\arraystretch}{1.1} % extra vertical space in tables

\begin{document}

\pagestyle{empty} % don't number pages

% custom title
\begin{center}
{\LARGE Classic Riddler}

\vspace{0.15in}

{\Large 7 August 2020}
\end{center}


\section*{Riddle:}

We usually think of addition as an operation applied to a field like the rational numbers or the real numbers.
And there is good reason for that---as Kareem says, ``Mathematicians have done all the hard work of figuring out how to make calculations track with reality.
They kept modifying and refining the number system until everything worked out.
It took centuries of brilliant minds to do this!''

Now suppose we defined addition another (admittedly less useful) way, using a classic model organism: the nematode.
To compute the sum of $x$ and $y$, you combine groups of $x$ and $y$ nematodes and leave them for 24 hours.
When you come back, you count up how many you have---and that's the sum!

It turns out that, over the course of 24 hours, the nematodes pair up, and each pair has one offspring 50 percent of the time.
(If you have an odd number of nematodes, they will still pair up, but one will be left out.)
So if you want to compute 1+1, half the time you'll get 2 and half the time you'll get 3.
If you compute 2+2, 25 percent of the time you get 4, 50 percent of the time you'll get 5, and 25 percent of the time you'll get 6.

While we're at it, let's define exponentiation for sums of nematodes.
Raising a sum to a power means leaving that sum of nematodes for the number of days specified by the exponent.

With this number system, what is the expected value of $(1+1)^{4}$?

\textit{Extra credit}: As $N$ gets larger and larger, what does the expected value of $(1+1)^{N}$ approach?

\section*{Solution:}

These numbers are small enough that they can just be calculated by hand.
First, I have calculated the probability distributions after 1 iteration for each starting number.
For example, starting with 2 (or 1+1), the results are 2 or 3, each with 50\% probability (as listed in the riddle).
I list the others below:

\begin{center}
\begin{tabular}{c c}
Start & Result (Probability) \\
\hline
3 & 3 (\nicefrac{1}{2}), 4 (\nicefrac{1}{2}) \\
4 & 4 (\nicefrac{1}{4}), 5 (\nicefrac{1}{2}), 6 (\nicefrac{1}{4}) \\
5 & 5 (\nicefrac{1}{4}), 6 (\nicefrac{1}{2}), 7 (\nicefrac{1}{4}) \\
6 & 6 (\nicefrac{1}{8}), 7 (\nicefrac{3}{8}), 8 (\nicefrac{3}{8}), 9 (\nicefrac{1}{8})
\end{tabular}
\end{center}

After iterating out the calculation four times, the possible outcomes and probabilities are 2 (\nicefrac{1}{16}), 3 (\nicefrac{1}{4}), 4 (\nicefrac{17}{64}), 5 (\nicefrac{3}{16}), 6 (\nicefrac{19}{128}), 7 (\nicefrac{7}{128}), 8 (\nicefrac{3}{128}), and 9 (\nicefrac{1}{128}).
Combining these yields an expected result of
\fcolorbox{red}{white}{$\bm{{\nicefrac{141}{32}=4.40625}}$}\,.

I don't have any idea about how to approach the extra credit.
I could probably simulate it many many times, but I'm not sure how large $N$ to use to get a useful pattern, or what the long-term pattern should be.

\end{document}