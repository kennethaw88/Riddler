\documentclass{article}

\usepackage{amsmath} % math stuff
\usepackage{amssymb} % math stuff
\usepackage{array} % equations and stuff
\usepackage{bm} % bold math
%\usepackage{booktabs} % extra table rule options
%\usepackage{caption} % suppressed table numbering; incompatible with revtex, and longtable, I think
\usepackage{comment} % comment environment
%\usepackage{enumitem} % customization of enumeration, itemize, and description
\usepackage[T1]{fontenc} % font encoding for special characters, must also use scalable font package
\usepackage[margin=0.8in]{geometry} % paper sizes and margins (but be careful not to mess up pre-defined pages)
\usepackage{graphicx} % for graphics
%\usepackage{helvet} % default font is the helvetica postscript font
\usepackage[utf8]{inputenc} % special characters in tex input
\usepackage{layouts} % print units like widths
\usepackage{lipsum} % lorem ipsum filler text
\usepackage{lmodern} % scalable font?
\usepackage{longtable} % multi-page tables
\usepackage{makecell} % specify line-breaks in table cells
\usepackage{mathrsfs} % math script font
\usepackage{mhchem} % easier chemical formula
\usepackage{microtype} % allows disabling of ligatures
\usepackage{multicol} % multicolumns
%\usepackage{newcent} % new century schoolbook font
\usepackage{nicefrac}
\usepackage{numprint} % print and format (large) numbers
\usepackage{parskip} % removes paragraph indentation, and adjusts paragraph skip, as well as list items
\usepackage{pdfpages} % add pdf files as pages
%\usepackage{setspace} % adjust text spacing and indents
\usepackage{siunitx} % decimal alignment
\usepackage{subfigure} % divided figures
%\usepackage{tabu} % extra table options
\usepackage{textcomp} % symbols
\usepackage{threeparttablex} % better footnotes with longtable
\usepackage{titling} % title placement
\usepackage{ulem} % strikethrough text
%\usepackage{url} % superceded by hyperref
\usepackage{verbatim} % verbatim environment
\usepackage{xcolor} % colors and color boxes
\usepackage{xspace} % commands that don't eat up white space
\usepackage{hyperref} % links and page setup; should always come last

\hypersetup{
 bookmarks=true,
 colorlinks=true,
 citecolor=blue,
 linkcolor=blue,
 urlcolor=blue,
 pdfstartview={XYZ null null 1.0} % default open view is 100%
}

\DisableLigatures[f,t]{encoding = T1} % disable ff, fi, fl, tt ligatures; without options, it also disables -- = endash
\renewcommand{\arraystretch}{1.0} % extra vertical (and horizontal?) space in tables

% define centered, left- and right-aligned columns with specified widths
\newcommand{\PreserveBackslash}[1]{\let\temp=\\#1\let\\=\temp}
\newcolumntype{C}[1]{>{\PreserveBackslash\centering}p{#1}}
\newcolumntype{L}[1]{>{\PreserveBackslash\raggedright}p{#1}}
\newcolumntype{R}[1]{>{\PreserveBackslash\raggedleft}p{#1}}

\begin{document}

\pagestyle{empty} % don't number pages

% custom title
\begin{center}
{\LARGE Express Riddler}

\vspace{0.15in}

{\Large 26 November 2021}
\end{center}


\section*{Riddle:}

On the Food Network's latest game show, Cranberries or Bust, you have a choice between two doors: A and B.
One door has a lifetime supply of cranberry sauce behind it, while the other door has absolutely nothing behind it.
And boy, do you love cranberry sauce.

Of course, there's a twist.
The host presents you with a coin with two sides, marked A and B, which correspond to each door.
The host tells you that the coin is weighted in favor of the cranberry door---without telling you which door that is---and that door's letter will turn up 60 percent of the time.
For example, if the sauce is behind door A, then the coin will turn up A 60 percent of the time and B the remaining 40 percent of the time.

You can flip the coin twice, after which you must make your selection.
Assuming you optimize your strategy, what are your chances of choosing the door with the cranberry sauce?

\textit{Extra credit}: Instead of two flips, what if you are allowed three or four flips?
Now what are your chances of choosing the door with the cranberry sauce?


\section*{Solution:}

With two flips, there are four possible outcomes: AA, AB, BA, and BB.
The optimal strategy is simple: if you get the same side both flips, choose that door; if you get each side once, choose a door randomly.
This strategy works for either door being correct.
Without loss of generality, assume that the correct door is A.
Flipping AA occurs with probability $0.6^{2}=0.36$; if you flip this, you would be correct also with probability 0.36.
Flipping BB occurs with probability $0.4^{2}=0.16$; if you flip this, you would lose with probability 0.16.
There is a combined probability of $2\cdot0.6\cdot0.4=0.48$ of flipping AB or BA; if you flip this, you would win probability $0.48/2=0.24$.
Thus your total probablilty of winning is $0.36+0.24=$
\fcolorbox{red}{white}{\bf 0.6}\,.
Interestingly, this is the same probability as flipping the coin a single time.

Similarly, the probability of winning with three or four flips is the same.
For the three-flip case, you should pick whichever side flips over at least twice.
For the four-flip case, you should pick whichever side flips over at least twice out of the first three.
The probability of flipping AAA is $0.6^{3}=0.216$.
The probability of flipping any of AAB, ABA, or BAA is $3\cdot0.6^{2}\cdot0.4=0.432$.
Thus the total probability of winning in this case is
\fcolorbox{red}{white}{\bf 0.648}\,.



\end{document}