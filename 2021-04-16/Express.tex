\documentclass{article}

\usepackage{amsmath} % math stuff
\usepackage{amssymb} % math stuff
\usepackage{array} % equations and stuff
\usepackage{bm} % bold math
%\usepackage{booktabs} % extra table rule options
%\usepackage{caption} % suppressed table numbering; incompatible with revtex, and longtable, I think
\usepackage{comment} % comment environment
%\usepackage{enumitem} % customization of enumeration, itemize, and description
\usepackage[T1]{fontenc} % font encoding for special characters, must also use scalable font package
\usepackage[margin=0.8in]{geometry} % paper sizes and margins (but be careful not to mess up pre-defined pages)
\usepackage{graphicx} % for graphics
%\usepackage{helvet} % default font is the helvetica postscript font
\usepackage{layouts} % print units like widths
\usepackage{lipsum} % lorem ipsum filler text
\usepackage{lmodern} % scalable font?
\usepackage{longtable} % multi-page tables
\usepackage{makecell} % specify line-breaks in table cells
\usepackage{mathrsfs} % math script font
\usepackage{mhchem} % easier chemical formula
\usepackage{microtype} % allows disabling of ligatures
%\usepackage{newcent} % new century schoolbook font
\usepackage{nicefrac}
\usepackage{numprint} % print and format (large) numbers
\usepackage{parskip} % removes paragraph indentation, and adjusts paragraph skip, as well as list items
\usepackage{pdfpages} % add pdf files as pages
%\usepackage{setspace} % adjust text spacing and indents
\usepackage{siunitx} % decimal alignment
\usepackage{subfigure} % divided figures
%\usepackage{tabu} % extra table options
\usepackage{textcomp} % symbols
\usepackage{threeparttablex} % better footnotes with longtable
\usepackage{titling} % title placement
\usepackage{ulem} % strikethrough text
%\usepackage{url} % superceded by hyperref
\usepackage{verbatim} % verbatim environment
\usepackage{xcolor} % colors and color boxes
\usepackage{xspace} % commands that don't eat up white space
\usepackage{hyperref} % links and page setup; should always come last

\hypersetup{
 bookmarks=true,
 colorlinks=true,
 citecolor=blue,
 linkcolor=blue,
 urlcolor=blue,
 pdfstartview={XYZ null null 1.0} % default open view is 100%
}

\DisableLigatures[f,t]{encoding = T1} % disable ff, fi, fl, tt ligatures; without options, it also disables -- = endash
\renewcommand{\arraystretch}{1.0} % extra vertical (and horizontal?) space in tables

% define centered, left- and right-aligned columns with specified widths
\newcommand{\PreserveBackslash}[1]{\let\temp=\\#1\let\\=\temp}
\newcolumntype{C}[1]{>{\PreserveBackslash\centering}p{#1}}
\newcolumntype{L}[1]{>{\PreserveBackslash\raggedright}p{#1}}
\newcolumntype{R}[1]{>{\PreserveBackslash\raggedleft}p{#1}}

\begin{document}

\pagestyle{empty} % don't number pages

% custom title
\begin{center}
{\LARGE Express Riddler}

\vspace{0.15in}

{\Large 16 April 2021}
\end{center}


\section*{Riddle:}

You are creating a variation of a Romulan pixmit deck.
Each card is an equilateral triangle, with one of the digits 0 through 9 (written in Romulan, of course) at the base of each side of the card.
No number appears more than once on each card.
Furthermore, every card in the deck is unique, meaning no card can be rotated so that it matches (i.e., can be superimposed on) any other card.

What is the greatest number of cards your pixmit deck can have?

\textit{Extra credit}: Suppose you allow numbers to appear two or three times on a given card.
Once again, no card can be rotated so that it matches any other card.
Now what is the greatest number of cards your pixmit deck can have?



\section*{Solution:}

I think it is a safe assumption that the cards are one-sided, which means that if a given card is mirrored, it produces a different card.
To choose three different numbers for the sides of each card, there are 10 available numbers for the first side, 9 for the second side, and 8 for the third side.
This gives a total of $10\times9\times8=720$ possible ways to number a card.
But because rotating the card doesn't change it, there are actually fewer possibilities.
The card could be rotated \nicefrac{1}{3} or \nicefrac{2}{3} of the way around, so each card actually accounts for 3 possible sets of three numbers (the two rotations in addition to the original set of three numbers).
So the correct solution is that there are $\nicefrac{10\times9\times8}{3}$=
\fcolorbox{red}{white}{\bf 240} possible cards.

When including cards with two numbers, there are now 10 options for the number which appears twice, and 9 resulting options for the number that appears once.
This adds $10\times9=90$ new cards.
This number does not change, because rotating one of these sets of numbers does create a new set of numbers.

When including cards with a single number, there are simply 10 additional cards: one for each digit 0--9.
Considering all these together, there would be
\fcolorbox{red}{white}{\bf 340} total cards.

\end{document}