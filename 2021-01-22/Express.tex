\documentclass{article}

\usepackage{amsmath} % math stuff
\usepackage{amssymb} % math stuff
\usepackage{array} % equations and stuff
\usepackage{bm} % bold math
%\usepackage{caption} % suppressed table numbering; incompatible with revtex, and longtable, I think
\usepackage{comment} % comment environment
%\usepackage{enumitem} % customization of enumeration, itemize, and description
\usepackage[T1]{fontenc} % font encoding for special characters, must also use scalable font package
\usepackage[margin=0.8in]{geometry} % paper sizes and margins (but be careful not to mess up pre-defined pages)
\usepackage{graphicx} % for graphics
%\usepackage{helvet} % default font is the helvetica postscript font
\usepackage{layouts} % print units like widths
\usepackage{lipsum} % lorem ipsum filler text
\usepackage{lmodern} % scalable font?
\usepackage{longtable} % multi-page tables
\usepackage{makecell} % specify line-breaks in table cells
\usepackage{mathrsfs} % math script font
\usepackage{mhchem} % easier chemical formula
\usepackage{microtype} % allows disabling of ligatures
%\usepackage{newcent} % new century schoolbook font
\usepackage{nicefrac}
\usepackage{parskip} % removes paragraph indentation, and adjusts paragraph skip, as well as list items
\usepackage{pdfpages} % add pdf files as pages
%\usepackage{setspace} % adjust text spacing and indents
\usepackage{siunitx} % decimal alignment
\usepackage{subfigure} % divided figures
%\usepackage{tabu} % extra table options
\usepackage{textcomp} % symbols
\usepackage{threeparttablex} % better footnotes with longtable
\usepackage{titling} % title placement
\usepackage{ulem} % strikethrough text
%\usepackage{url} % superceded by hyperref
\usepackage{verbatim} % verbatim environment
\usepackage{xcolor} % colors and color boxes
\usepackage{xspace} % commands that don't eat up white space
\usepackage{hyperref} % links and page setup; should always come last

\hypersetup{
 bookmarks=true,
 colorlinks=true,
 citecolor=blue,
 linkcolor=blue,
 urlcolor=blue,
 pdfstartview={XYZ null null 1.0} % default open view is 100%
}

\DisableLigatures[f,t]{encoding = T1} % disable ff, fi, fl, tt ligatures, without f option, it also disables -- = endash
\renewcommand{\arraystretch}{2.0} % extra vertical space in tables

\begin{document}

\pagestyle{empty} % don't number pages

% custom title
\begin{center}
{\LARGE Express Riddler}

\vspace{0.15in}

{\Large 22 January 2021}
\end{center}


\section*{Riddle:}

You're reviewing some of the survey data that was randomly collected from the residents of Riddler City.
As you'll recall, the city is quite large.

Ten randomly selected residents were asked how many people (including them) lived in their household.
As it so happened, their answers were 1, 2, 3, 4, 5, 6, 7, 8, 9 and 10.

It's your job to use this (admittedly limited) data to estimate the average household size in Riddler City.
Your co-worker suggests averaging the 10 numbers, which would give you an answer of about 5.5 people.
But you're not so sure.

Would your best estimate be exactly 5.5, less than 5.5 or greater than 5.5?


\section*{Solution:}

The correct interpretation of the co-worker's average is that the average resident lives in a 5.5-member household, or equivalently, each resident has on average 4.5 housemates.
This is the average from the ``perspective'' of the residents.
From the perspective, of the households themselves, the average should be lower.
To illustrate, in a 10-person household would have 10 people contributing to the residents' average, but only a single household contributing to the household average.

The sample data indicate that 10\% of residents live in a one-person household, 10\% live in a two-person household, and so on.
So for every single ten-person household, there are 10 one-person households, 5 two-person households, 3.33\dots three-person households, 2.5 four-person households, 2 five-person households, 1.66\dots six-person households, 1.42\dots seven-person households, 1.25 eight-person households, and 1.11\dots nine-person households.
In general the number of $n$-person households (per ten-person household) is \nicefrac{10}{n}.
From this, the relative number of residents in each size household is $n(\nicefrac{10}{n})=10$, so the household sizes are equally distributed among residents.

To calculate the average household size $H$, the sizes must be added according to the relative weight (the number of households of each size) and normalized to the weighting:

\[
H=\frac{\displaystyle\sum_{n=1}^{10}n\left(\frac{10}{n}\right)}{\displaystyle\sum_{n=1}^{10}\frac{10}{n}}
\]

Evaluating this sum gives an average household size of
\fcolorbox{red}{white}{$\bm{H\approx3.415}$}\,.





\end{document}