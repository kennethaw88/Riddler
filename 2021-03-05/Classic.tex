\documentclass{article}

\usepackage{amsmath} % math stuff
\usepackage{amssymb} % math stuff
\usepackage{array} % equations and stuff
\usepackage{bm} % bold math
%\usepackage{booktabs} % extra table rule options
%\usepackage{caption} % suppressed table numbering; incompatible with revtex, and longtable, I think
\usepackage{comment} % comment environment
%\usepackage{enumitem} % customization of enumeration, itemize, and description
\usepackage[T1]{fontenc} % font encoding for special characters, must also use scalable font package
\usepackage[margin=0.8in]{geometry} % paper sizes and margins (but be careful not to mess up pre-defined pages)
\usepackage{graphicx} % for graphics
%\usepackage{helvet} % default font is the helvetica postscript font
\usepackage{layouts} % print units like widths
\usepackage{lipsum} % lorem ipsum filler text
\usepackage{lmodern} % scalable font?
\usepackage{longtable} % multi-page tables
\usepackage{makecell} % specify line-breaks in table cells
\usepackage{mathrsfs} % math script font
\usepackage{mhchem} % easier chemical formula
\usepackage{microtype} % allows disabling of ligatures
%\usepackage{newcent} % new century schoolbook font
\usepackage{nicefrac}
\usepackage{numprint} % print and format (large) numbers
\usepackage{parskip} % removes paragraph indentation, and adjusts paragraph skip, as well as list items
\usepackage{pdfpages} % add pdf files as pages
%\usepackage{setspace} % adjust text spacing and indents
\usepackage{siunitx} % decimal alignment
\usepackage{subfigure} % divided figures
%\usepackage{tabu} % extra table options
\usepackage{textcomp} % symbols
\usepackage{threeparttablex} % better footnotes with longtable
\usepackage{titling} % title placement
\usepackage{ulem} % strikethrough text
%\usepackage{url} % superceded by hyperref
\usepackage{verbatim} % verbatim environment
\usepackage{xcolor} % colors and color boxes
\usepackage{xspace} % commands that don't eat up white space
\usepackage{hyperref} % links and page setup; should always come last

\hypersetup{
 bookmarks=true,
 colorlinks=true,
 citecolor=blue,
 linkcolor=blue,
 urlcolor=blue,
 pdfstartview={XYZ null null 1.0} % default open view is 100%
}

\DisableLigatures[f,t]{encoding = T1} % disable ff, fi, fl, tt ligatures; without options, it also disables -- = endash
\renewcommand{\arraystretch}{1.0} % extra vertical (and horizontal?) space in tables

% define centered, left- and right-aligned columns with specified widths
\newcommand{\PreserveBackslash}[1]{\let\temp=\\#1\let\\=\temp}
\newcolumntype{C}[1]{>{\PreserveBackslash\centering}p{#1}}
\newcolumntype{L}[1]{>{\PreserveBackslash\raggedright}p{#1}}
\newcolumntype{R}[1]{>{\PreserveBackslash\raggedleft}p{#1}}

\begin{document}

\pagestyle{empty} % don't number pages

% custom title
\begin{center}
{\LARGE Classic Riddler}

\vspace{0.15in}

{\Large 5 March 2021}
\end{center}


\section*{Riddle:}

On Feb. 23, baseball statistician Jim Passon tweeted out some recent stats for baseball's God of WAR, Mike Trout:

\begin{quote}
Mike Trout's past$\dots$

$\dots$299 games = .299 BA

$\dots$440 games = .440 OBP

$\dots$610 games = .610 SLG

$\dots$1,020 games = 1.020 OPS

— Jim Passon (@PassonJim) \href{https://twitter.com/PassonJim/status/1364086333581393923}{February 23, 2021}
\end{quote}

Yes, Trout's numbers are gaudy.
But what caught social media's attention was that each stat (batting average, on-base percentage, slugging percentage and OPS) \textit{somehow} had a corresponding number of games over which Trout averaged those stats.
For example, Trout batted .299 over his last 299 games. Could this possibly be a coincidence?

If you think about it for a minute, you'll realize that there's nothing magical at all about having a batting average that matches a corresponding number of games.

Suppose a baseball player has four at-bats per game (not including walks), so their batting average is the number of hits they got divided by four times the number of games they played.
For many games, it's possible to have a corresponding batting average that, when rounded to three digits, equals the number of games divided by 1,000.
For example, if a player typically gets one hit per game in their at-bats, then they could very well have a .250 average over 250 games.

What is the greatest number of games for which it is \textit{not} possible to have a matching rounded batting average?
Again, assume four at-bats per game.



\section*{Solution:}

I set up the math for this riddle in a spreadsheet, called \texttt{Batting\_average.xlsx}\,.
I started counting down from 1,000 games in order to find the first instance where having a matching stat is not possible.
For each number of games, I multiplied by the corresponding fraction to get an exact (decimal) value for number of hits.
This decimal was rounded up and down to get the best estimates for the integer number of hits.
Then the actual stat was calculated for each of these integer values.
Finally, I just searched for the first instance where neither of these calculated stats matched the number of games.

The first instance where this occurs is for 239 games (956 at-bats).
The closest possible stats are $\nicefrac{228}{239}\approx0.238$ and $\nicefrac{229}{239}\approx0.240$.
So the solution is
\fcolorbox{red}{white}{\bf 239}\,.


\end{document}