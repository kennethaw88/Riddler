\documentclass{article}


\usepackage{amsmath} % math stuff
\usepackage{amssymb} % math stuff
\usepackage{array} % equations and stuff
\usepackage{bm} % bold math
%\usepackage{caption} % suppressed table numbering; incompatible with revtex, and longtable, I think
\usepackage{comment} % comment environment
%\usepackage{enumitem} % customization of enumeration, itemize, and description
\usepackage[T1]{fontenc} % font encoding for special characters, must also use scalable font package
\usepackage[margin=0.8in]{geometry} % paper sizes and margins (but be careful not to mess up pre-defined pages)
\usepackage{graphicx} % for graphics
%\usepackage{helvet} % default font is the helvetica postscript font
\usepackage{lipsum} % lorem ipsum filler text
\usepackage{lmodern} % scalable font?
\usepackage{longtable} % multi-page tables
\usepackage{mathrsfs} % math script font
\usepackage{mhchem} % easier chemical formula
\usepackage{microtype} % allows disabling of ligatures
%\usepackage{newcent} % new century schoolbook font
\usepackage{nicefrac}
\usepackage{parskip} % removes paragraph indentation, and adjusts paragraph skip, as well as list items
%\usepackage{setspace} % adjust text spacing and indents
\usepackage{siunitx} % decimal alignment
\usepackage{subfigure} % divided figures
%\usepackage{tabu} % extra table options
\usepackage{textcomp} % symbols
\usepackage{threeparttablex} % better footnotes with longtable
\usepackage{titling} % title placement
\usepackage{ulem} % strikethrough text
%\usepackage{url} % superceded by hyperref
\usepackage{verbatim} % verbatim environment
\usepackage{xcolor} % colors and color boxes
\usepackage{xspace} % commands that don't eat up white space
\usepackage{hyperref} % links and page setup; should always come last

\hypersetup{
	bookmarks=true,
	colorlinks=true,
	citecolor=blue,
	linkcolor=blue,
	urlcolor=blue,
	pdfstartview={XYZ null null 1.0} % default open view is 100%
}

\DisableLigatures[f]{encoding = *, family = * } % disable ff, fi, fl ligatures, without f option, it also disables -- = endash
\renewcommand{\arraystretch}{1} % extra vertical space in tables

\begin{document}

\pagestyle{empty} % don't number pages

% custom title
\begin{center}
{\LARGE Classic Riddler}

\vspace{0.15in}

{\Large 24 April 2020}
\end{center}


\section*{Riddle:}

The Monty Hall problem is a classic case of conditional probability.
In the original problem, there are three doors, two of which have goats behind them, while the third has a prize.
You pick one of the doors, and then Monty (who knows in advance which door has the prize) will always open another door, revealing a goat behind it.
It’s then up to you to choose whether to stay with your initial guess or to switch to the remaining door.
Your best bet is to switch doors, in which case you will win the prize two-thirds of the time.

Now suppose Monty changes the rules.
First, he will randomly pick a number of goats to put behind the doors: zero, one, two or three, each with a 25 percent chance.
After the number of goats is chosen, they are assigned to the doors at random, and each door has at most one goat.
Any doors that don’t have a goat behind them have an identical prize behind them.

At this point, you choose a door.
If Monty is able to open another door, revealing a goat, he will do so.
But if no other doors have goats behind them, he will tell you that is the case.

It just so happens that when you play, Monty is able to open another door, revealing a goat behind it.
Should you stay with your original selection or switch?
And what are your chances of winning the prize?

\section*{Solution:}

These types of problems always confuse me.
To get through this, I typed up the table laid out in \texttt{Super\_Monty\_Hall.xlsx}.
This is basically a list of all possible combinations of outcomes.
Each outcome has to consider several things, and the table goes through these in order:

\begin{enumerate}
\item the number of goats
\item which door the contestant chooses
\item if Monty can open a door with a goat
\item which door Monty opens
\item to which door the contestant switches
\item whether the contestant wins
\end{enumerate}

Without loss of generality, however, I let the contestant always choose the first door first, and then if there is a door opened switch to the next available door.
This doesn't affect the result, because each choice can be permuted, and the total statistics would look the same, only with more outcomes to sort through.
In principle, each outcome could just be listed out, but the probability of each outcome is not the same; having a table makes the calculations much easier.

In the spreadsheet, the successive probability of getting through each step is calculated.
Of course, the probability of each first step (number of goats) is just \nicefrac{1}{4}, as stated in the riddle.
But each successive step has different probabilities, depending on how the outcomes are divided up.
For example, because there are three different ways to arrange one goat, but only 1 way to arrange three goats, the probability of getting the door arrangements for these cases are different.
Similarly, when there are multiple doors available for Monty to open, these lead to multiple outcomes, splitting the probabilities further in those cases.

The end result is that the probability of both winning by not switching and Monty opening one of the doors is \nicefrac{1}{4}.
Similarly, the probability of winning by switching with an open door is \nicefrac{1}{3}.
But the solution to the riddle needs to take into account that Monty did in fact open a door; the answer is a conditional probability.
So these numbers need to be divided by the overall probability that Monty opened a door, which is \nicefrac{2}{3}.
The probabilities then become \nicefrac{3}{8} and \nicefrac{1}{2}, respectively.
The solution to the riddle is therefore to
\fcolorbox{red}{white}{\bf switch doors}\,,
giving the contestant a probability of winning of
\fcolorbox{red}{white}{\bf \nicefrac{1}{2}}\,.

I found it interesting that the overall probability of selecting a winning door from the beginning is \nicefrac{1}{2}.
This is because of the symmetry of the problem.
It is just as likely for there to be three goats as there are no goats.
So really it seems like the above situation is only as good or worse than just picking a door and opening it immediately with no second option.
But then how can the outcome be worse if the contestant has more information?
Really, by opening a door with a goat, Monty is revealing that the constestant is in some outcome with goats, in which on average there is more than one goat.
This overall lowers the initial probability of choosing a winning door.
But if there are no goats---or if the contestant picks the single door with a goat---Monty has no doors to open, and the contestant is guaranteed a win by switching.
So the two sets of outcomes do balance out.



\end{document}