\documentclass{article}

\usepackage{amsmath} % math stuff
\usepackage{amssymb} % math stuff
\usepackage{array} % equations and stuff
\usepackage{bm} % bold math
%\usepackage{booktabs} % extra table rule options
%\usepackage{caption} % suppressed table numbering; incompatible with revtex, and longtable, I think
\usepackage{comment} % comment environment
%\usepackage{enumitem} % customization of enumeration, itemize, and description
\usepackage[T1]{fontenc} % font encoding for special characters, must also use scalable font package
\usepackage[margin=0.8in]{geometry} % paper sizes and margins (but be careful not to mess up pre-defined pages)
\usepackage{graphicx} % for graphics
%\usepackage{helvet} % default font is the helvetica postscript font
\usepackage[utf8]{inputenc} % special characters in tex input
\usepackage{layouts} % print units like widths
\usepackage{lipsum} % lorem ipsum filler text
\usepackage{lmodern} % scalable font?
\usepackage{longtable} % multi-page tables
\usepackage{makecell} % specify line-breaks in table cells
\usepackage{mathrsfs} % math script font
\usepackage{mhchem} % easier chemical formula
\usepackage{microtype} % allows disabling of ligatures
\usepackage{multicol} % multicolumns
%\usepackage{newcent} % new century schoolbook font
\usepackage{nicefrac}
\usepackage{numprint} % print and format (large) numbers
\usepackage{parskip} % removes paragraph indentation, and adjusts paragraph skip, as well as list items
\usepackage{pdfpages} % add pdf files as pages
%\usepackage{setspace} % adjust text spacing and indents
\usepackage{siunitx} % decimal alignment
\usepackage{subfigure} % divided figures
%\usepackage{tabu} % extra table options
\usepackage{textcomp} % symbols
\usepackage{textgreek} % Greek letters in text mode
\usepackage{threeparttablex} % better footnotes with longtable
\usepackage{titling} % title placement
\usepackage{ulem} % strikethrough text
\usepackage{upgreek} % upright Greek letters
%\usepackage{url} % superceded by hyperref
\usepackage{verbatim} % verbatim environment
\usepackage{xcolor} % colors and color boxes
\usepackage{xspace} % commands that don't eat up white space
\usepackage{hyperref} % links and page setup; should always come last

\hypersetup{
 bookmarks=true,
 colorlinks=true,
 citecolor=blue,
 linkcolor=blue,
 urlcolor=blue,
 pdfstartview={XYZ null null 1.0} % default open view is 100%
}

\DisableLigatures[f,t]{encoding = T1} % disable ff, fi, fl, tt ligatures; without options, it also disables -- = endash
\renewcommand{\arraystretch}{1.0} % extra vertical (and horizontal?) space in tables

% define centered, left- and right-aligned columns with specified widths
\newcommand{\PreserveBackslash}[1]{\let\temp=\\#1\let\\=\temp}
\newcolumntype{C}[1]{>{\PreserveBackslash\centering}p{#1}}
\newcolumntype{L}[1]{>{\PreserveBackslash\raggedright}p{#1}}
\newcolumntype{R}[1]{>{\PreserveBackslash\raggedleft}p{#1}}

\begin{document}

\pagestyle{empty} % don't number pages

% custom title
\begin{center}
{\LARGE Express Riddler}

\vspace{0.15in}

{\Large 5 November 2021}
\end{center}


\section*{Riddle:}

As of today, The Riddler Social Network is being rebranded as $\upmu\upepsilon\uptau\upalpha$---that's mu epsilon tau alpha.
Those Greek letters really augment the brand, don't you think?

A group of 101 people join $\upmu\upepsilon\uptau\upalpha$, and each person has a random, 50 percent chance of being friends with each of the other 100 people.
Friendship is a symmetric relationship on $\upmu\upepsilon\uptau\upalpha$, so if you're friends with me, then I am also friends with you.

I pick a random person among the 101---let's suppose her name is Marcia.
On average, how many friends would you expect each of Marcia's friends to have?


\section*{Solution:}

I decided to tackle this problem via simulation.
The code is located in \texttt{social\_network.C}.

I started by creating a $101\times101$ array.
This array was randomly filled with 0s or 1s; a 1 in the $a$th row and $b$th column means that $a$ and $b$ are friends.
For completeness, and to save computing power, I only had to randomly determine the top half of the array where $a>b$; the bottom half was then simply mirrored.
Additionally, the diagonal (where $a=b$) was filled only with 0s.

Once the array was filled, I read off the friend list from the first column, then counted the total number of 1s from each of the 1 columns in that list.
Then the friends' friend count was averaged.

After several thousand simulations, the results showed that Marcia has about 50 friends on average, while her friends have about \fcolorbox{red}{white}{\textbf{50.5}}\,.
So on average, Marcia's friends have about 0.5 more friends than her.


\end{document}