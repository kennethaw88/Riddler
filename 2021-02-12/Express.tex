\documentclass{article}

\usepackage{amsmath} % math stuff
\usepackage{amssymb} % math stuff
\usepackage{array} % equations and stuff
\usepackage{bm} % bold math
%\usepackage{booktabs} % extra table rule options
%\usepackage{caption} % suppressed table numbering; incompatible with revtex, and longtable, I think
\usepackage{comment} % comment environment
%\usepackage{enumitem} % customization of enumeration, itemize, and description
\usepackage[T1]{fontenc} % font encoding for special characters, must also use scalable font package
\usepackage[margin=0.8in]{geometry} % paper sizes and margins (but be careful not to mess up pre-defined pages)
\usepackage{graphicx} % for graphics
%\usepackage{helvet} % default font is the helvetica postscript font
\usepackage{layouts} % print units like widths
\usepackage{lipsum} % lorem ipsum filler text
\usepackage{lmodern} % scalable font?
\usepackage{longtable} % multi-page tables
\usepackage{makecell} % specify line-breaks in table cells
\usepackage{mathrsfs} % math script font
\usepackage{mhchem} % easier chemical formula
\usepackage{microtype} % allows disabling of ligatures
%\usepackage{newcent} % new century schoolbook font
\usepackage{nicefrac}
\usepackage{numprint} % print and format (large) numbers
\usepackage{parskip} % removes paragraph indentation, and adjusts paragraph skip, as well as list items
\usepackage{pdfpages} % add pdf files as pages
%\usepackage{setspace} % adjust text spacing and indents
\usepackage{siunitx} % decimal alignment
\usepackage{subfigure} % divided figures
%\usepackage{tabu} % extra table options
\usepackage{textcomp} % symbols
\usepackage{threeparttablex} % better footnotes with longtable
\usepackage{titling} % title placement
\usepackage{ulem} % strikethrough text
%\usepackage{url} % superceded by hyperref
\usepackage{verbatim} % verbatim environment
\usepackage{xcolor} % colors and color boxes
\usepackage{xspace} % commands that don't eat up white space
\usepackage{hyperref} % links and page setup; should always come last

\hypersetup{
 bookmarks=true,
 colorlinks=true,
 citecolor=blue,
 linkcolor=blue,
 urlcolor=blue,
 pdfstartview={XYZ null null 1.0} % default open view is 100%
}

\DisableLigatures[f,t]{encoding = T1} % disable ff, fi, fl, tt ligatures, without f option, it also disables -- = endash
\renewcommand{\arraystretch}{2.0} % extra vertical space in tables

% define centered, left- and right-aligned columns with specified widths
\newcommand{\PreserveBackslash}[1]{\let\temp=\\#1\let\\=\temp}
\newcolumntype{C}[1]{>{\PreserveBackslash\centering}p{#1}}
\newcolumntype{L}[1]{>{\PreserveBackslash\raggedright}p{#1}}
\newcolumntype{R}[1]{>{\PreserveBackslash\raggedleft}p{#1}}

\begin{document}

\pagestyle{empty} % don't number pages

% custom title
\begin{center}
{\LARGE Express Riddler}

\vspace{0.15in}

{\Large 12 February 2021}
\end{center}


\section*{Riddle:}

It’s the second week in our four weeks of CrossProduct™ puzzles!

This time around, there are \textit{six} three-digit numbers---each belongs in a row of the table below, with one digit per cell.
The products of the three digits of each number are shown in the rightmost column.
Meanwhile, the products of the digits in the hundreds, tens and ones places, respectively, are shown in the bottom row.

\begin{center}
\begin{tabular}{|C{0.75in}|C{0.75in}|C{0.75in}!{\vrule width 1.5pt}R{0.75in}|}
\hline
 & & & 210 \\
\hline
 & & & 144 \\
\hline
 & & & 54 \\
\hline
 & & & 135 \\
\hline
 & & & 4 \\
\hline
 & & & 49 \\
\noalign{\hrule height 2pt}
6,615 & 15,552 & 420 & \\
\hline
\end{tabular}
\end{center}

Can you find all six three-digit numbers and complete the table?

\section*{Solution:}

The riddle is essentially asking for 18 digits to be placed in a $6\times3$ table.
The first step is to decompose each number into either three or six (not-necessarily-prime) single-digit factors.
For most, there are multiple sets of possible factors.
I list these below:

\begin{center}
\setlength{\tabcolsep}{3pt}
\begin{tabular}{lllll}
210   & $=5\cdot6\cdot7$ \\
144   & $=2\cdot8\cdot9$ & $=3\cdot6\cdot8$ & $=4\cdot4\cdot9$ & $=4\cdot6\cdot6$ \\
54    & $=1\cdot6\cdot9$ & $=2\cdot3\cdot9$ & $=3\cdot3\cdot6$ \\
135   & $=3\cdot5\cdot9$
4     & $=1\cdot1\cdot4$ & $=1\cdot2\cdot2$ \\
49    & $=1\cdot7\cdot7$ \\
6615  & $=1\cdot3\cdot5\cdot7\cdot7\cdot9$ & $=3\cdot3\cdot3\cdot5\cdot7\cdot7$ \\
15552 & $=1\cdot3\cdot8\cdot8\cdot9\cdot9$ & $=1\cdot4\cdot6\cdot8\cdot9\cdot9$ & $=1\cdot6\cdot6\cdot6\cdot8\cdot9$ & $=2\cdot2\cdot6\cdot8\cdot9\cdot9$ \\
      & $=2\cdot3\cdot4\cdot8\cdot9\cdot9$ & $=2\cdot4\cdot4\cdot6\cdot9\cdot9$ & $=2\cdot4\cdot6\cdot6\cdot6\cdot9$ & $=2\cdot6\cdot6\cdot6\cdot6\cdot6$ \\
      & $=3\cdot4\cdot4\cdot4\cdot9\cdot9$ & $=3\cdot4\cdot6\cdot6\cdot6\cdot6$ \\
420   & $=1\cdot1\cdot2\cdot5\cdot6\cdot7$ & $=1\cdot1\cdot3\cdot4\cdot5\cdot7$ & $=1\cdot2\cdot2\cdot3\cdot5\cdot7$
\end{tabular}
\end{center}

The easiest first step is to notice that the last row (49) only has two possible factors: 1 and 7.
Because 7 is not a factor of 15,552, the 1 must go in the center column, with the 7s in the left and right columns.
So the sixth row is 7,1,7.

Now that the 6,615 has one of its 7s accounted for, the other 7 must come from the 210 in the first row.
The only remaining factors in 210 are 5 and 6.
Again, the 5 is not a factor of 15,552, so the middle column is 6, and the right column is 5.
The top row is thus 7,6,5.

Because one of the two factors of 5 was used for 210, the only other row that has a 5 is the fourth row (135).
Its factor of 5 must belong to the 6,615 in the first column.
In the fourth row, the other two factors are 3 and 9.
The 9 is not a factor of 420, so the 9 is in the middle column, leaving the 3 in the right column.
The fourth row is therefore 5,9,3.

At this point, several of the factorizations listed above can be eliminated.
Because the middle column (15,552) now has a 1, 6, and 9,all but two factorizations are eliminated.
Similarly, the last column (420) has a 3, 5, and 7, so that the $1\cdot1\cdot2\cdot5\cdot6\cdot7$ factorization is eliminated.
The 6,615 and 4 have no factors in common besides 1, so the 1 must go in the first column of the fifth row.
This eliminates $3\cdot3\cdot3\cdot5\cdot7\cdot7$ as a factorization of 6,615.
For the first column, the two factors not yet assigned are 3 and 9.
Thus the factorization of 144 must include one of these numbers, so $4\cdot6\cdot6$ is eliminated.

At this point, my procedure is to list out all sets of remaining factors for both the rows and columns.
There are eight positions left unassigned.
There are eighteen possible ways to factor the second, third, and fifth rows, and four ways to factor the second and third columns.
Looking at these, the only factorizations that match up are $1\cdot2\cdot2\cdot3\cdot4\cdot8\cdot9\cdot9$, $1\cdot2\cdot2\cdot3\cdot6\cdot6\cdot8\cdot9$, and $1\cdot1\cdot3\cdot4\cdot6\cdot6\cdot7\cdot9$.

Trying to place all of these factors in the table is impossible for the last two factorizations.
Only the first of the three matches can have the factors placed correctly.
The final solution is

\begin{center}
\begin{tabular}{|C{0.75in}|C{0.75in}|C{0.75in}|}
\hline
7 & 6 & 5 \\
\hline
9 & 8 & 2 \\
\hline
3 & 9 & 2 \\
\hline
5 & 9 & 3 \\
\hline
1 & 4 & 1 \\
\hline
7 & 1 & 7 \\
\hline
\end{tabular}
\end{center}


\end{document}