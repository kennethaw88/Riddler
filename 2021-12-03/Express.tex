\documentclass{article}

\usepackage{amsmath} % math stuff
\usepackage{amssymb} % math stuff
\usepackage{array} % equations and stuff
\usepackage{bm} % bold math
%\usepackage{booktabs} % extra table rule options
%\usepackage{caption} % suppressed table numbering; incompatible with revtex, and longtable, I think
\usepackage{comment} % comment environment
%\usepackage{enumitem} % customization of enumeration, itemize, and description
\usepackage[T1]{fontenc} % font encoding for special characters, must also use scalable font package
\usepackage[margin=0.8in]{geometry} % paper sizes and margins (but be careful not to mess up pre-defined pages)
\usepackage{graphicx} % for graphics
%\usepackage{helvet} % default font is the helvetica postscript font
\usepackage[utf8]{inputenc} % special characters in tex input
\usepackage{layouts} % print units like widths
\usepackage{lipsum} % lorem ipsum filler text
\usepackage{lmodern} % scalable font?
\usepackage{longtable} % multi-page tables
\usepackage{makecell} % specify line-breaks in table cells
\usepackage{mathrsfs} % math script font
\usepackage{mhchem} % easier chemical formula
\usepackage{microtype} % allows disabling of ligatures
\usepackage{multicol} % multicolumns
%\usepackage{newcent} % new century schoolbook font
\usepackage{nicefrac}
\usepackage{numprint} % print and format (large) numbers
\usepackage{parskip} % removes paragraph indentation, and adjusts paragraph skip, as well as list items
\usepackage{pdfpages} % add pdf files as pages
%\usepackage{setspace} % adjust text spacing and indents
\usepackage{siunitx} % decimal alignment
\usepackage{subfigure} % divided figures
%\usepackage{tabu} % extra table options
\usepackage{textcomp} % symbols
\usepackage{threeparttablex} % better footnotes with longtable
\usepackage{titling} % title placement
\usepackage{ulem} % strikethrough text
\usepackage{upgreek} % upright Greek letters
%\usepackage{url} % superceded by hyperref
\usepackage{verbatim} % verbatim environment
\usepackage{xcolor} % colors and color boxes
\usepackage{xspace} % commands that don't eat up white space
\usepackage{hyperref} % links and page setup; should always come last

\hypersetup{
 bookmarks=true,
 colorlinks=true,
 citecolor=blue,
 linkcolor=blue,
 urlcolor=blue,
 pdfstartview={XYZ null null 1.0} % default open view is 100%
}

\DisableLigatures[f,t]{encoding = T1} % disable ff, fi, fl, tt ligatures; without options, it also disables -- = endash
\renewcommand{\arraystretch}{1.0} % extra vertical (and horizontal?) space in tables

% define centered, left- and right-aligned columns with specified widths
\newcommand{\PreserveBackslash}[1]{\let\temp=\\#1\let\\=\temp}
\newcolumntype{C}[1]{>{\PreserveBackslash\centering}p{#1}}
\newcolumntype{L}[1]{>{\PreserveBackslash\raggedright}p{#1}}
\newcolumntype{R}[1]{>{\PreserveBackslash\raggedleft}p{#1}}

\begin{document}

\pagestyle{empty} % don't number pages

% custom title
\begin{center}
{\LARGE Express Riddler}

\vspace{0.15in}

{\Large 3 December 2021}
\end{center}


\section*{Riddle:}

Tonight marks the sixth night of Hanukkah, which means it's time for some more Menorah Math!

I have a most peculiar menorah.
Like most menorahs, it has nine total candles---a central candle, called the shamash, four to the left of the shamash and another four to the right.
But unlike most menorahs, the eight candles on either side of the shamash are numbered.
The two candles adjacent to the shamash are both ``1,'' the next two candles out from the shamash are ``2,'' the next pair are ``3,'' and the outermost pair are ``4.''

The shamash is always lit.
How many ways are there to light the remaining eight candles so that sums on either side of the menorah are ``balanced''?
(For example, one such way is to light candles 1 and 4 on one side and candles 2 and 3 on the other side.
In this case, the sums on both sides are 5, so the menorah is balanced.)


\section*{Solution:}

Using the numbers 1--4, the possible sums are 0--10.
With this range, it is possible to list out all of the possible balanced sums:

\vspace{0.1in}
\begin{center}
\begin{tabular}{ccccccccccc}
0 & 1 & 2 & 3 & 4 & 5 & 6 & 7 & 8 & 9 & 10 \\
\hline
(empty) & 1-1 & 2-2 & 1,2-1,2 & 1,3-1,3 & 1,4-1,4 & 2,4-2,4 & 3,4-3,4 & 1,3,4-1,3,4 & 2,3,4-2,3,4 & 1,2,3,4-1,2,3,4 \\
 & & & 1,2-3 & 1,3-4 & 1,4-2,3 & 2,4-1,2,3 & 3,4-1,2,4 & & & \\
 & & & 3-1,2 & 4-1,3 & 2,3-1,4 & 1,2,3-2,4 & 1,2,4-3,4 & & &\\
 & & & 3-3 & 4-4 & 2,3-2,3 & 1,2,3-1,2,3 & 1,2,4-1,2,4 & & &
\end{tabular}
\end{center}
\vspace{0.1in}

Thus, there are \fcolorbox{red}{white}{\textbf{26}} possible balanced sums.


\end{document}