\documentclass{article}

\usepackage{amsmath} % math stuff
\usepackage{amssymb} % math stuff
\usepackage{array} % equations and stuff
\usepackage{bm} % bold math
%\usepackage{booktabs} % extra table rule options
%\usepackage{caption} % suppressed table numbering; incompatible with revtex, and longtable, I think
\usepackage{comment} % comment environment
%\usepackage{enumitem} % customization of enumeration, itemize, and description
\usepackage[T1]{fontenc} % font encoding for special characters, must also use scalable font package
\usepackage[margin=0.8in]{geometry} % paper sizes and margins (but be careful not to mess up pre-defined pages)
\usepackage{graphicx} % for graphics
%\usepackage{helvet} % default font is the helvetica postscript font
\usepackage{layouts} % print units like widths
\usepackage{lipsum} % lorem ipsum filler text
\usepackage{lmodern} % scalable font?
\usepackage{longtable} % multi-page tables
\usepackage{makecell} % specify line-breaks in table cells
\usepackage{mathrsfs} % math script font
\usepackage{mhchem} % easier chemical formula
\usepackage{microtype} % allows disabling of ligatures
%\usepackage{newcent} % new century schoolbook font
\usepackage{nicefrac}
\usepackage{numprint} % print and format (large) numbers
\usepackage{parskip} % removes paragraph indentation, and adjusts paragraph skip, as well as list items
\usepackage{pdfpages} % add pdf files as pages
%\usepackage{setspace} % adjust text spacing and indents
\usepackage{siunitx} % decimal alignment
\usepackage{subfigure} % divided figures
%\usepackage{tabu} % extra table options
\usepackage{textcomp} % symbols
\usepackage{threeparttablex} % better footnotes with longtable
\usepackage{titling} % title placement
\usepackage{ulem} % strikethrough text
%\usepackage{url} % superceded by hyperref
\usepackage{verbatim} % verbatim environment
\usepackage{xcolor} % colors and color boxes
\usepackage{xspace} % commands that don't eat up white space
\usepackage{hyperref} % links and page setup; should always come last

\hypersetup{
 bookmarks=true,
 colorlinks=true,
 citecolor=blue,
 linkcolor=blue,
 urlcolor=blue,
 pdfstartview={XYZ null null 1.0} % default open view is 100%
}

\DisableLigatures[f,t]{encoding = T1} % disable ff, fi, fl, tt ligatures; without options, it also disables -- = endash
\renewcommand{\arraystretch}{1.0} % extra vertical (and horizontal?) space in tables

% define centered, left- and right-aligned columns with specified widths
\newcommand{\PreserveBackslash}[1]{\let\temp=\\#1\let\\=\temp}
\newcolumntype{C}[1]{>{\PreserveBackslash\centering}p{#1}}
\newcolumntype{L}[1]{>{\PreserveBackslash\raggedright}p{#1}}
\newcolumntype{R}[1]{>{\PreserveBackslash\raggedleft}p{#1}}

\begin{document}

\pagestyle{empty} % don't number pages

% custom title
\begin{center}
{\LARGE Express Riddler}

\vspace{0.15in}

{\Large 9 April 2021}
\end{center}


\section*{Riddle:}

You have two 16-ounce cups---cup A and cup B.
Both cups initially have 8 ounces of water in them.

You take half of the water in cup A and pour it into cup B.
Then, you take half of the water in cup B and pour it back into cup A.
You do this again.
And again.
And again.
And then many, many, many more times---always pouring half the contents of A into B, and then half of B back into A.

When you finally pause for a breather, what fraction of the total water is in cup A?

\textit{Extra credit}: Now suppose both cups initially have somewhere between 0 and 8 ounces of water in them.
You don't know the precise amount in each cup, but you know that both cups are not empty.
Again, you pour half the water from cup A into cup B, and then half from cup B back to A.
You do this many, many times.
When you again finally pause for a breather, what fraction of the total water is in cup A?



\section*{Solution:}

It's pretty intuitive that as the number of pours increases, the amount of water being transferred approaches an equilibrium value.
In this limit, each pour just moves the same water back and forth each time, while the remaining water in each cup never changes.
And of course, the amount being transferred each time is equal to both the leftover amount, and the amount in the other cup (before pouring).
Calling this amount $x$, then in the end cup B is left with $x$, and cup A has $2x$.
Then the total amount of water is $3x=1$, so that $x=\nicefrac{1}{3}$.
So in the end cup A has
\fcolorbox{red}{white}{\bf \nicefrac{2}{3}}\,.
This solution does not depend on the intial values, so it applies to both the original riddle and the extra credit.

\end{document}