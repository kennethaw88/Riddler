\documentclass{article}


\usepackage{amsmath} % math stuff
\usepackage{amssymb} % math stuff
\usepackage{array} % equations and stuff
\usepackage{bm} % bold math
%\usepackage{caption} % suppressed table numbering; incompatible with revtex, and longtable, I think
\usepackage{comment} % comment environment
%\usepackage{enumitem} % customization of enumeration, itemize, and description
\usepackage[T1]{fontenc} % font encoding for special characters, must also use scalable font package
\usepackage[margin=0.8in]{geometry} % paper sizes and margins (but be careful not to mess up pre-defined pages)
\usepackage{graphicx} % for graphics
%\usepackage{helvet} % default font is the helvetica postscript font
\usepackage{lipsum} % lorem ipsum filler text
\usepackage{lmodern} % scalable font?
\usepackage{longtable} % multi-page tables
\usepackage{mathrsfs} % math script font
\usepackage{mhchem} % easier chemical formula
\usepackage{microtype} % allows disabling of ligatures
%\usepackage{newcent} % new century schoolbook font
\usepackage{nicefrac}
\usepackage{parskip} % removes paragraph indentation, and adjusts paragraph skip, as well as list items
%\usepackage{setspace} % adjust text spacing and indents
\usepackage{siunitx} % decimal alignment
\usepackage{subfigure} % divided figures
%\usepackage{tabu} % extra table options
\usepackage{textcomp} % symbols
\usepackage{threeparttablex} % better footnotes with longtable
\usepackage{titling} % title placement
\usepackage{ulem} % strikethrough text
%\usepackage{url} % superceded by hyperref
\usepackage{verbatim} % verbatim environment
\usepackage{xcolor} % colors and color boxes
\usepackage{xspace} % commands that don't eat up white space
\usepackage{hyperref} % links and page setup; should always come last

\hypersetup{
	bookmarks=true,
	colorlinks=true,
	citecolor=blue,
	linkcolor=blue,
	urlcolor=blue,
	pdfstartview={XYZ null null 1.0} % default open view is 100%
}

\DisableLigatures[f]{encoding = *, family = * } % disable ff, fi, fl ligatures, without f option, it also disables -- = endash
\renewcommand{\arraystretch}{1.25} % extra vertical space in tables

\begin{document}

\pagestyle{empty} % don't number pages

% custom title
\begin{center}
{\LARGE Classic Riddler}

\vspace{0.15in}

{\Large 12 June 2020}
\end{center}


\section*{Riddle:}

There's a technique for rolling dice called ``bowling,'' in which you place your index finger and thumb on two opposite sides of the die and roll it along the table.
If done correctly, the die will never land on the faces on which you were holding the die, leaving you with a 25 percent chance of landing on each of the remaining four faces.

You'd like to apply this technique to improve your chances of winning a simplified game of craps, in which your goal is simply to roll a 7 or 11 using two dice.
With a standard rolling technique, your chances of rolling a 7 or 11 are 2/9, or about 22.2 percent.

Now suppose you're using your bowling technique, and you roll the dice one at a time (i.e., you know the outcome of the first die before rolling the second).
If you play to maximize your chances of rolling a 7 or 11, what will be your chances of winning?

\textit{Extra credit}: Suppose you get one point for rolling a 7 or 11, but now you lose a point for rolling a 2, 3 or 12. With a standard rolling technique, you'd average 1/9 of a point. But if you ``bowl'' to maximize your expected score, what will that average be?

\section*{Solution:}

When using the bowling technique, there are three possible ways to roll, for each of the three pairs of sides on a standard die: hold 1-6, leaving 2-3-4-5; hold 2-5, leaving 1-3-4-6; and hold 3-4, leaving 1-2-5-6.
Depending on the result of the first roll, each of these rolls has a distinct set of results.
I have listed these results in the following tables, which show possible sums from any given first roll for each of the bowled rolls.
Of course, they are all redundant to the overall $6\times6$ sums for any two dice rolls, but breaking it up this way visually makes it easier to mentally separate the results.

\vspace{0.1in}

\begin{center}
\begin{tabular}{ c|c|c|c|c| }
\multicolumn{1}{c}{} & \multicolumn{1}{c}{1} & \multicolumn{1}{c}{2} & \multicolumn{1}{c}{5} & \multicolumn{1}{c}{6} \\
\cline{2-5}
1 & 2 & 3 & 6  & 7 \\
\cline{2-5}
2 & 3 & 4 & 7  & 8 \\
\cline{2-5}
3 & 4 & 5 & 8  & 9 \\
\cline{2-5}
4 & 5 & 6 & 9  & 10 \\
\cline{2-5}
5 & 6 & 7 & 10 & 11 \\
\cline{2-5}
6 & 7 & 8 & 11 & 12 \\
\cline{2-5}
\end{tabular}
\qquad
\begin{tabular}{ c|c|c|c|c| }
\multicolumn{1}{c}{} & \multicolumn{1}{c}{1} & \multicolumn{1}{c}{3} & \multicolumn{1}{c}{4} & \multicolumn{1}{c}{6} \\
\cline{2-5}
1 & 2 & 4 & 5  & 7 \\
\cline{2-5}
2 & 3 & 5 & 6  & 8 \\
\cline{2-5}
3 & 4 & 6 & 7  & 9 \\
\cline{2-5}
4 & 5 & 7 & 8  & 10 \\
\cline{2-5}
5 & 6 & 8 & 9  & 11 \\
\cline{2-5}
6 & 7 & 9 & 10 & 12 \\
\cline{2-5}
\end{tabular}
\qquad
\begin{tabular}{ c|c|c|c|c| }
\multicolumn{1}{c}{} & \multicolumn{1}{c}{2} & \multicolumn{1}{c}{3} & \multicolumn{1}{c}{4} & \multicolumn{1}{c}{5} \\
\cline{2-5}
1 & 3 & 4 & 5  & 6 \\
\cline{2-5}
2 & 4 & 5 & 6  & 7 \\
\cline{2-5}
3 & 5 & 6 & 7  & 8 \\
\cline{2-5}
4 & 6 & 7 & 8  & 9 \\
\cline{2-5}
5 & 7 & 8 & 9  & 10 \\
\cline{2-5}
6 & 8 & 9 & 10 & 11 \\
\cline{2-5}
\end{tabular}
\end{center}

\vspace{0.1in}

Now for any given first roll, we can choose the second roll that maximizes the chance of getting a 7 or 11.
For example, with a first roll of 1, rolling 1-2-5-6 or 1-3-4-6 next gives a chance of winning of 0.25.
I summarize the ideal strategy and subsequent winning probability below:

\vspace{0.1in}
\begin{center}
\begin{tabular}{c|c|c}
\multicolumn{1}{c}{First roll} & \multicolumn{1}{c}{Strategy} & \multicolumn{1}{c}{Winning probability} \\
\hline
1 & 1-2-5-6 or 1-3-4-6 & 0.25 \\
2 & 1-2-5-6 or 2-3-4-5 & 0.25 \\
3 & 1-3-4-6 or 2-3-4-5 & 0.25 \\
4 & 1-3-4-6 or 2-3-4-5 & 0.25 \\
5 & 1-2-5-6            & 0.5 \\
6 & 1-2-5-6            & 0.5 \\
\end{tabular}
\end{center}
\vspace{0.1in}

There has to be a strategy for the first roll, of course.
This is just the average winning percentage for each of the three types of rolls.
Thus, a 1-2-5-6 roll has a winning probability of 0.375, a 1-3-4-6 roll has a winning probability of 0.3125, and a 2-3-4-5 roll has a winning probability of 0.3125.
So the best strategy is to roll a 1-2-5-6 on both rolls, with a winning probability of
\fcolorbox{red}{white}{\bf 0.375}\,.
Interestingly, in this particular case, the strategy of seeing the first roll doesn't actually help; the probability of winning is the same if the dice are rolled simultaneously.

For the extra credit, the only difference is that the second table is modified.
Instead of the last column representing winning probability, it simply becomes average score.
I show it below:

\vspace{0.1in}
\begin{center}
\begin{tabular}{c|c|c}
\multicolumn{1}{c}{First roll} & \multicolumn{1}{c}{Strategy} & \multicolumn{1}{c}{Average score} \\
\hline
1 & 1-3-4-6            & 0 \\
2 & 2-3-4-5            & 0.25 \\
3 & 1-3-4-6 or 2-3-4-5 & 0.25 \\
4 & 1-3-4-6 or 2-3-4-5 & 0.25 \\
5 & 1-2-5-6            & 0.5 \\
6 & 1-3-4-6 or 2-3-4-5 & 0.25 \\
\end{tabular}
\end{center}
\vspace{0.1in}

Now, a 1-2-5-6 roll has an average score of 0.25, a 1-3-4-6 roll has an average score of 0.1875, and a 2-3-4-5 roll has an average score of 0.3125.
So the new best strategy is to start with a 2-3-4-5 roll, and the overall average score is
\fcolorbox{red}{white}{\bf 0.3125}\,.
Now, in fact, the strategy for the second roll does matter, so seeing the first roll is important.


\end{document}