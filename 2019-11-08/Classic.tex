\documentclass{article}


\usepackage{amsmath} % math stuff
\usepackage{amssymb} % math stuff
\usepackage{array} % equations and stuff
\usepackage{bm} % bold math
%\usepackage{caption} % suppressed table numbering; incompatible with revtex, and longtable, I think
\usepackage{comment} % comment environment
%\usepackage{enumitem} % customization of enumeration, itemize, and description
\usepackage[T1]{fontenc} % font encoding for special characters, must also use scalable font package
\usepackage[margin=0.8in]{geometry} % paper sizes and margins (but be careful not to mess up pre-defined pages)
\usepackage{graphicx} % for graphics
%\usepackage{helvet} % default font is the helvetica postscript font
\usepackage{lipsum} % lorem ipsum filler text
\usepackage{lmodern} % scalable font?
\usepackage{longtable} % multi-page tables
\usepackage{mathrsfs} % math script font
\usepackage{mhchem} % easier chemical formula
\usepackage{microtype} % allows disabling of ligatures
%\usepackage{newcent} % new century schoolbook font
\usepackage{parskip} % removes paragraph indentation, and adjusts paragraph skip, as well as list items
%\usepackage{setspace} % adjust text spacing and indents
\usepackage{siunitx} % decimal alignment
\usepackage{subfigure} % divided figures
%\usepackage{tabu} % extra table options
\usepackage{textcomp} % symbols
\usepackage{threeparttablex} % better footnotes with longtable
\usepackage{titling} % title placement
%\usepackage{url} % superceded by hyperref
\usepackage{verbatim} % verbatim environment
\usepackage{xcolor} % colors and color boxes
\usepackage{xspace} % commands that don't eat up white space
\usepackage{hyperref} % links and page setup; should always come last

\hypersetup{
	bookmarks=true,
	colorlinks=true,
	citecolor=blue,
	linkcolor=blue,
	urlcolor=blue,
	pdfstartview={XYZ null null 1.0} % default open view is 100%
}

\DisableLigatures[f]{encoding = *, family = * } % disable ff, fi, fl ligatures, without f option, it also disables -- = endash

% repeated symbols
\makeatletter
\newcount\my@repeat@count
\newcommand{\myrepeat}[2]{%
  \begingroup
  \my@repeat@count=\z@
  \@whilenum\my@repeat@count<#1\do{#2\advance\my@repeat@count\@ne}%
  \endgroup
}
\makeatother

\begin{document}

\pagestyle{empty} % don't number pages

% custom title
\begin{center}
{\LARGE Classic Riddler}

\vspace{0.15in}

{\Large 8 November 2019}
\end{center}


\section*{Riddle:}

Now that Halloween has come and gone, your chances of getting free candy have similarly disappeared.
Desperate for sugar, you wander into the candy store, where three kinds of candy are being sold: Almond Soys (yummy, sounds vegan!), Butterflingers and Candy Kernels.

You'd like to buy at least one candy and at most 100, but you don't care precisely how many you get of each or how many you get overall.
So you might buy one of each, or you might buy 30 Almond Soys, six Butterflingers and 64 Candy Kernels.
As long as you have somewhere between one and 100 candies, you'll leave the store happy.

But as a member of Riddler Nation, you can't help but wonder: How many distinct ways are there for you to make your candy purchase?


\section*{Solution:}

Another way of stating this problem is how many ways are there to divide up 100 items among 4 groups, and in this case, one group is just ``nothing''.
So for however many candies below 100 are bought, the remaining items become the ``nothing'' group.
I wasn't sure how exactly to frame this problem, since combinatorics can get really involved and confusing.
So I basically set up the problem for numbers smaller than 100 and made exhaustive lists to get the general pattern.
Suppose you are buying $n$ candies, still with 4 possible groups.
Letting $C(n)$ be the numbers of ways to buy $n$ candies, I have for small $n$

\vspace{0.2in}

$n=0$, $C(n)=1$, trivially.

\vspace{0.2in}

$n=1$, $C(n)=4$, also somewhat trivially
\begin{center}
\begin{tabular}{c c c c}
0 & 0 & 0 & 1 \\
0 & 0 & 1 & 0 \\
0 & 1 & 0 & 0 \\
1 & 0 & 0 & 0
\end{tabular}
\end{center}

\vspace{0.2in}

$n=2$, $C(n)=10$
\begin{center}
\begin{tabular}{c c c c}
0 & 0 & 0 & 2 \\
0 & 0 & 1 & 1 \\
0 & 0 & 2 & 0 \\
0 & 1 & 0 & 1 \\
0 & 1 & 1 & 0
\end{tabular}
\qquad
\begin{tabular}{c c c c}
0 & 2 & 0 & 0 \\
1 & 0 & 0 & 1 \\
1 & 0 & 1 & 0 \\
1 & 1 & 0 & 0 \\
2 & 0 & 0 & 0
\end{tabular}
\end{center}

\vspace{0.2in}

$n=3$, $C(n)=20$
\begin{center}
\begin{tabular}{c c c c}
0 & 0 & 0 & 3 \\
0 & 0 & 1 & 2 \\
0 & 0 & 2 & 1 \\
0 & 0 & 3 & 1 \\
0 & 1 & 0 & 2
\end{tabular}
\qquad
\begin{tabular}{c c c c}
0 & 1 & 1 & 1 \\
0 & 1 & 2 & 0 \\
0 & 2 & 0 & 1 \\
0 & 2 & 1 & 0 \\
0 & 3 & 0 & 0
\end{tabular}
\qquad
\begin{tabular}{c c c c}
1 & 0 & 0 & 2 \\
1 & 0 & 1 & 1 \\
1 & 0 & 2 & 0 \\
1 & 1 & 0 & 1 \\
1 & 1 & 1 & 0
\end{tabular}
\qquad
\begin{tabular}{c c c c}
1 & 2 & 0 & 0 \\
2 & 0 & 0 & 1 \\
2 & 0 & 1 & 0 \\
2 & 1 & 0 & 0 \\
3 & 0 & 0 & 0
\end{tabular}
\end{center}

\newpage

And just in case the pattern isn't clear, $n=4$, $C(n)=35$
\begin{center}
\begin{tabular}{c c c c}
0 & 0 & 0 & 4 \\
0 & 0 & 1 & 3 \\
0 & 0 & 2 & 2 \\
0 & 0 & 3 & 1 \\
0 & 0 & 4 & 0 \\
0 & 1 & 0 & 3 \\
0 & 1 & 1 & 2
\end{tabular}
\qquad
\begin{tabular}{c c c c}
0 & 1 & 2 & 1 \\
0 & 1 & 3 & 0 \\
0 & 2 & 0 & 2 \\
0 & 2 & 1 & 1 \\
0 & 2 & 2 & 0 \\
0 & 3 & 0 & 1 \\
0 & 3 & 1 & 0
\end{tabular}
\qquad
\begin{tabular}{c c c c}
0 & 4 & 0 & 0 \\
1 & 0 & 0 & 3 \\
1 & 0 & 1 & 2 \\
1 & 0 & 2 & 1 \\
1 & 0 & 3 & 0 \\
1 & 1 & 0 & 2 \\
1 & 1 & 1 & 1
\end{tabular}
\qquad
\begin{tabular}{c c c c}
1 & 1 & 2 & 0 \\
1 & 2 & 0 & 1 \\
1 & 2 & 1 & 0 \\
1 & 3 & 0 & 0 \\
2 & 0 & 0 & 2 \\
2 & 0 & 1 & 1 \\
2 & 0 & 2 & 0
\end{tabular}
\qquad
\begin{tabular}{c c c c}
2 & 1 & 0 & 1 \\
2 & 1 & 1 & 0 \\
2 & 2 & 0 & 0 \\
3 & 0 & 0 & 1 \\
3 & 0 & 1 & 0 \\
3 & 1 & 0 & 0 \\
4 & 0 & 0 & 0
\end{tabular}
\end{center}

\vspace{0.2in}

Now it is clear that the pattern is a sum of triangular numbers, i.e., $1=1$, $4=1+3$, $10=1+3+6$, etc., so that I have
\[
C(n)=\sum\limits_{m=1}^{n+1}Tri_{m}=\sum\limits_{m=1}^{n+1}\frac{m(m+1)}{2}
\]

But then I started thinking: geometrically, triangular numbers are just sums of the whole numbers, and a triangle is formed by stacking them.
Similarly, stacking triangles actually gives tetrahedra, so in principle, I am just re-calculating the tetrahedral numbers.
Sure enough,
\[
Tet_{n}=\sum\limits_{m=1}^{n}Tri_{m}=\sum\limits_{m=1}^{n}\frac{m(m+1)}{2}=\frac{n(n+1)(n+2)}{6}
\]

So now I get $C(100)=Tet_{101}=176851$.
But wait!
This number includes the possibility of having all 100 candies in the ``nothing'' group.
So I reduce this by 1 to get the final solution of
\fcolorbox{red}{white}{\bf 176,850}\,.


\end{document}