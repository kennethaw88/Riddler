\documentclass{article}


\usepackage{amsmath} % math stuff
\usepackage{amssymb} % math stuff
\usepackage{array} % equations and stuff
\usepackage{bm} % bold math
%\usepackage{caption} % suppressed table numbering; incompatible with revtex, and longtable, I think
\usepackage{comment} % comment environment
%\usepackage{enumitem} % customization of enumeration, itemize, and description
\usepackage[T1]{fontenc} % font encoding for special characters, must also use scalable font package
\usepackage[margin=0.8in]{geometry} % paper sizes and margins (but be careful not to mess up pre-defined pages)
\usepackage{graphicx} % for graphics
%\usepackage{helvet} % default font is the helvetica postscript font
\usepackage{lipsum} % lorem ipsum filler text
\usepackage{lmodern} % scalable font?
\usepackage{longtable} % multi-page tables
\usepackage{mathrsfs} % math script font
\usepackage{mhchem} % easier chemical formula
\usepackage{microtype} % allows disabling of ligatures
%\usepackage{newcent} % new century schoolbook font
\usepackage{parskip} % removes paragraph indentation, and adjusts paragraph skip, as well as list items
%\usepackage{setspace} % adjust text spacing and indents
\usepackage{siunitx} % decimal alignment
\usepackage{subfigure} % divided figures
%\usepackage{tabu} % extra table options
\usepackage{textcomp} % symbols
\usepackage{threeparttablex} % better footnotes with longtable
\usepackage{titling} % title placement
%\usepackage{url} % superceded by hyperref
\usepackage{verbatim} % verbatim environment
\usepackage{xcolor} % colors and color boxes
\usepackage{xspace} % commands that don't eat up white space
\usepackage{hyperref} % links and page setup; should always come last

\hypersetup{
	bookmarks=true,
	colorlinks=true,
	citecolor=blue,
	linkcolor=blue,
	urlcolor=blue,
	pdfstartview={XYZ null null 1.0} % default open view is 100%
}

\DisableLigatures[f]{encoding = *, family = * } % disable ff, fi, fl ligatures, without f option, it also disables -- = endash

% repeated symbols
\makeatletter
\newcount\my@repeat@count
\newcommand{\myrepeat}[2]{%
  \begingroup
  \my@repeat@count=\z@
  \@whilenum\my@repeat@count<#1\do{#2\advance\my@repeat@count\@ne}%
  \endgroup
}
\makeatother

\begin{document}

\pagestyle{empty} % don't number pages

% custom title
\begin{center}
{\LARGE Riddler Express}

\vspace{0.15in}

{\Large 8 November 2019}
\end{center}


\section*{Riddle:}

Suppose I asked you to generate the biggest number you could using exactly three nines.
Specifically, you can add, subtract, multiply, divide, exponentiate or write them side-by-side.
Given this challenge, $9\times9\times9$ is a pretty good start---it equals 729.
Better yet is just writing the nines side-by-side, giving you 999.
The biggest number is $9^{9^{9}}$, which equals $9^{387420489}$.
If you were to write one digit of that number every second, it would take you more than a decade to write the whole thing.

Now let’s up the challenge: What’s the biggest number you can generate using exactly four threes?


\section*{Solution:}

Since the most powerful function I can use is exponentiation, that will obviously be in the solution.
But which combination of threes is best?
Some first guesses are:
\begin{align*}
3^{3^{3^{3}}} \\
3^{333} \\
33^{33} \\
333^{3}
\end{align*}
Very quickly, I can see that $333^{3}$ is much too small---I can calculate that by hand!
I definitely need a solution with multiple steps (really, tetration).
Now I also note that $33>3^{3}=27$, so I can reduce the possibilities to
\begin{align*}
3^{3^{33}} \\
3^{333} \\
33^{33} \\
\end{align*}
From here, I can calculate (online) that $3^{33}>5\times10^{15}\gg 333$, leaving only two possible answers.
But I can make a further observation that
\begin{align*}
3^{5\times10^{15}} &= \left(3^{5}\right)^{10^{15}} \\
 &> 33^{10^{15}} \\
 &\gg 33^{33}
\end{align*}
So the final answer is
\fcolorbox{red}{white}{$\bm{3^{3^{33}}}$}\,, which is such an enormously large number, I won't bother to calculate it at all.

Of course, for fun, I also wanted to see how else I could combine four threes with additional operations.
I am not using fuctions such as $\text{TREE}(n)$ or $g_{n}$, but I will use the extension of the tetration notation, Knuth's up arrows.
That way I'm only using proper operators and the number three itself.
Since I already know $33>3^{3}$, I believe the best answer I can get this way is
\[
3\uparrow^{33}3 = 3\myrepeat{33}{\uparrow}3
\]
which puts the former solution to shame.
Actually, once a single up arrow is introduced as an operator, there's no limit to how many to put, so this method is really unlimited.
An alternative way is to just use tetration proper (not pentation or anything higher), and by analogy to the real solution, the largest number produced in this way is
\[
^{^{33}3}3
\]
which, although not as large as the up-arrow solution, is still unimaginably larger than the real solution.


\end{document}