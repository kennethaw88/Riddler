\documentclass{article}

\usepackage{amsmath} % math stuff
\usepackage{amssymb} % math stuff
\usepackage{array} % equations and stuff
\usepackage{bm} % bold math
%\usepackage{caption} % suppressed table numbering; incompatible with revtex, and longtable, I think
\usepackage{comment} % comment environment
%\usepackage{enumitem} % customization of enumeration, itemize, and description
\usepackage[T1]{fontenc} % font encoding for special characters, must also use scalable font package
\usepackage[margin=0.8in]{geometry} % paper sizes and margins (but be careful not to mess up pre-defined pages)
\usepackage{graphicx} % for graphics
%\usepackage{helvet} % default font is the helvetica postscript font
\usepackage{layouts} % print units like widths
\usepackage{lipsum} % lorem ipsum filler text
\usepackage{lmodern} % scalable font?
\usepackage{longtable} % multi-page tables
\usepackage{makecell} % specify line-breaks in table cells
\usepackage{mathrsfs} % math script font
%\usepackage{mathtools} % displaystyle for cases environment (dcases)
\usepackage{mhchem} % easier chemical formula
\usepackage{microtype} % allows disabling of ligatures
%\usepackage{newcent} % new century schoolbook font
\usepackage{nicefrac}
\usepackage{parskip} % removes paragraph indentation, and adjusts paragraph skip, as well as list items
\usepackage{pdfpages} % add pdf files as pages
%\usepackage{setspace} % adjust text spacing and indents
\usepackage{siunitx} % decimal alignment
\usepackage{subfigure} % divided figures
%\usepackage{tabu} % extra table options
\usepackage{textcomp} % symbols
\usepackage{threeparttablex} % better footnotes with longtable
\usepackage{titling} % title placement
\usepackage{ulem} % strikethrough text
%\usepackage{url} % superceded by hyperref
\usepackage{verbatim} % verbatim environment
\usepackage{xcolor} % colors and color boxes
\usepackage{xspace} % commands that don't eat up white space
\usepackage{hyperref} % links and page setup; should always come last

\hypersetup{
 bookmarks=true,
 colorlinks=true,
 citecolor=blue,
 linkcolor=blue,
 urlcolor=blue,
 pdfstartview={XYZ null null 1.0} % default open view is 100%
}

\DisableLigatures[f,t]{encoding = T1} % disable ff, fi, fl, tt ligatures, without f option, it also disables -- = endash
\renewcommand{\arraystretch}{1.1} % extra vertical space in tables

\begin{document}

\pagestyle{empty} % don't number pages

% custom title
\begin{center}
{\LARGE Express Riddler}

\vspace{0.15in}

{\Large 20 November 2020}
\end{center}


\section*{Riddle:}

Depending on the year, there can be one, two or three Friday the 13ths.
Last week happened to be the second Friday the 13th of 2020.

What is the \textit{greatest} number of Friday the 13ths that can occur over the course of four consecutive calendar years?

\textit{Extra credit}: What's the greatest number of Friday the 13ths that can occur over a four-year period (i.e., a period that doesn't necessarily begin on January 1)?


\section*{Solution:}

The question (the extra credit, really) basically boils down to finding a 48-month span with the most Friday-the-13ths.
The actual particulars of what the start date are don't matter as much as what is the first month in which the 13th occurs (i.e., starting a four-year period on the 1st of a month is really the same as starting on the 2nd, or the 3rd, up through the 13th.)

There are several cases to consider.
First, a four-year period could have zero or one leap year.
For one leap year, it could happen in the first, second, third, or fourth year.
Actually, considering the general case, the 48-month period could start and end on a leap year.
But what's important is that the leap day can only occur once in the 48-month period, or not occur at all.

For the no-leap-year case, there are 12 possible months to start with, and seven permutations of weekdays, so 84 possibilities.
For the leap-year case, there are 48 possible months, so including the seven weekdays, there are 336 possibilities.
This is a manageable total number of cases, so I listed them all out in an Excel spreadsheet, which is at \texttt{Months.xlsx}\,.
It is nice that Excel can do modular arithmetic with days.

For each yearly cycle, I calculated how the days are iterated for the 13th of each month.
For example, if January 13th is a Monday, February 13th is always a Thursday.
Once each iteration was set up, I copied it for several years in a row, and for each possible weekday permutation.
Then, for each possible starting month, I counted how many times Friday occurs for the following 47 months.
The maximum number of Fridays was 9, which happens in four cases.
These are when the 48-month span starts with Wednesday April 13th, Friday May 13th, or Tuesday December 13th of a year preceding a leap year; or Friday January 13th of a leap year.

This happened for real for the 4-year periods beginning March 14 -- May 13, 1983 and beginning November 13, 1984 -- January 13, 1984.
This happens to include the full calendar years 1984--1987.
So the solution to both parts of the riddle is
\fcolorbox{red}{white}{\bf 9}\,.


\end{document}