\documentclass{article}


\usepackage{amsmath} % math stuff
\usepackage{amssymb} % math stuff
\usepackage{array} % equations and stuff
\usepackage{bm} % bold math
%\usepackage{caption} % suppressed table numbering; incompatible with revtex, and longtable, I think
\usepackage{comment} % comment environment
%\usepackage{enumitem} % customization of enumeration, itemize, and description
\usepackage[T1]{fontenc} % font encoding for special characters, must also use scalable font package
\usepackage[margin=0.8in]{geometry} % paper sizes and margins (but be careful not to mess up pre-defined pages)
\usepackage{graphicx} % for graphics
%\usepackage{helvet} % default font is the helvetica postscript font
\usepackage{lipsum} % lorem ipsum filler text
\usepackage{lmodern} % scalable font?
\usepackage{longtable} % multi-page tables
\usepackage{mathrsfs} % math script font
\usepackage{mhchem} % easier chemical formula
\usepackage{microtype} % allows disabling of ligatures
%\usepackage{newcent} % new century schoolbook font
\usepackage{parskip} % removes paragraph indentation, and adjusts paragraph skip, as well as list items
%\usepackage{setspace} % adjust text spacing and indents
\usepackage{siunitx} % decimal alignment
\usepackage{subfigure} % divided figures
%\usepackage{tabu} % extra table options
\usepackage{textcomp} % symbols
\usepackage{threeparttablex} % better footnotes with longtable
\usepackage{titling} % title placement
%\usepackage{url} % superceded by hyperref
\usepackage{verbatim} % verbatim environment
\usepackage{xcolor} % colors and color boxes
\usepackage{xspace} % commands that don't eat up white space
\usepackage{hyperref} % links and page setup; should always come last

\hypersetup{
	bookmarks=true,
	colorlinks=true,
	citecolor=blue,
	linkcolor=blue,
	urlcolor=blue,
	pdfstartview={XYZ null null 1.0} % default open view is 100%
}

\DisableLigatures[f]{encoding = *, family = * } % disable ff, fi, fl ligatures, without f option, it also disables -- = endash

%\setlength{\droptitle}{-4em} % remove huge white space above title

%\title{Riddler Express}
%\author{}
%\date{13 December 2019}

\begin{document}

\pagestyle{empty} % don't number pages

%\maketitle

% custom title
\begin{center}
{\LARGE Classic Riddler}

\vspace{0.15in}

{\Large 22 November 2019}
\end{center}


\section*{Riddle:}

Five friends with a lot in common are playing the Riddler Lottery, in which each must choose exactly five numbers from 1 to 70.
After they all picked their numbers, the first friend notices that no number was selected by two or more friends.
Unimpressed, the second friend observes that all 25 selected numbers are composite (i.e., not prime).
Not to be outdone, the third friend points out that each selected number has at least two distinct prime factors.
After some more thinking, the fourth friend excitedly remarks that the product of selected numbers on each ticket is exactly the same.
At this point, the fifth friend is left speechless.
(You can tell why all these people are friends.)

What is the product of the selected numbers on each ticket?

\textit{Extra credit}: How many \textit{different} ways could the friends have selected five numbers each so that all their statements are true?


\section*{Solution:}

This is a tough riddle.
At the first level, since there are 25 numbers that the friends chose, there are $\binom{70}{25}> 6\times10^{18}$ possible sets of numbers to choose.
Of course, many numbers can be immediately eliminated.

First, I eliminate the primes (and 1):
\begin{equation*}
1,2,3,5,7,11,13,17,19,23,29,31,37,41,43,47,53,59,61,67
\end{equation*}

Next, I can eliminate powers of single primes:
\begin{equation*}
4,8,9,16,25,27,32,49,64
\end{equation*}

That leaves 41 numbers, so we narrow the possibilities to $\binom{41}{25}> 1\times10^{11}$ possible sets.
Now comes the tricky part.
Because each friend has the same product, any prime factor that appears in one friend's ticket must appear at least five times overall.
So I now can eliminate multiples of primes $p$ where $6p>70$ (since we already removed the primes themselves):
\begin{equation*}
26,34,38,39,46,51,52,57,58,62,65,68,69
\end{equation*}

There are just 28 numbers and $\binom{28}{25}= 3{,}276$ possible sets left.
The numbers are:
\begin{equation*}
6,10,12,14,15,18,20,21,22,24,28,30,33,35,36,40,42,44,45,48,50,54,55,56,60,63,66,70
\end{equation*}

So which three remaining numbers are leftover?
I now look at the prime factorization of the total product of these numbers:
\begin{equation*}
2^{36}\cdot3^{22}\cdot5^{12}\cdot7^{8}\cdot11^{5}
\end{equation*}

The final result must have each power term be a multiple of five and therefore be a number raised to the fifth power (that number is the individual product on each ticket).
By removing three numbers, the power term of 11 will be reduced by at most three, so I cannot eliminate any multiple of 11.
Similar for seven, the power term can only be reduced by at most three, but I must reduce it by three to get to a fifth power.
So the three leftover numbers are all multiples of seven.
Again, for five, the power term can only be reduced by at most three, but I must reduce it by exactly two to get to a fifth power.
So two of the numbers must be 35 and 70, also eliminating one factor of two.
This leaves a leftover prime factorization of $3^{2}\cdot7^{1}=63$.
So the final set of 25 numbers becomes:
\begin{equation*}
6,10,12,14,15,18,20,21,22,24,28,30,33,36,40,42,44,45,48,50,54,55,56,60,66
\end{equation*}
with a total product of $3.166\ldots\times10^{36}$.
The product of each ticket is the fifth root of this number,
\fcolorbox{red}{white}{19,958,400}\,,
with a prime factorization of $2^{7}\cdot3^{4}\cdot5^{2}\cdot7^{1}\cdot11^{1}$.
I have no idea how to tackle the question of how many ways there are to pick numbers with this product.


\end{document}