\documentclass{article}

\usepackage{amsmath} % math stuff
\usepackage{amssymb} % math stuff
\usepackage{array} % equations and stuff
\usepackage{bm} % bold math
%\usepackage{caption} % suppressed table numbering; incompatible with revtex, and longtable, I think
\usepackage{comment} % comment environment
%\usepackage{enumitem} % customization of enumeration, itemize, and description
\usepackage[T1]{fontenc} % font encoding for special characters, must also use scalable font package
\usepackage[margin=0.8in]{geometry} % paper sizes and margins (but be careful not to mess up pre-defined pages)
\usepackage{graphicx} % for graphics
%\usepackage{helvet} % default font is the helvetica postscript font
\usepackage{layouts} % print units like widths
\usepackage{lipsum} % lorem ipsum filler text
\usepackage{lmodern} % scalable font?
\usepackage{longtable} % multi-page tables
\usepackage{mathrsfs} % math script font
\usepackage{mhchem} % easier chemical formula
\usepackage{microtype} % allows disabling of ligatures
%\usepackage{newcent} % new century schoolbook font
\usepackage{nicefrac}
\usepackage{parskip} % removes paragraph indentation, and adjusts paragraph skip, as well as list items
%\usepackage{setspace} % adjust text spacing and indents
\usepackage{siunitx} % decimal alignment
\usepackage{subfigure} % divided figures
%\usepackage{tabu} % extra table options
\usepackage{textcomp} % symbols
\usepackage{threeparttablex} % better footnotes with longtable
\usepackage{titling} % title placement
\usepackage{ulem} % strikethrough text
%\usepackage{url} % superceded by hyperref
\usepackage{verbatim} % verbatim environment
\usepackage{xcolor} % colors and color boxes
\usepackage{xspace} % commands that don't eat up white space
\usepackage{hyperref} % links and page setup; should always come last

\hypersetup{
	bookmarks=true,
	colorlinks=true,
	citecolor=blue,
	linkcolor=blue,
	urlcolor=blue,
	pdfstartview={XYZ null null 1.0} % default open view is 100%
}

\DisableLigatures[f]{encoding = *, family = * } % disable ff, fi, fl ligatures, without f option, it also disables -- = endash
\renewcommand{\arraystretch}{1.1} % extra vertical space in tables

\begin{document}

\pagestyle{empty} % don't number pages

% custom title
\begin{center}
{\LARGE Express Riddler}

\vspace{0.15in}

{\Large 9 October 2020}
\end{center}


\section*{Riddle:}

Every weekend, I drive into town for contactless curbside pickup at a local restaurant.
Across the street from the restaurant are six parking spots, lined up in a row.

While I \textit{can} parallel park, it's definitely not my preference.
No parallel parking is required when the rearmost of the six spots is available or when there are two consecutive open spots.
If there is a random arrangement of cars currently occupying four of the six spots, what's the probability that I will have to parallel park?

\section*{Solution:}

This problem is small enough that it is possible to enumerate all arrangements of cars.
There are only $\binom{6}{4}=15$ ways to arrange four cars in six sponts, which are listed below:

\begin{center}
\begin{tabular}{c c c}
1-2-3-4 & 1-2-3-5 & 1-2-3-6 \\
1-2-4-5 & 1-2-4-6 & 1-2-5-6 \\
1-3-4-5 & 1-3-4-6 & 1-3-5-6 \\
1-4-5-6 & 2-3-4-5 & 2-3-4-6 \\
2-3-5-6 & 2-4-5-6 & 3-4-5-6 \\
\end{tabular}
\end{center}

Equivalently, there are 15 ways to have two open spots among six:

\begin{center}
\begin{tabular}{c c c}
1-2 & 1-3 & 1-4 \\
1-5 & 1-6 & 2-3 \\
2-4 & 2-5 & 2-6 \\
3-4 & 3-5 & 3-6 \\
4-5 & 4-6 & 5-6 \\
\end{tabular}
\end{center}

Looking at these, there are six arrangements (1-3, 1-4, 1-5, 2-4, 2-5, 3-5) in which both the sixth spot is filled and there are no consecutive spots open.
Thus the solution is
\fcolorbox{red}{white}{$\bm{{\nicefrac{6}{15}=0.4}}$}\,.



\end{document}