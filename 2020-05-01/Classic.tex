\documentclass{article}


\usepackage{amsmath} % math stuff
\usepackage{amssymb} % math stuff
\usepackage{array} % equations and stuff
\usepackage{bm} % bold math
%\usepackage{caption} % suppressed table numbering; incompatible with revtex, and longtable, I think
\usepackage{comment} % comment environment
%\usepackage{enumitem} % customization of enumeration, itemize, and description
\usepackage[T1]{fontenc} % font encoding for special characters, must also use scalable font package
\usepackage[margin=0.8in]{geometry} % paper sizes and margins (but be careful not to mess up pre-defined pages)
\usepackage{graphicx} % for graphics
%\usepackage{helvet} % default font is the helvetica postscript font
\usepackage{lipsum} % lorem ipsum filler text
\usepackage{lmodern} % scalable font?
\usepackage{longtable} % multi-page tables
\usepackage{mathrsfs} % math script font
\usepackage{mhchem} % easier chemical formula
\usepackage{microtype} % allows disabling of ligatures
%\usepackage{newcent} % new century schoolbook font
\usepackage{nicefrac}
\usepackage{parskip} % removes paragraph indentation, and adjusts paragraph skip, as well as list items
%\usepackage{setspace} % adjust text spacing and indents
\usepackage{siunitx} % decimal alignment
\usepackage{subfigure} % divided figures
%\usepackage{tabu} % extra table options
\usepackage{textcomp} % symbols
\usepackage{threeparttablex} % better footnotes with longtable
\usepackage{titling} % title placement
\usepackage{ulem} % strikethrough text
%\usepackage{url} % superceded by hyperref
\usepackage{verbatim} % verbatim environment
\usepackage{xcolor} % colors and color boxes
\usepackage{xspace} % commands that don't eat up white space
\usepackage{hyperref} % links and page setup; should always come last

\hypersetup{
	bookmarks=true,
	colorlinks=true,
	citecolor=blue,
	linkcolor=blue,
	urlcolor=blue,
	pdfstartview={XYZ null null 1.0} % default open view is 100%
}

\DisableLigatures[f]{encoding = *, family = * } % disable ff, fi, fl ligatures, without f option, it also disables -- = endash
\renewcommand{\arraystretch}{1} % extra vertical space in tables

\begin{document}

\pagestyle{empty} % don't number pages

% custom title
\begin{center}
{\LARGE Classic Riddler}

\vspace{0.15in}

{\Large 1 May 2020}
\end{center}


\section*{Riddle:}

You are locked in the dungeon of a faraway castle with three fellow prisoners (i.e., there are four prisoners in total), each in a separate cell with no means of communication.
But it just so happens that all of you are logicians (of course).

To entertain themselves, the guards have decided to give you all a single chance for immediate release.
Each prisoner will be given a fair coin, which can either be fairly flipped one time or returned to the guards without being flipped.
If all flipped coins come up heads, you will all be set free!
But if any of the flipped coins comes up tails, or if no one chooses to flip a coin, you will all be doomed to spend the rest of your lives in the castle's dungeon.

The only tools you and your fellow prisoners have to aid you are random number generators, which will give each prisoner a random number, uniformly and independently chosen between zero and one.

What are your chances of being released?

\textit{Extra credit}: Instead of four prisoners, suppose there are N prisoners. Now what are your chances of being released?

\section*{Solution:}

Intuitively, this problem requires the balancing of two needs.
The first need is the have as few prisoners as possible flip a coin; more coin flipping increases the chances of tails.
The other is the need to avoid having nobody flip a coin at all, since this eliminates any possibility of release.
My first thought was simply to maximize the individual probability of exactly one prisoner flipping a coin.
But the solution is a bit more involved, since scenarios which tend to lead to one coin flip also lead to zero coin flips.
Additionally, the solution needs to take into account the overall probability of each coin flipper getting heads.

To identify a forumula for the solution, I define two terms which relate the final probability of release.
First, let $P_{f}(n)$ be the probability of successfully flipping heads for each of $n$ prisoners (taking into account there needs to be at least one heads) is
\[
P_{f}(n)=
\begin{cases}
0 & n=0 \\
\frac{1}{2^{n}} & n\geq1
\end{cases}
.
\]

Second, let $P_{p}(n,N)$ be the probability of having $n$ coin flippers out of $N$ prisoners given a threshold (or equivalently, range) of $p$:
\[
P_{p}(n)=\binom{N}{n}p^{n}(1-p)^{N-n}.
\]

The $p$ value is where the random number generators come in.
Each prisoner should generate a number and if it is less than $p$ (or equivalently, within a specified range of length $p$) should flip the coin.

The total probability of release $P_{success}$ is
\[
P_{success}=\sum_{n=0}^{N}P_{p}(n)P_{f}(n)=\sum_{n=1}^{N}\binom{N}{n}p^{n}(1-p)^{N-n}
\]
which becomes a polynomial in $p$.
The solution is to maximize this with respect to $p$.
For $N=4$, the polynomial is
\[
-\frac{15}{16}p^{3}+\frac{7}{2}p^{2}-\frac{9}{2}p+2
\]
which has a maximum of approximately 0.2848 at
\fcolorbox{red}{white}{$\bm{p\approx0.3420}$}\,,
which is the solution to the riddle.

For the extra credit, I'm not sure how to turn this into an analytical solution.
I have solved for a few other small values, but I'm not sure of the limiting result.

\begin{center}
\begin{tabular}{c c c}
$N$ & $P_{success}$ & $p$ \\
    &               & \\
4   & 0.2848        & 0.3420 \\
6   & 0.2720        & 0.2292 \\
8   & 0.2661        & 0.1723 \\
10  & 0.2627        & 0.1380
\end{tabular}
\end{center}

It seems that the probability of success approches 0.25, but I can't prove that it does.
It does make sense, though, that the optimum $p$ gets smaller, and probably approaches 0 for large $N$.
This is because we want very few prisoners to have to flip a coin, and the expected number of coin flippers is always $Np$.



\end{document}