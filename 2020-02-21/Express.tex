\documentclass{article}


\usepackage{amsmath} % math stuff
\usepackage{amssymb} % math stuff
\usepackage{array} % equations and stuff
\usepackage{bm} % bold math
%\usepackage{caption} % suppressed table numbering; incompatible with revtex, and longtable, I think
\usepackage{comment} % comment environment
%\usepackage{enumitem} % customization of enumeration, itemize, and description
\usepackage[T1]{fontenc} % font encoding for special characters, must also use scalable font package
\usepackage[margin=0.8in]{geometry} % paper sizes and margins (but be careful not to mess up pre-defined pages)
\usepackage{graphicx} % for graphics
%\usepackage{helvet} % default font is the helvetica postscript font
\usepackage{lipsum} % lorem ipsum filler text
\usepackage{lmodern} % scalable font?
\usepackage{longtable} % multi-page tables
\usepackage{mathrsfs} % math script font
\usepackage{mhchem} % easier chemical formula
\usepackage{microtype} % allows disabling of ligatures
%\usepackage{newcent} % new century schoolbook font
\usepackage{nicefrac}
\usepackage{parskip} % removes paragraph indentation, and adjusts paragraph skip, as well as list items
%\usepackage{setspace} % adjust text spacing and indents
\usepackage{siunitx} % decimal alignment
\usepackage{subfigure} % divided figures
%\usepackage{tabu} % extra table options
\usepackage{textcomp} % symbols
\usepackage{threeparttablex} % better footnotes with longtable
\usepackage{titling} % title placement
\usepackage{ulem} % strikethrough text
%\usepackage{url} % superceded by hyperref
\usepackage{verbatim} % verbatim environment
\usepackage{xcolor} % colors and color boxes
\usepackage{xspace} % commands that don't eat up white space
\usepackage{hyperref} % links and page setup; should always come last

\hypersetup{
	bookmarks=true,
	colorlinks=true,
	citecolor=blue,
	linkcolor=blue,
	urlcolor=blue,
	pdfstartview={XYZ null null 1.0} % default open view is 100%
}

\DisableLigatures[f]{encoding = *, family = * } % disable ff, fi, fl ligatures, without f option, it also disables -- = endash
\renewcommand{\arraystretch}{1} % extra vertical space in tables

\begin{document}

\pagestyle{empty} % don't number pages

% custom title
\begin{center}
{\LARGE Express Riddler}

\vspace{0.15in}

{\Large 21 February 2020}
\end{center}


\section*{Riddle:}

On a warm, sunny day, Nick glanced at a thermometer, and noticed something quite interesting.
When he toggled between the Fahrenheit and Celsius scales, the digits of the temperature---when rounded to the nearest degree---had switched.
For example, this works for a temperature of 61 degrees Fahrenheit, which corresponds to a temperature of 16 degrees Celsius.

However, the temperature that day was not 61 degrees Fahrenheit.
What was the temperature?

\section*{Solution:}

The easiest way to solve this is to plug in several Fahrenheit temperatures into a spreadsheet, calculate the Celsius values, and visually compare the results.
I have done this in the file \texttt{Temperatures.xlsx}, where I have checked values between 50\textdegree\ and 100\textdegree\ Fahrenheit.
The values I checked were in 0.4\textdegree increments, so that the resolution was more than twice that of whole-number increments, to be wary of any intermediate rounding issues.
The value of 0.4 is otherwise just arbitrary.

Looking at the file, it is clear there are two solutions: 61\textdegree\ F/16\textdegree\ C and 82\textdegree\ F/28\textdegree\ C.
Since 61/16 was already ruled out, the solution is
\fcolorbox{red}{white}{82\textdegree\ F/28\textdegree\ C}\,.

Out of curiosity, I wanted to check what actual range of temperatures to which this solution corresponds.
The limits for 82\textdegree\ F are 81.5--82.5, which correspond to 27.5--28.056\textdegree\ C.
The limits for 28\textdegree\ C are 27.5--28.5, which correspond to 81.5--83.3\textdegree\ F.
Of course, the more stringent limits give the answer, which means the more precise temperature the thermometer recorded were in the range 81.5--82.499\textdegree\ F/27.5--28.056\textdegree\ C.


\end{document}