\documentclass{article}

\usepackage{amsmath} % math stuff
\usepackage{amssymb} % math stuff
\usepackage{array} % equations and stuff
\usepackage{bm} % bold math
%\usepackage{booktabs} % extra table rule options
%\usepackage{caption} % suppressed table numbering; incompatible with revtex, and longtable, I think
\usepackage{comment} % comment environment
%\usepackage{enumitem} % customization of enumeration, itemize, and description
\usepackage[T1]{fontenc} % font encoding for special characters, must also use scalable font package
\usepackage[margin=0.8in]{geometry} % paper sizes and margins (but be careful not to mess up pre-defined pages)
\usepackage{graphicx} % for graphics
%\usepackage{helvet} % default font is the helvetica postscript font
\usepackage[utf8]{inputenc} % special characters in tex input
\usepackage{layouts} % print units like widths
\usepackage{lipsum} % lorem ipsum filler text
\usepackage{lmodern} % scalable font?
\usepackage{longtable} % multi-page tables
\usepackage{makecell} % specify line-breaks in table cells
\usepackage{mathrsfs} % math script font
\usepackage{mhchem} % easier chemical formula
\usepackage{microtype} % allows disabling of ligatures
\usepackage{multicol} % multicolumns
%\usepackage{newcent} % new century schoolbook font
\usepackage{nicefrac}
\usepackage{numprint} % print and format (large) numbers
\usepackage{parskip} % removes paragraph indentation, and adjusts paragraph skip, as well as list items
\usepackage{pdfpages} % add pdf files as pages
%\usepackage{setspace} % adjust text spacing and indents
\usepackage{siunitx} % decimal alignment
\usepackage{subfigure} % divided figures
%\usepackage{tabu} % extra table options
\usepackage{textcomp} % symbols
\usepackage{threeparttablex} % better footnotes with longtable
\usepackage{titling} % title placement
\usepackage{ulem} % strikethrough text
%\usepackage{url} % superceded by hyperref
\usepackage{verbatim} % verbatim environment
\usepackage{xcolor} % colors and color boxes
\usepackage{xspace} % commands that don't eat up white space
\usepackage{hyperref} % links and page setup; should always come last

\hypersetup{
 bookmarks=true,
 colorlinks=true,
 citecolor=blue,
 linkcolor=blue,
 urlcolor=blue,
 pdfstartview={XYZ null null 1.0} % default open view is 100%
}

\DisableLigatures[f,t]{encoding = T1} % disable ff, fi, fl, tt ligatures; without options, it also disables -- = endash
\renewcommand{\arraystretch}{1.0} % extra vertical (and horizontal?) space in tables

% define centered, left- and right-aligned columns with specified widths
\newcommand{\PreserveBackslash}[1]{\let\temp=\\#1\let\\=\temp}
\newcolumntype{C}[1]{>{\PreserveBackslash\centering}p{#1}}
\newcolumntype{L}[1]{>{\PreserveBackslash\raggedright}p{#1}}
\newcolumntype{R}[1]{>{\PreserveBackslash\raggedleft}p{#1}}

\begin{document}

\pagestyle{empty} % don't number pages

% custom title
\begin{center}
{\LARGE Classic Riddler}

\vspace{0.15in}

{\Large 1 October 2021}
\end{center}


\section*{Riddle:}

You are responsible for setting the ranger schedule at Riddler River National Park. Four rangers are assigned to two locations: the mountain lookout in the north and the lakeside campground in the south.
Each assignment lasts one week (Monday through Friday), and every week two rangers should be in the north and two should be in the south.

Your task is to set an assignment schedule that lasts a certain number of weeks and then repeats indefinitely.

In the spirit of fairness, the rangers propose the following conditions for the schedule:

\begin{itemize}
\item Each ranger should spend as many weeks in the north as they do in the south.
\item Each ranger should spend the same number of weeks paired with each other ranger.
\item All rangers should move the same number of times over the course of the schedule.
This includes potentially moving back to their starting assignment after the last week of the schedule.
\item Exactly two rangers should switch locations each week.
\end{itemize}

What is the shortest possible repeating schedule that meets the rangers' conditions?

\section*{Solution:}

I use the letters A, B, C, and D to represent each of the four rangers, and designate the location of the rangers as a fraction.
For example, $\frac{\textrm{AB}}{\textrm{CD}}$ indicates A and B in the north location, and C and D in the south location.
There are six possible arrangements of rangers: $\frac{\textrm{AB}}{\textrm{CD}}$, $\frac{\textrm{AC}}{\textrm{BD}}$, $\frac{\textrm{AD}}{\textrm{BC}}$, $\frac{\textrm{CD}}{\textrm{AB}}$, $\frac{\textrm{BC}}{\textrm{AD}}$, and $\frac{\textrm{BD}}{\textrm{AC}}$.
I number these arrangements as 1, 2, 3, 4, 5, and 6, respectively.
For each arrangement, there are four possible ways for exactly two rangers to switch, leading to another arrangement.
For example, to move between arrangements 1 and 2, B and C must switch.
I drew out all possible arrangements and movements between them on the attached page.

Based on that drawing, it is possible to visualize each arrangement as a vertex of an octohedron, with each possible movement as an edge.
Vertices on opposite ends of the octohedron (1 and 4, 2 and 6, 3 and 5) do not have an edge between them, because moving between those arrangements would require all four rangers to switch at the same time.

With this image, I could imagine paths along the edges of the octohedron.
By the way the paths are defined, all paths meet the fourth condition.
In order to meet the repeating requirement, a path must end on the same vertex it started.
Visiting every vertex along the path further ensures it meets the first and second conditions.
It is possible to visit every vertex once using six steps.
For example, both the paths 1-2-3-4-5-6-1 and 1-3-6-4-2-5-1 (among numerous others) use six movements to visit all six vertices.
However, with the first path, A and B each switch twice, while C and D each switch four times.
Similarly, with the second path, A and B each switch 4 times, while C and D each switch twice.
In fact, all paths of six movements have this phenomenon.
If the above two paths are combined into a 12-movement sequence, though, the original conditions are still met, and each ranger switches six times, meeting the third condition.

Thus the solution is \fcolorbox{red}{white}{\textbf{12 weeks}}\,.
Here is my full solution:

\vspace{0.1in}
\begin{center}
\begin{tabular}{ccc}
Week & North & South \\
1  & AB & CD \\
2  & AC & BD \\
3  & AD & BC \\
4  & CD & AB \\
5  & BC & AD \\
6  & BD & AC \\
7  & AB & CD \\
8  & AD & BC \\
9  & BD & AC \\
10 & CD & AB \\
11 & AC & BD \\
12 & BC & AD
\end{tabular}
\end{center}
\vspace{0.1in}

\includepdf{rangers}


\end{document}