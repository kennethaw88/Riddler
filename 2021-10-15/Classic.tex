\documentclass{article}

\usepackage{amsmath} % math stuff
\usepackage{amssymb} % math stuff
\usepackage{array} % equations and stuff
\usepackage{bm} % bold math
%\usepackage{booktabs} % extra table rule options
%\usepackage{caption} % suppressed table numbering; incompatible with revtex, and longtable, I think
\usepackage{comment} % comment environment
%\usepackage{enumitem} % customization of enumeration, itemize, and description
\usepackage[T1]{fontenc} % font encoding for special characters, must also use scalable font package
\usepackage[margin=0.8in]{geometry} % paper sizes and margins (but be careful not to mess up pre-defined pages)
\usepackage{graphicx} % for graphics
%\usepackage{helvet} % default font is the helvetica postscript font
\usepackage[utf8]{inputenc} % special characters in tex input
\usepackage{layouts} % print units like widths
\usepackage{lipsum} % lorem ipsum filler text
\usepackage{lmodern} % scalable font?
\usepackage{longtable} % multi-page tables
\usepackage{makecell} % specify line-breaks in table cells
\usepackage{mathrsfs} % math script font
\usepackage{mhchem} % easier chemical formula
\usepackage{microtype} % allows disabling of ligatures
\usepackage{multicol} % multicolumns
%\usepackage{newcent} % new century schoolbook font
\usepackage{nicefrac}
\usepackage{numprint} % print and format (large) numbers
\usepackage{parskip} % removes paragraph indentation, and adjusts paragraph skip, as well as list items
\usepackage{pdfpages} % add pdf files as pages
%\usepackage{setspace} % adjust text spacing and indents
\usepackage{siunitx} % decimal alignment
\usepackage{subfigure} % divided figures
%\usepackage{tabu} % extra table options
\usepackage{textcomp} % symbols
\usepackage{threeparttablex} % better footnotes with longtable
\usepackage{titling} % title placement
\usepackage{ulem} % strikethrough text
%\usepackage{url} % superceded by hyperref
\usepackage{verbatim} % verbatim environment
\usepackage{xcolor} % colors and color boxes
\usepackage{xspace} % commands that don't eat up white space
\usepackage{hyperref} % links and page setup; should always come last

\hypersetup{
 bookmarks=true,
 colorlinks=true,
 citecolor=blue,
 linkcolor=blue,
 urlcolor=blue,
 pdfstartview={XYZ null null 1.0} % default open view is 100%
}

\DisableLigatures[f,t]{encoding = T1} % disable ff, fi, fl, tt ligatures; without options, it also disables -- = endash
\renewcommand{\arraystretch}{1.0} % extra vertical (and horizontal?) space in tables

% define centered, left- and right-aligned columns with specified widths
\newcommand{\PreserveBackslash}[1]{\let\temp=\\#1\let\\=\temp}
\newcolumntype{C}[1]{>{\PreserveBackslash\centering}p{#1}}
\newcolumntype{L}[1]{>{\PreserveBackslash\raggedright}p{#1}}
\newcolumntype{R}[1]{>{\PreserveBackslash\raggedleft}p{#1}}

\begin{document}

\pagestyle{empty} % don't number pages

% custom title
\begin{center}
{\LARGE Classic Riddler}

\vspace{0.15in}

{\Large 15 October 2021}
\end{center}


\section*{Riddle:}

Over in the National League Championship Series, the Washington Rationals and the St. Louis Ordinals (known as the ``Ords'' for short) are also evenly matched.
Again, both teams are equally likely to win each game of the best-of-seven series.

You enter a competition in which you must predict the winner of each of the seven games before the series begins.
If any or all of the fifth, sixth or seventh game are not played, you are not credited with predicting a winner.

You win the competition if you predict \textit{at least two} games correctly.
If you optimize your strategy for picking winners, what is the probability you will win the competition?

\textit{Extra credit:} You enter a second competition in which you must pick the winner of the first game and then of each next game, knowing who won in all the previous games.
Again, if you optimize your strategy, \textit{now} what is the probability you will predict at least two games correctly?


\section*{Solution:}

I designate a completed Championship Series as a list of Ws and Ss corresponding to the winner of each game in order.
For example, if the Washington Rationals win the first four games, this would be WWWW; similarly, if the St. Louis Ordinals win the first four games, this would be SSSS.

Because there are up to seven possible games to predict, there are $2^{7}=128$ possible predictions.
However, the actual strategy does not depend on the prediction of the first game, only on the pattern after the first game.
Any strategy that predicts Washington as the winner of the first game has a counterpart strategy that predicts St. Louis as the winner of the first game; these counterpart strategies have the same probability of winning the competition.
Additionally, the prediction for the seventh game doesn't matter; if the championship goes to seven games, each team will win with 50\%\ probability, so both predictions are equal.
Therefore, there are only $2^{5}=32$ strategies to consider.

As for the game-by-game outcomes of the series, there are only 70 possible outcomes.
Two outcomes (WWWW and SSSS) last only four games, and each have a probability of $(\nicefrac{1}{2})^4=\nicefrac{1}{16}$ of occurring.
There are eight outcomes that last five games, each with probability \nicefrac{1}{32}.
For a Washington win, these are

\vspace{0.1in}
\begin{center}
WWWSW \quad WWSWW \quad WSWWW \quad SWWWW
\end{center}
\vspace{0.1in}

and similar for St. Louis.
There are eight outcomes that last five games, each with probability \nicefrac{1}{64}.
For a Washington win, these are

\vspace{0.1in}
\begin{center}
WWWSSW \quad WWSWSW \quad WSWWSW \quad SWWWSW \quad WWSSWW \\
WSWSWW \quad SWWSWW \quad WSSWWW \quad SWSWWW \quad SSWWW
\end{center}
\vspace{0.1in}

and similar for St. Louis.
Finally, there are forty outcomes that last five games, each with probability \nicefrac{1}{128}.
For a Washington win, these are

\vspace{0.1in}
\begin{center}
WWWSSSW \quad WWSWSSW \quad WSWWSSW \quad SWWWSSW \quad WWSSWSW \\
WSWSWSW \quad SWWSWSW \quad WSSWWSW \quad SWSWWSW \quad SSWWWSW \\
WWSSSWW \quad WSWSSWW \quad SWWSSWW \quad WSSWSWW \quad SWSWSWW \\
SSWWSWW \quad WSSSWWW \quad SWSSWWW \quad SSWSWWW \quad SSSWWWW
\end{center}
\vspace{0.1in}

and similar for St. Louis.

The number of strategies and outcomes is small enough that it is possible to type them all out and make a tally of correct predictions for each strategy.
I have done so in \texttt{Predictions.xlsx}.
Using W as the prediction for the first and last games, this shows that the best strategy is WWWWSSW, with a probability of winning the competition of \fcolorbox{red}{white}{\textbf{0.90625}}\,.



\end{document}