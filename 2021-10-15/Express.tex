\documentclass{article}

\usepackage{amsmath} % math stuff
\usepackage{amssymb} % math stuff
\usepackage{array} % equations and stuff
\usepackage{bm} % bold math
%\usepackage{booktabs} % extra table rule options
%\usepackage{caption} % suppressed table numbering; incompatible with revtex, and longtable, I think
\usepackage{comment} % comment environment
%\usepackage{enumitem} % customization of enumeration, itemize, and description
\usepackage[T1]{fontenc} % font encoding for special characters, must also use scalable font package
\usepackage[margin=0.8in]{geometry} % paper sizes and margins (but be careful not to mess up pre-defined pages)
\usepackage{graphicx} % for graphics
%\usepackage{helvet} % default font is the helvetica postscript font
\usepackage[utf8]{inputenc} % special characters in tex input
\usepackage{layouts} % print units like widths
\usepackage{lipsum} % lorem ipsum filler text
\usepackage{lmodern} % scalable font?
\usepackage{longtable} % multi-page tables
\usepackage{makecell} % specify line-breaks in table cells
\usepackage{mathrsfs} % math script font
\usepackage{mhchem} % easier chemical formula
\usepackage{microtype} % allows disabling of ligatures
\usepackage{multicol} % multicolumns
%\usepackage{newcent} % new century schoolbook font
\usepackage{nicefrac}
\usepackage{numprint} % print and format (large) numbers
\usepackage{parskip} % removes paragraph indentation, and adjusts paragraph skip, as well as list items
\usepackage{pdfpages} % add pdf files as pages
%\usepackage{setspace} % adjust text spacing and indents
\usepackage{siunitx} % decimal alignment
\usepackage{subfigure} % divided figures
%\usepackage{tabu} % extra table options
\usepackage{textcomp} % symbols
\usepackage{threeparttablex} % better footnotes with longtable
\usepackage{titling} % title placement
\usepackage{ulem} % strikethrough text
%\usepackage{url} % superceded by hyperref
\usepackage{verbatim} % verbatim environment
\usepackage{xcolor} % colors and color boxes
\usepackage{xspace} % commands that don't eat up white space
\usepackage{hyperref} % links and page setup; should always come last

\hypersetup{
 bookmarks=true,
 colorlinks=true,
 citecolor=blue,
 linkcolor=blue,
 urlcolor=blue,
 pdfstartview={XYZ null null 1.0} % default open view is 100%
}

\DisableLigatures[f,t]{encoding = T1} % disable ff, fi, fl, tt ligatures; without options, it also disables -- = endash
\renewcommand{\arraystretch}{1.0} % extra vertical (and horizontal?) space in tables

% define centered, left- and right-aligned columns with specified widths
\newcommand{\PreserveBackslash}[1]{\let\temp=\\#1\let\\=\temp}
\newcolumntype{C}[1]{>{\PreserveBackslash\centering}p{#1}}
\newcolumntype{L}[1]{>{\PreserveBackslash\raggedright}p{#1}}
\newcolumntype{R}[1]{>{\PreserveBackslash\raggedleft}p{#1}}

\begin{document}

\pagestyle{empty} % don't number pages

% custom title
\begin{center}
{\LARGE Express Riddler}

\vspace{0.15in}

{\Large 15 October 2021}
\end{center}


\section*{Riddle:}

The American League Championship Series of Riddler League Baseball determines one of the teams that will compete in the Riddler World Series.
This year's teams---the Tampa Bay Lines and the Minnesota Twin Primes---are evenly matched.
In other words, both teams are equally likely to win each game of the best-of-seven series.

On average, how many games will the series last?
(Note that the series ends as soon as one team has won four games.)


\section*{Solution:}

I designate a completed Riddler World series as a list of Ts and Ms corresponding to the winner of each game in order.
For example, if the Tampa Bay Lines win the first four games, this would be TTTT; similarly, if the Minnesota Twin Primes win the first four games, this would be MMMM.
Because the probability of either team winning a single games is fixed at \nicefrac{1}{2}, the probability of winning four games is $(\nicefrac{1}{2})^4=\nicefrac{1}{16}$.
Because there are two ways for the series to end in four games, the total probability to end in four games is $2(\nicefrac{1}{16})=\nicefrac{1}{8}$.

There are four ways for either team to win in the fifth game.
For Tampa Bay to win, these are:

\vspace{0.1in}
\begin{center}
TTTMT \quad TTMTT \quad TMTTT \quad MTTTT
\end{center}
\vspace{0.1in}

Similarly, there are four ways for Minnesota to win in the fifth game.
So there are eight ways for the series to end in five games, and the total probability is $8(\nicefrac{1}{2})^5=\nicefrac{1}{4}$.

There are ten ways for Tampa Bay to win in the sixth game:

\vspace{0.1in}
\begin{center}
TTTMMT \quad TTMTMT \quad TMTTMT \quad MTTTMT \quad TTMMTT \\
TMTMTT \quad MTTMTT \quad TMMTTT \quad MTMTTT \quad MMTTTT
\end{center}
\vspace{0.1in}

So the total probability to end in six games is $20(\nicefrac{1}{2})^6=\nicefrac{5}{16}$.

Finally, there are 20 ways for Tampa Bay to win in the seventh game:

\vspace{0.1in}
\begin{center}
TTTMMMT \quad TTMTMMT \quad TMTTMMT \quad MTTTMMT \quad TTMMTMT \\
TMTMTMT \quad MTTMTMT \quad TMMTTMT \quad MTMTTMT \quad MMTTTMT \\
TTMMMTT \quad TMTMMTT \quad MTTMMTT \quad TMMTMTT \quad MTMTMTT \\
MMTTMTT \quad TMMMTTT \quad MTMMTTT \quad MMTMTTT \quad MMMTTTT
\end{center}
\vspace{0.1in}

So the total probability to end in six games is $40(\nicefrac{1}{2})^7=\nicefrac{5}{16}$.

The average number of games $N$ in the series is the weighted sum:

\[
N=4\left(\frac{1}{8}\right)+5\left(\frac{1}{4}\right)+6\left(\frac{5}{16}\right)
 +7\left(\frac{5}{16}\right)
\]

which gives a sum of \fcolorbox{red}{white}{\textbf{\nicefrac{93}{16}=5.8125}}\,.

\end{document}