\documentclass{article}


\usepackage{amsmath} % math stuff
\usepackage{amssymb} % math stuff
\usepackage{array} % equations and stuff
\usepackage{bm} % bold math
%\usepackage{caption} % suppressed table numbering; incompatible with revtex, and longtable, I think
\usepackage{comment} % comment environment
%\usepackage{enumitem} % customization of enumeration, itemize, and description
\usepackage[T1]{fontenc} % font encoding for special characters, must also use scalable font package
\usepackage[margin=0.8in]{geometry} % paper sizes and margins (but be careful not to mess up pre-defined pages)
\usepackage{graphicx} % for graphics
%\usepackage{helvet} % default font is the helvetica postscript font
\usepackage{lipsum} % lorem ipsum filler text
\usepackage{lmodern} % scalable font?
\usepackage{longtable} % multi-page tables
\usepackage{mathrsfs} % math script font
\usepackage{mhchem} % easier chemical formula
\usepackage{microtype} % allows disabling of ligatures
%\usepackage{newcent} % new century schoolbook font
\usepackage{nicefrac}
\usepackage{parskip} % removes paragraph indentation, and adjusts paragraph skip, as well as list items
%\usepackage{setspace} % adjust text spacing and indents
\usepackage{siunitx} % decimal alignment
\usepackage{subfigure} % divided figures
%\usepackage{tabu} % extra table options
\usepackage{textcomp} % symbols
\usepackage{threeparttablex} % better footnotes with longtable
\usepackage{titling} % title placement
\usepackage{ulem} % strikethrough text
%\usepackage{url} % superceded by hyperref
\usepackage{verbatim} % verbatim environment
\usepackage{xcolor} % colors and color boxes
\usepackage{xspace} % commands that don't eat up white space
\usepackage{hyperref} % links and page setup; should always come last

\hypersetup{
	bookmarks=true,
	colorlinks=true,
	citecolor=blue,
	linkcolor=blue,
	urlcolor=blue,
	pdfstartview={XYZ null null 1.0} % default open view is 100%
}

\DisableLigatures[f]{encoding = *, family = * } % disable ff, fi, fl ligatures, without f option, it also disables -- = endash
\renewcommand{\arraystretch}{1.1} % extra vertical space in tables

\begin{document}

\pagestyle{empty} % don't number pages

% custom title
\begin{center}
{\LARGE Express Riddler}

\vspace{0.15in}

{\Large 17 July 2020}
\end{center}


\section*{Riddle:}

Riddler Township is having its quadrennial presidential election.
Each of the town's 10 ``shires'' is allotted a certain number of electoral votes: two, plus one additional vote for every 10 citizens (rounded to the nearest 10).

The names and populations of the 10 shires are summarized in the table below.

\vspace{0.1in}
\begin{center}
\begin{tabular*}{0.6\textwidth}{ @{\extracolsep{\fill}} l c c}
\textbf{Shire} & \textbf{Population} & \textbf{Electoral Votes} \\
\hline
Oneshire   & 11  & 3 \\
Twoshire   & 21  & 4 \\
Threeshire & 31  & 5 \\
Fourshire  & 41  & 6 \\
Fiveshire  & 51  & 7 \\
Sixshire   & 61  & 8 \\
Sevenshire & 71  & 9 \\
Eightshire & 81  & 10 \\
Nineshire  & 91  & 11 \\
Tenshire   & 101 & 12 \\
\end{tabular*}
\end{center}
\vspace{0.1in}

As you may know, under this sort of electoral system, it is quite possible for a presidential candidate to lose the popular vote and still win the election.

If there are two candidates running for president of Riddler Township, and every single citizen votes for one or the other, then what is the \textit{lowest} percentage of the popular vote that a candidate can get while still winning the election?

\section*{Solution:}

There are 75 total electoral votes in this scenario.
The best way to get the minimum number of actual votes is to just barely get a majority of electoral votes, which is 38.
Whatever combination of shires gets to 38 electoral votes, those shires shires should also just barely get a majority of votes to swing their shires.
The remaining shires should have 100\% of their actual votes go the other way.
The majority combination of shires should also be preferentially lower-population shires.
For example, although Eightshire has as many electoral votes as Twoshire and Fourshire, it represents a larger actual voter population, so in this case, Twoshire and Fourshire would be preferred to Eightshire in forming an electoral majority.

By my non-exhaustive calculations, the best way to get to 38 electoral votes is with Oneshire (3), Threeshire (5), Fourshire (6), Fiveshire (7), Sixshire (8), and Sevenshire (9).
With each shire winning only a majority by one actual vote, that totals $6+16+21+26+31+36=136$ electoral votes.
The total population is 560, so my solution is \nicefrac{136}{560}~$\approx$~\fcolorbox{red}{white}{\bf 24.29\%}\,.


\end{document}