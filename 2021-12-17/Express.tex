\documentclass{article}

\usepackage{amsmath} % math stuff
\usepackage{amssymb} % math stuff
\usepackage{array} % equations and stuff
\usepackage{bm} % bold math
%\usepackage{booktabs} % extra table rule options
%\usepackage{caption} % suppressed table numbering; incompatible with revtex, and longtable, I think
\usepackage{comment} % comment environment
%\usepackage{enumitem} % customization of enumeration, itemize, and description
\usepackage[T1]{fontenc} % font encoding for special characters, must also use scalable font package
\usepackage[margin=0.8in]{geometry} % paper sizes and margins (but be careful not to mess up pre-defined pages)
\usepackage{graphicx} % for graphics
%\usepackage{helvet} % default font is the helvetica postscript font
\usepackage[utf8]{inputenc} % special characters in tex input
\usepackage{layouts} % print units like widths
\usepackage{lipsum} % lorem ipsum filler text
\usepackage{lmodern} % scalable font?
\usepackage{longtable} % multi-page tables
\usepackage{makecell} % specify line-breaks in table cells
\usepackage{mathrsfs} % math script font
\usepackage{mhchem} % easier chemical formula
\usepackage{microtype} % allows disabling of ligatures
\usepackage{multicol} % multicolumns
%\usepackage{newcent} % new century schoolbook font
\usepackage{nicefrac}
\usepackage{numprint} % print and format (large) numbers
\usepackage{parskip} % removes paragraph indentation, and adjusts paragraph skip, as well as list items
\usepackage{pdfpages} % add pdf files as pages
%\usepackage{setspace} % adjust text spacing and indents
\usepackage{siunitx} % decimal alignment
\usepackage{subfigure} % divided figures
%\usepackage{tabu} % extra table options
\usepackage{textcomp} % symbols
\usepackage{threeparttablex} % better footnotes with longtable
\usepackage{titling} % title placement
\usepackage{ulem} % strikethrough text
\usepackage{upgreek} % upright Greek letters
%\usepackage{url} % superceded by hyperref
\usepackage{verbatim} % verbatim environment
\usepackage{xcolor} % colors and color boxes
\usepackage{xspace} % commands that don't eat up white space
\usepackage{hyperref} % links and page setup; should always come last

\hypersetup{
 bookmarks=true,
 colorlinks=true,
 citecolor=blue,
 linkcolor=blue,
 urlcolor=blue,
 pdfstartview={XYZ null null 1.0} % default open view is 100%
}

\DisableLigatures[f,t]{encoding = T1} % disable ff, fi, fl, tt ligatures; without options, it also disables -- = endash
\renewcommand{\arraystretch}{1.0} % extra vertical (and horizontal?) space in tables

% define centered, left- and right-aligned columns with specified widths
\newcommand{\PreserveBackslash}[1]{\let\temp=\\#1\let\\=\temp}
\newcolumntype{C}[1]{>{\PreserveBackslash\centering}p{#1}}
\newcolumntype{L}[1]{>{\PreserveBackslash\raggedright}p{#1}}
\newcolumntype{R}[1]{>{\PreserveBackslash\raggedleft}p{#1}}

\begin{document}

\pagestyle{empty} % don't number pages

% custom title
\begin{center}
{\LARGE Express Riddler}

\vspace{0.15in}

{\Large 17 December 2021}
\end{center}


\section*{Riddle:}

You have been tasked with painting a modern building that is shaped like a regular tetrahedron. When the building is viewed from above, the architect wants it to appear as four congruent equilateral triangles---one central blue triangle surrounded by three white triangles.

That means that three faces of the tetrahedron contain a blue kite, as shown in the animation below:

\begin{center}
\includegraphics[width=1.7in]{frame_001}
\includegraphics[width=2.5in]{frame_049}
\includegraphics[width=2.5in]{frame_069}
\end{center}
\vspace{0.2in}

What is the measure of the smallest angle in this kite?


\section*{Solution:}

To solve this, I use the diagram below, viewing the tetrahedron from directly above (like the first frame earlier).
I label the nine points A--I that I use to solve the problem.

\vspace{0.1in}
\begin{center}
\includegraphics[width=2.5in]{diagram.png}
\end{center}
\vspace{0.1in}

Using a side length of 1, and placing A at the origin, I can determine the three-dimensional coordinates of each point:

\vspace{0.1in}
\begin{align*}
\mathrm{A} &: (0,0,0) \\
\mathrm{B} &: (\nicefrac{1}{2},\nicefrac{\sqrt{3}}{2},0) \\
\mathrm{C} &: (1,0,0) \\
\mathrm{D} &: (\nicefrac{1}{2},\nicefrac{\sqrt{3}}{6},\nicefrac{\sqrt{6}}{3}) \\
\mathrm{E} &: (\nicefrac{1}{2},0,0) \\
\mathrm{F} &: (\nicefrac{1}{4},\nicefrac{\sqrt{3}}{4},0) \\
\mathrm{G} &: (\nicefrac{3}{4},\nicefrac{\sqrt{3}}{4},0) \\
\mathrm{H} &: (\nicefrac{3}{8},\nicefrac{\sqrt{3}}{8},\nicefrac{\sqrt{6}}{4}) \\
\mathrm{I} &: (\nicefrac{5}{8},\nicefrac{\sqrt{3}}{8},\nicefrac{\sqrt{6}}{4})
\end{align*}
\vspace{0.1in}

With these, I can determine vectors of the edges of the blue area on the front face:

\vspace{0.1in}
\begin{align*}
\overrightarrow{\mathrm{EH}} &= (\nicefrac{3}{8},\nicefrac{\sqrt{3}}{8},\nicefrac{\sqrt{6}}{4})-(\nicefrac{1}{2},0,0) \\
 &= (-\nicefrac{1}{8},\nicefrac{\sqrt{3}}{8},\nicefrac{\sqrt{6}}{4}) \\
\overrightarrow{\mathrm{EI}} &= (\nicefrac{5}{8},\nicefrac{\sqrt{3}}{8},\nicefrac{\sqrt{6}}{4})-(\nicefrac{1}{2},0,0) \\
 &= (\nicefrac{1}{8},\nicefrac{\sqrt{3}}{8},\nicefrac{\sqrt{6}}{4})
\end{align*}
\vspace{0.1in}

The angle $\angle\mathrm{HEI}$ can then be determined by the formula:

\[
\cos(\angle\mathrm{HEI})=\frac{\overrightarrow{\mathrm{EH}}\cdot\overrightarrow{\mathrm{EI}}}{\left|\overrightarrow{\mathrm{EH}}\right|\left|\overrightarrow{\mathrm{EI}}\right|}
\]

This gives $\angle\mathrm{HEI}=\arccos(\nicefrac{13}{14})\approx\ang{21.8}$.
For completeness, it needs to be shown that this is indeed the smallest angle.
The top angle of the kite ($\angle\mathrm{HDI}$) is $\ang{60}$ because it is the corner of an equilateral triangle.
The other two side angles of the kite ($\angle\mathrm{DHE}$ and $\angle\mathrm{DIE}$) are both equal to $(\ang{360}-\ang{60}-\ang{21.8})/2\approx\ang{139.1}$.
So the smallest angle in the kite is (approximately) \fcolorbox{red}{white}{\textbf{21.8\textdegree}}\,.


\end{document}