\documentclass{article}

\usepackage{amsmath} % math stuff
\usepackage{amssymb} % math stuff
\usepackage{array} % equations and stuff
\usepackage{bm} % bold math
%\usepackage{booktabs} % extra table rule options
%\usepackage{caption} % suppressed table numbering; incompatible with revtex, and longtable, I think
\usepackage{comment} % comment environment
%\usepackage{enumitem} % customization of enumeration, itemize, and description
\usepackage[T1]{fontenc} % font encoding for special characters, must also use scalable font package
\usepackage[margin=0.8in]{geometry} % paper sizes and margins (but be careful not to mess up pre-defined pages)
\usepackage{graphicx} % for graphics
%\usepackage{helvet} % default font is the helvetica postscript font
\usepackage[utf8]{inputenc} % special characters in tex input
\usepackage{layouts} % print units like widths
\usepackage{lipsum} % lorem ipsum filler text
\usepackage{lmodern} % scalable font?
\usepackage{longtable} % multi-page tables
\usepackage{makecell} % specify line-breaks in table cells
\usepackage{mathrsfs} % math script font
\usepackage{mhchem} % easier chemical formula
\usepackage{microtype} % allows disabling of ligatures
\usepackage{multicol} % multicolumns
%\usepackage{newcent} % new century schoolbook font
\usepackage{nicefrac}
\usepackage{numprint} % print and format (large) numbers
\usepackage{parskip} % removes paragraph indentation, and adjusts paragraph skip, as well as list items
\usepackage{pdfpages} % add pdf files as pages
%\usepackage{setspace} % adjust text spacing and indents
\usepackage{siunitx} % decimal alignment
\usepackage{subfigure} % divided figures
%\usepackage{tabu} % extra table options
\usepackage{textcomp} % symbols
\usepackage{threeparttablex} % better footnotes with longtable
\usepackage{titling} % title placement
\usepackage{ulem} % strikethrough text
%\usepackage{url} % superceded by hyperref
\usepackage{verbatim} % verbatim environment
\usepackage{xcolor} % colors and color boxes
\usepackage{xspace} % commands that don't eat up white space
\usepackage{hyperref} % links and page setup; should always come last

\hypersetup{
 bookmarks=true,
 colorlinks=true,
 citecolor=blue,
 linkcolor=blue,
 urlcolor=blue,
 pdfstartview={XYZ null null 1.0} % default open view is 100%
}

\DisableLigatures[f,t]{encoding = T1} % disable ff, fi, fl, tt ligatures; without options, it also disables -- = endash
\renewcommand{\arraystretch}{1.0} % extra vertical (and horizontal?) space in tables

% define centered, left- and right-aligned columns with specified widths
\newcommand{\PreserveBackslash}[1]{\let\temp=\\#1\let\\=\temp}
\newcolumntype{C}[1]{>{\PreserveBackslash\centering}p{#1}}
\newcolumntype{L}[1]{>{\PreserveBackslash\raggedright}p{#1}}
\newcolumntype{R}[1]{>{\PreserveBackslash\raggedleft}p{#1}}

\begin{document}

\pagestyle{empty} % don't number pages

% custom title
\begin{center}
{\LARGE Classic Riddler}

\vspace{0.15in}

{\Large 10 September 2021}
\end{center}


\section*{Riddle:}


One morning, Phil was playing with his daughter, who loves to cut paper with her safety scissors.
She especially likes cutting paper into ``strips,'' which are rectangular pieces of paper whose shorter sides are at most 1 inch long.

Whenever Phil gives her a piece of standard printer paper (8.5 inches by 11 inches), she picks one of the four sides at random and then cuts a 1-inch wide strip parallel to that side.
Next, she discards the strip and repeats the process, picking another side at random and cutting the strip.
Eventually, she is left with nothing but strips.

On average, how many cuts will she make before she is left only with strips?

\textit{Extra credit}: Instead of 8.5 by 11-inch paper, what if the paper measures $m$ by $n$ inches?
(And for a special case of this, what if the paper is square?)


\section*{Solution:}


I note that the paper size of $8.5\times11$ produces the same result as $9\times11$, since it will take 8 cuts along a side with any length greater than 8 and at most 9 inches.
This allows the problem to be generalized to any two integers $m$ and $n$.

Because the two opposite sides of paper are the same, any cut on those two sides results in the same new leftover paper.
Therefore, there are only two options for cutting: the $m$ side and the $n$ side.
Each side has a 50\%\ probability of being cut.
I will label the expected number of remaining cuts from an $m\times n$ paper as $C(m,n)$.
If either $m$ or $n$ is 1, then $C=0$.
The minimum number of actual cuts starts from $m=n=2$, with $C(2,2)=1$, since a cut from either side leads to a single leftover strip with width 1.

After a few steps, it is clear that there is a recursive formula for $C$:

\[
C(m,n)=1+\left(C(m,n-1)+C(m-1,n)\right)/2,\quad m,n\ge2
\]

I don't know how to turn this into a general formula for $C(m,n)$.
For the specific case of $m=9$, $n=11$, I generated a table in Excel and got a result of \fcolorbox{red}{white}{$\bm{{\nicefrac{936,548}{65,536}\approx14.29}}$}\,.


\end{document}