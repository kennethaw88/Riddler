\documentclass{article}

\usepackage{amsmath} % math stuff
\usepackage{amssymb} % math stuff
\usepackage{array} % equations and stuff
\usepackage{bm} % bold math
%\usepackage{booktabs} % extra table rule options
%\usepackage{caption} % suppressed table numbering; incompatible with revtex, and longtable, I think
\usepackage{comment} % comment environment
%\usepackage{enumitem} % customization of enumeration, itemize, and description
\usepackage[T1]{fontenc} % font encoding for special characters, must also use scalable font package
\usepackage[margin=0.8in]{geometry} % paper sizes and margins (but be careful not to mess up pre-defined pages)
\usepackage{graphicx} % for graphics
%\usepackage{helvet} % default font is the helvetica postscript font
\usepackage[utf8]{inputenc} % special characters in tex input
\usepackage{layouts} % print units like widths
\usepackage{lipsum} % lorem ipsum filler text
\usepackage{lmodern} % scalable font?
\usepackage{longtable} % multi-page tables
\usepackage{makecell} % specify line-breaks in table cells
\usepackage{mathrsfs} % math script font
\usepackage{mhchem} % easier chemical formula
\usepackage{microtype} % allows disabling of ligatures
\usepackage{multicol} % multicolumns
%\usepackage{newcent} % new century schoolbook font
\usepackage{nicefrac}
\usepackage{numprint} % print and format (large) numbers
\usepackage{parskip} % removes paragraph indentation, and adjusts paragraph skip, as well as list items
\usepackage{pdfpages} % add pdf files as pages
%\usepackage{setspace} % adjust text spacing and indents
\usepackage{siunitx} % decimal alignment
\usepackage{subfigure} % divided figures
%\usepackage{tabu} % extra table options
\usepackage{textcomp} % symbols
\usepackage{threeparttablex} % better footnotes with longtable
\usepackage{titling} % title placement
\usepackage{ulem} % strikethrough text
%\usepackage{url} % superceded by hyperref
\usepackage{verbatim} % verbatim environment
\usepackage{xcolor} % colors and color boxes
\usepackage{xspace} % commands that don't eat up white space
\usepackage{hyperref} % links and page setup; should always come last

\hypersetup{
 bookmarks=true,
 colorlinks=true,
 citecolor=blue,
 linkcolor=blue,
 urlcolor=blue,
 pdfstartview={XYZ null null 1.0} % default open view is 100%
}

\DisableLigatures[f,t]{encoding = T1} % disable ff, fi, fl, tt ligatures; without options, it also disables -- = endash
\renewcommand{\arraystretch}{1.0} % extra vertical (and horizontal?) space in tables

% define centered, left- and right-aligned columns with specified widths
\newcommand{\PreserveBackslash}[1]{\let\temp=\\#1\let\\=\temp}
\newcolumntype{C}[1]{>{\PreserveBackslash\centering}p{#1}}
\newcolumntype{L}[1]{>{\PreserveBackslash\raggedright}p{#1}}
\newcolumntype{R}[1]{>{\PreserveBackslash\raggedleft}p{#1}}

\begin{document}

\pagestyle{empty} % don't number pages

% custom title
\begin{center}
{\LARGE Classic Riddler}

\vspace{0.15in}

{\Large 20 August 2021}
\end{center}


\section*{Riddle:}

When you roll a pair of fair dice, the most likely outcome is 7 (which occurs \nicefrac{1}{6} of the time) and the least likely outcomes are 2 and 12 (which each occur \nicefrac{1}{36} of the time).

Annoyed by the variance of these probabilities, I set out to create a pair of ``uniform dice.''
These dice still have sides that are uniquely numbered from 1 to 6, and they are identical to each other.
However, they are weighted so that their sum is more uniformly distributed between 2 and 12 than that of fair dice.

Unfortunately, it is impossible to create a pair of such dice so that the probabilities of all 11 sums from 2 to 12 are identical (i.e., they are all \nicefrac{1}{11}).
But I bet we can get pretty close.

The variance of the 11 probabilities is the average value of the squared difference between each probability and the average probability (which is, again, \nicefrac{1}{11}).
One way to make my dice as uniform as possible is to minimize this variance.

So how should I make my dice as uniform as possible?
In other words, which specific weighting of the dice minimizes the variance among the 11 probabilities?
That is, what should the probabilities be for rolling 1, 2, 3, 4, 5 or 6 with one of the dice?


\section*{Solution:}

The probabilities of rolling each side are constrained, since the six probabilities must sum to 1.
Further, because of the symmetry of the sums, I can determine that the probability of rolling a 1 or a 6 must be the same, and similarly with 2 and 5, and 3 and 4.
So there are only three actual probabilities, and because of the summing constraint, only two variable probabilities.

I will let $P$ be the probability of rolling a given value on one die.
Specifically, let $P(1)=P(6)=x$ and $P(2)=P(5)=y$.
This leaves $P(3)=P(4)=\nicefrac{1}{2}-x-y$.

I will further let $Q$ be the probability of rolling a given sum of two dice.
These probabilities are

\begin{align*}
Q(2)=Q(12) &= P(1)P(1) \\
&= x^{2} \\
Q(3)=Q(11) &= P(1)P(2) + P(2)P(1) \\
&= 2xy \\
Q(4)=Q(10) &= P(1)P(3) + P(2)P(2) + P(3)P(1) \\
&= -2x^{2}+y^{2}-2xy+x \\
Q(5)=Q(9) &= P(1)P(4) + P(2)P(3) + P(3)P(2) + P(4)P(1) \\
&= -2x^{2}-2y^{2}-4xy+x+y \\
Q(6)=Q(8) &= P(1)P(5) + P(2)P(4) + P(3)P(3) + P(4)P(2) + P(5)P(1) \\
&= x^{2}-y^{2}+2xy-x+\nicefrac{1}{4} \\
Q(7) &= P(1)P(6) + P(2)P(5) + P(3)P(4) + P(4)P(3) + P(5)P(2) + P(6)P(1) \\
&= 4x^{2} +4y^{2}+4xy-2x-2y+\nicefrac{1}{2}
\end{align*}

From here, I can define the average variance $V(x,y)$ as

\[
V(x,y)=\nicefrac{1}{11}\sum_{n=2}^{12}\left(Q(n)-\nicefrac{1}{11}\right)^{2}
\]

To find the minimum, I can take the partial derivatives with respect to $x$ and $y$.
This generates two cubic equations that must be solved simultaneously:

\begin{align*}
0 &= 144x^{3}+264x^{2}y+216xy^{2}+48y^{3}-108x^{2}-144xy-48y^{2}+30x+18y-3 \\
0 &= 88x^{3}+216x^{2}y+144xy^{2}+112y^{3}-72x^{2}-96xy-72y^{2}+18x+18y-2
\end{align*}

Plugging these into an online equation solver gives the real solutions $x\approx0.2439$ and $y\approx0.1375$.
This gives the following (approximate) probabilities:

\begin{align*}
P(1)=P(6) &\approx 0.2439 \\
P(2)=P(5) &\approx 0.1375 \\
P(3)=P(4) &\approx 0.1186 \\
{} \\
Q(2)=Q(12) &\approx 0.0594 \\
Q(3)=Q(11) &\approx 0.0671 \\
Q(4)=Q(10) &\approx 0.0768 \\
Q(5)=Q(9) &\approx 0.0905 \\
Q(6)=Q(8) &\approx 0.1138 \\
Q(7) &\approx 0.1849 \\
\end{align*}

The final average variance is $V\approx0.001218$.


\end{document}