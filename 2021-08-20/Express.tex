\documentclass{article}

\usepackage{amsmath} % math stuff
\usepackage{amssymb} % math stuff
\usepackage{array} % equations and stuff
\usepackage{bm} % bold math
%\usepackage{booktabs} % extra table rule options
%\usepackage{caption} % suppressed table numbering; incompatible with revtex, and longtable, I think
\usepackage{comment} % comment environment
%\usepackage{enumitem} % customization of enumeration, itemize, and description
\usepackage[T1]{fontenc} % font encoding for special characters, must also use scalable font package
\usepackage[margin=0.8in]{geometry} % paper sizes and margins (but be careful not to mess up pre-defined pages)
\usepackage{graphicx} % for graphics
%\usepackage{helvet} % default font is the helvetica postscript font
\usepackage[utf8]{inputenc} % special characters in tex input
\usepackage{layouts} % print units like widths
\usepackage{lipsum} % lorem ipsum filler text
\usepackage{lmodern} % scalable font?
\usepackage{longtable} % multi-page tables
\usepackage{makecell} % specify line-breaks in table cells
\usepackage{mathrsfs} % math script font
\usepackage{mhchem} % easier chemical formula
\usepackage{microtype} % allows disabling of ligatures
\usepackage{multicol} % multicolumns
%\usepackage{newcent} % new century schoolbook font
\usepackage{nicefrac}
\usepackage{numprint} % print and format (large) numbers
\usepackage{parskip} % removes paragraph indentation, and adjusts paragraph skip, as well as list items
\usepackage{pdfpages} % add pdf files as pages
%\usepackage{setspace} % adjust text spacing and indents
\usepackage{siunitx} % decimal alignment
\usepackage{subfigure} % divided figures
%\usepackage{tabu} % extra table options
\usepackage{textcomp} % symbols
\usepackage{threeparttablex} % better footnotes with longtable
\usepackage{titling} % title placement
\usepackage{ulem} % strikethrough text
%\usepackage{url} % superceded by hyperref
\usepackage{verbatim} % verbatim environment
\usepackage{xcolor} % colors and color boxes
\usepackage{xspace} % commands that don't eat up white space
\usepackage{hyperref} % links and page setup; should always come last

\hypersetup{
 bookmarks=true,
 colorlinks=true,
 citecolor=blue,
 linkcolor=blue,
 urlcolor=blue,
 pdfstartview={XYZ null null 1.0} % default open view is 100%
}

\DisableLigatures[f,t]{encoding = T1} % disable ff, fi, fl, tt ligatures; without options, it also disables -- = endash
\renewcommand{\arraystretch}{1.0} % extra vertical (and horizontal?) space in tables

% define centered, left- and right-aligned columns with specified widths
\newcommand{\PreserveBackslash}[1]{\let\temp=\\#1\let\\=\temp}
\newcolumntype{C}[1]{>{\PreserveBackslash\centering}p{#1}}
\newcolumntype{L}[1]{>{\PreserveBackslash\raggedright}p{#1}}
\newcolumntype{R}[1]{>{\PreserveBackslash\raggedleft}p{#1}}

\begin{document}

\pagestyle{empty} % don't number pages

% custom title
\begin{center}
{\LARGE Express Riddler}

\vspace{0.15in}

{\Large 20 August 2021}
\end{center}


\section*{Riddle:}

Help, there's a cricket on my floor!
I want to trap it with a cup so that I can safely move it outside.
But every time I get close, it hops exactly 1 foot in a random direction.

I take note of its starting position and come closer.
Boom---it hops in a random direction. I get close again.
Boom---it takes another hop in a random direction, independent of the direction of the first hop.

What is the \textit{most probable} distance between the cricket's current position after two random jumps and its starting position?
(Note: This puzzle is not asking for the \textit{expected} distance, but rather the \textit{most probable} distance.
In other words, if you consider the probability distribution over all possible distances, where is the peak of this distribution?)


\section*{Solution:}

We can assume that the first jump moves from the edge of a circle (with 1-ft radius) to the center of the circle.
Then the second jump will land somewhere on the edge of the circle.
We can call the angle between the two jumps $\theta$, and the distance (in ft) between the initial and final positions $L$.
This is shown below:

\vspace{0.1in}
\begin{center}
\includegraphics[width=1.5in]{circle2.png}
\end{center}
\vspace{0.1in}

The distance $L$ can be written as

\begin{align*}
L(\theta) &= \sqrt{(1-\cos\theta)^{2}+(\sin\theta)^{2}} \\
&= \sqrt{2-2\cos\theta}
\end{align*}

Finding the most probably distance requires maximizing this equation.
Because $L$ is positive, it is maximized at the same points as $L^{2}$, which is easier to differentiate:

\[
L^{2} = 2-2\cos\theta
\]
\[
\frac{d}{d\theta}L^{2} = 2\sin\theta
\]

Because of symmetry, this only needs to be evaluated in the range $[0,\pi]$.
The zeroes are at $\theta=0$ and $\pi$, which conveniently are also the edges of the range.
Evaluating these shows that the maximum is at $\theta=\pi$, with a most-likely distance of \fcolorbox{red}{white}{\textbf{2~ft}}\,.

\end{document}