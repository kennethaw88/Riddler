\documentclass{article}


\usepackage{amsmath} % math stuff
\usepackage{amssymb} % math stuff
\usepackage{array} % equations and stuff
\usepackage{bm} % bold math
%\usepackage{caption} % suppressed table numbering; incompatible with revtex, and longtable, I think
\usepackage{comment} % comment environment
%\usepackage{enumitem} % customization of enumeration, itemize, and description
\usepackage[T1]{fontenc} % font encoding for special characters, must also use scalable font package
\usepackage[margin=0.8in]{geometry} % paper sizes and margins (but be careful not to mess up pre-defined pages)
\usepackage{graphicx} % for graphics
%\usepackage{helvet} % default font is the helvetica postscript font
\usepackage{lipsum} % lorem ipsum filler text
\usepackage{lmodern} % scalable font?
\usepackage{longtable} % multi-page tables
\usepackage{mathrsfs} % math script font
\usepackage{mhchem} % easier chemical formula
\usepackage{microtype} % allows disabling of ligatures
%\usepackage{newcent} % new century schoolbook font
\usepackage{nicefrac}
\usepackage{parskip} % removes paragraph indentation, and adjusts paragraph skip, as well as list items
%\usepackage{setspace} % adjust text spacing and indents
\usepackage{siunitx} % decimal alignment
\usepackage{subfigure} % divided figures
%\usepackage{tabu} % extra table options
\usepackage{textcomp} % symbols
\usepackage{threeparttablex} % better footnotes with longtable
\usepackage{titling} % title placement
\usepackage{ulem} % strikethrough text
%\usepackage{url} % superceded by hyperref
\usepackage{verbatim} % verbatim environment
\usepackage{xcolor} % colors and color boxes
\usepackage{xspace} % commands that don't eat up white space
\usepackage{hyperref} % links and page setup; should always come last

\hypersetup{
	bookmarks=true,
	colorlinks=true,
	citecolor=blue,
	linkcolor=blue,
	urlcolor=blue,
	pdfstartview={XYZ null null 1.0} % default open view is 100%
}

\DisableLigatures[f]{encoding = *, family = * } % disable ff, fi, fl ligatures, without f option, it also disables -- = endash

\begin{document}

\pagestyle{empty} % don't number pages

% custom title
\begin{center}
{\LARGE Express Riddler}

\vspace{0.15in}

{\Large 24 January 2020}
\end{center}


\section*{Riddle:}

Derek Jeter and Larry Walker were just elected to the Baseball Hall of Fame!
That got Stephanie thinking.
Suppose there are 20 players on the ballot and 400 voters in a given year.
Each voter can select up to 10 players for induction without voting for any given player more than once.
To gain entry, a player must have been selected on at least 75 percent of the ballots.

Under these circumstances, what is the \textit{maximum} number of players that can be inducted into the Hall of Fame?


\section*{Solution:}

There are 400 total ballots, and I am assuming that each ballot must be cast, so that the 75\% threshold means that a player must receive 300 votes to be inducted.
The first way to determine the maximum is to divide the total votes by the 300-vote threshold.
Dividing 4000 votes by 300 gives 13.3\dots, which means that in principle, 13 players can be inducted.
But is there a way that it works out for 13 players?

The answer is yes.
For this argument, I will label the 13 lucky players A--M.
First, voters 1--300 each vote for player A.
Then voters 301--400 and 1--200 vote for player B.
Next, voters 201--400 and 1--100 vote for player C, and finally voters 101--400 vote for player D.
In this scenario, A, B, C, and D have each received exactly 300 votes, and each voter has used 3 votes.
This process repeats for players E--H and I--L.
Now 12 players have each received the necessary votes, and each voter has voted nine times.
Player M now just needs any 300 (or all 400) of the remaining votes.
At most, there are only 100 votes remaining for player N, so the estimate above holds true, that the fourteenth player cannot be inducted.
So the answer to the riddle is
\fcolorbox{red}{white}{13 players}\,.

\end{document}