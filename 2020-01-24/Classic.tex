\documentclass{article}


\usepackage{amsmath} % math stuff
\usepackage{amssymb} % math stuff
\usepackage{array} % equations and stuff
\usepackage{bm} % bold math
%\usepackage{caption} % suppressed table numbering; incompatible with revtex, and longtable, I think
\usepackage{comment} % comment environment
%\usepackage{enumitem} % customization of enumeration, itemize, and description
\usepackage[T1]{fontenc} % font encoding for special characters, must also use scalable font package
\usepackage[margin=0.8in]{geometry} % paper sizes and margins (but be careful not to mess up pre-defined pages)
\usepackage{graphicx} % for graphics
%\usepackage{helvet} % default font is the helvetica postscript font
\usepackage{lipsum} % lorem ipsum filler text
\usepackage{lmodern} % scalable font?
\usepackage{longtable} % multi-page tables
\usepackage{mathrsfs} % math script font
\usepackage{mhchem} % easier chemical formula
\usepackage{microtype} % allows disabling of ligatures
%\usepackage{newcent} % new century schoolbook font
\usepackage{nicefrac}
\usepackage{parskip} % removes paragraph indentation, and adjusts paragraph skip, as well as list items
%\usepackage{setspace} % adjust text spacing and indents
\usepackage{siunitx} % decimal alignment
\usepackage{subfigure} % divided figures
%\usepackage{tabu} % extra table options
\usepackage{textcomp} % symbols
\usepackage{threeparttablex} % better footnotes with longtable
\usepackage{titling} % title placement
\usepackage{ulem} % strikethrough text
%\usepackage{url} % superceded by hyperref
\usepackage{verbatim} % verbatim environment
\usepackage{xcolor} % colors and color boxes
\usepackage{xspace} % commands that don't eat up white space
\usepackage{hyperref} % links and page setup; should always come last

\hypersetup{
	bookmarks=true,
	colorlinks=true,
	citecolor=blue,
	linkcolor=blue,
	urlcolor=blue,
	pdfstartview={XYZ null null 1.0} % default open view is 100%
}

\DisableLigatures[f]{encoding = *, family = * } % disable ff, fi, fl ligatures, without f option, it also disables -- = endash

\begin{document}

\pagestyle{empty} % don't number pages

% custom title
\begin{center}
{\LARGE Classic Riddler}

\vspace{0.15in}

{\Large 24 January 2020}
\end{center}


\section*{Riddle:}

From Dean Ballard comes another coin-related challenge---a game of ``Pinching Pennies'':

The game starts with somewhere between 20 and 30 pennies, which I then divide into two piles any way I like.
Then we alternate taking turns, with you first, until someone wins the game.
For each turn, a player may take any number of pennies he or she likes from either pile, or instead take the same number of pennies from both piles.
Each player must also take at least one penny every turn.
The winner of the game is the one who takes the last penny.

If we both play optimally, what starting numbers of pennies (again, between 20 and 30) guarantee that you can win the game?


\section*{Solution:}

First off, this is a NIM game, which I generally dislike, mostly because I didn't \textit{get it} for a long time.
Now that I understand the general solution methods for NIM games, I only dislike them a bit less.

Luckily, it turns out the solution to this riddle has a simple, but non-obvious pattern.
(Truthfully, all NIM solutions are non-obvious).
The first step is to identify winning positions---sets of piles of given sizes which guaranteee that one player will win.
I will consider the winning positions from the perspective of the player who just finished his or her turn.
That means that the solution is for the starting player (``I'') not to have placed the pennies in a winning position.
I will denote a position as $(m,n)$ where $m$ and $n$ are the sizes of the two piles and, without loss of generality, let $m\leq <n$.
Therefore, the total number of pennies in a position is $m+n$.
The final winning position is (0,0), because of course, the player just won by removing all of the remaining pennies.

The next step is to identify the lowest possible number of pennies which creates a winning position.
For one remaining penny, the only position is (0,1), which is not a winning position, because the opponent \sout{can} \textit{must} remove the penny.
For two remaining pennies, there are two possible positions.
The (0,2) position is not winning, because the opponent can remove both pennies from the same pile.
Neither is (1,1) a winning position, because the opponent can remove one penny from both piles at the same time.

The first non-trivial winning position is (1,2).
From here, the opponent could remove the single penny from the first pile, leaving the first player the other pile altogether.
Alternately, the opponent could remove one penny from the second pile, allowing the first player to remove one penny from both piles at once.
Or, the opponent could remove two pennies from the second pile, forcing the first player to win (as above).
Finally, the opponent could remove one penny from both piles, also forcing the first player to win.
In general, any position of the form $(0,n)$ or $(m,m)$ leaves the other player with a trivial move to (0,0).
I will now call these losing positions.
It turns out every position is either winning or losing, so if a move does not create a guaranteed win, it in fact guarantees a loss.
All of the possible moves from (1,2) lead to either of these losing positions, so it is a winning position.

Now I can deduce three new losing positions: $(1,n)$, $(2,n)$, and $(m,m+1)$ (for $n>2$ and $m>1$).
From $(1,n)$ or $(2,n)$, a player can remove enough pennies from the $n$ pile to arrive at the winning position (1,2).
From any $(m,m+1)$ position, a player can remove $m-1$ pennies from both piles to get to (1,2).

Here I make two more observations: \textit{any} position with zero, one, or two pennies left in a pile (other than (1,2) of course) is a losing position, and \textit{any} position where the difference between the piles is zero or one (also excluding (1,2)) is a losing position.
Therefore, any other winning positions must have an $m$ that does not appear in smaller winning positions.
Further, the difference for winning positions ($n-m$) must also be larger than all smaller winning positions.
In order to force wins, the $m$ and $n-m$ values must be the minimum that meets these criteria.
So the next winning position will have $m=3$ with a difference of two, or (3,5).
I manually checked all possible moves from this position, and indeed, none directly lead to a winning position, but do create losing positions that then lead to a winning position.

\vspace{0.3in}
Following this logic, the first few non-trivial winning positions are:

\begin{center}
\begin{tabular}{c}
(1,2) \\
(3,5) \\
(4,7) \\
(6,10) \\
(8,13)
\end{tabular}
\qquad
\begin{tabular}{c}
(9,15) \\
(11,18) \\
(12,20) \\
(14,23) \\
(16,26)
\end{tabular}
\qquad
\begin{tabular}{c}
(17,28) \\
(19,31) \\
(21,34) \\
(22,36) \\
(23,38)
\end{tabular}
\end{center}

The winning positions are easy to generate, but it's hard to find an exact pattern or formula for generating them.
Concerning the original riddle, with total pennies in the range 20--30, the winning positions are (8,13), (9,15), and (11,18) for a total of 21, 24, and 29 pennies, respectively.
But the riddle asked from which numbers of pennies can the opponent (``you'') win after the first player sets up the piles.
This is just the reverse, so the solution is
\fcolorbox{red}{white}{20, 22, 23, 25, 26, 27, 28, and 30}\,.

\end{document}