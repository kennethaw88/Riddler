\documentclass{article}


\usepackage{amsmath} % math stuff
\usepackage{amssymb} % math stuff
\usepackage{array} % equations and stuff
\usepackage{bm} % bold math
%\usepackage{caption} % suppressed table numbering; incompatible with revtex, and longtable, I think
\usepackage{comment} % comment environment
%\usepackage{enumitem} % customization of enumeration, itemize, and description
\usepackage[T1]{fontenc} % font encoding for special characters, must also use scalable font package
\usepackage[margin=0.8in]{geometry} % paper sizes and margins (but be careful not to mess up pre-defined pages)
\usepackage{graphicx} % for graphics
%\usepackage{helvet} % default font is the helvetica postscript font
\usepackage{lipsum} % lorem ipsum filler text
\usepackage{lmodern} % scalable font?
\usepackage{longtable} % multi-page tables
\usepackage{mathrsfs} % math script font
\usepackage{mhchem} % easier chemical formula
\usepackage{microtype} % allows disabling of ligatures
%\usepackage{newcent} % new century schoolbook font
\usepackage{nicefrac}
\usepackage{parskip} % removes paragraph indentation, and adjusts paragraph skip, as well as list items
%\usepackage{setspace} % adjust text spacing and indents
\usepackage{siunitx} % decimal alignment
\usepackage{subfigure} % divided figures
\usepackage{sudoku} % sudoku grid
%\usepackage{tabu} % extra table options
\usepackage{textcomp} % symbols
\usepackage{threeparttablex} % better footnotes with longtable
\usepackage{titling} % title placement
\usepackage{ulem} % strikethrough text
%\usepackage{url} % superceded by hyperref
\usepackage{verbatim} % verbatim environment
\usepackage{xcolor} % colors and color boxes
\usepackage{xspace} % commands that don't eat up white space
\usepackage{hyperref} % links and page setup; should always come last

\hypersetup{
	bookmarks=true,
	colorlinks=true,
	citecolor=blue,
	linkcolor=blue,
	urlcolor=blue,
	pdfstartview={XYZ null null 1.0} % default open view is 100%
}

\DisableLigatures[f]{encoding = *, family = * } % disable ff, fi, fl ligatures, without f option, it also disables -- = endash
\renewcommand{\arraystretch}{1} % extra vertical space in tables

\setlength\sudokusize{8cm}
\renewcommand*\sudokuformat[1]{\LARGE\rmfamily#1}

\begin{document}

\pagestyle{empty} % don't number pages

% custom title
\begin{center}
{\LARGE Express Riddler}

\vspace{0.15in}

{\Large 15 May 2020}
\end{center}


\section*{Riddle:}

As you sit down to pass the time with a sudoku puzzle, you immediately notice that the grid is oddly sparse---only a handful of numbers are initially filled in.
But it gets worse.
While there aren't any numbers that occur more than once in the same row, column, or square (i.e., the grid doesn't \textit{ostensibly} break any of the sudoku rules), upon closer inspection, you can see that the puzzle is impossible.

What is the \textit{smallest} possible sum of the initial numbers in the grid?
(Note that multiple instances of the same number count separately.
So if your impossible grid happened to consist of eight 4s and two 5s, the sum would be 42.)

\section*{Solution:}

Here is my solution:

\vspace{0.1in}
\begin{sudoku}
|1| | | | | | | | |.
| | | |1| | | | | |.
| | | | | | |2| | |.
| | | | | | | |1| |.
| | | | | | | | | |.
| | | | | | | | | |.
| | | | | | | | |1|.
| | | | | | | | | |.
| | | | | | | | | |.
\end{sudoku}
\vspace{0.1in}

It is impossible because the single 2 is occupying the only spot where the upper-right 1 must be.
The sum of these numbers is a mere 6, which intuitively seems impossible to beat.
I don't think this can be improved because there need to be at least four numbers to specify a fifth position, and just having a single other number to block it seems to be the minimum required to create an impossible  scenario.
Therefore, using four 1s and a single 2 gives the minimum sum, so the solution to the riddle is
\fcolorbox{red}{white}{6}\,.



\end{document}