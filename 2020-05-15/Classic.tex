\documentclass{article}


\usepackage{amsmath} % math stuff
\usepackage{amssymb} % math stuff
\usepackage{array} % equations and stuff
\usepackage{bm} % bold math
%\usepackage{caption} % suppressed table numbering; incompatible with revtex, and longtable, I think
\usepackage{comment} % comment environment
%\usepackage{enumitem} % customization of enumeration, itemize, and description
\usepackage[T1]{fontenc} % font encoding for special characters, must also use scalable font package
\usepackage[margin=0.8in]{geometry} % paper sizes and margins (but be careful not to mess up pre-defined pages)
\usepackage{graphicx} % for graphics
%\usepackage{helvet} % default font is the helvetica postscript font
\usepackage{lipsum} % lorem ipsum filler text
\usepackage{lmodern} % scalable font?
\usepackage{longtable} % multi-page tables
\usepackage{mathrsfs} % math script font
\usepackage{mhchem} % easier chemical formula
\usepackage{microtype} % allows disabling of ligatures
%\usepackage{newcent} % new century schoolbook font
\usepackage{nicefrac}
\usepackage{parskip} % removes paragraph indentation, and adjusts paragraph skip, as well as list items
%\usepackage{setspace} % adjust text spacing and indents
\usepackage{siunitx} % decimal alignment
\usepackage{subfigure} % divided figures
%\usepackage{tabu} % extra table options
\usepackage{textcomp} % symbols
\usepackage{threeparttablex} % better footnotes with longtable
\usepackage{titling} % title placement
\usepackage{ulem} % strikethrough text
%\usepackage{url} % superceded by hyperref
\usepackage{verbatim} % verbatim environment
\usepackage{xcolor} % colors and color boxes
\usepackage{xspace} % commands that don't eat up white space
\usepackage{hyperref} % links and page setup; should always come last

\hypersetup{
	bookmarks=true,
	colorlinks=true,
	citecolor=blue,
	linkcolor=blue,
	urlcolor=blue,
	pdfstartview={XYZ null null 1.0} % default open view is 100%
}

\DisableLigatures[f]{encoding = *, family = * } % disable ff, fi, fl ligatures, without f option, it also disables -- = endash
\renewcommand{\arraystretch}{1} % extra vertical space in tables

\begin{document}

\pagestyle{empty} % don't number pages

% custom title
\begin{center}
{\LARGE Express Riddler}

\vspace{0.15in}

{\Large 15 May 2020}
\end{center}


\section*{Riddle:}

The fifth edition of Dungeons \& Dragons introduced a system of ``advantage and disadvantage.''
When you roll a die ``with advantage,'' you roll the die twice and keep the higher result.
Rolling ``with disadvantage'' is similar, except you keep the lower result instead.
The rules further specify that when a player rolls with both advantage and disadvantage, they cancel out, and the player rolls a single die.
Yawn!

There are two other, more mathematically interesting ways that advantage and disadvantage could be combined.
First, you could have ``advantage of disadvantage,'' meaning you roll twice with disadvantage and then keep the higher result.
Or, you could have ``disadvantage of advantage,'' meaning you roll twice with advantage and then keep the lower result.
With a fair 20-sided die, which situation produces the highest expected roll: advantage of disadvantage, disadvantage of advantage or rolling a single die?

\textit{Extra Credit}: Instead of maximizing your expected roll, suppose you need to roll $N$ or better with your 20-sided die.
For each value of $N$, is it better to use advantage of disadvantage, disadvantage of advantage or rolling a single die?

\section*{Solution:}

Obviously, with a single roll, each number has a probability $P(n)=\nicefrac{1}{20}$ chance of being rolled, and the average roll is 10.5.
How the advantage-of-disadvantage (AOD) and disadvantage-of-advantage (DOA) rolls compare to this depends on the probabilties that come out of the advantage and disadvantage rolling.
I have covered all of this in the spreadsheet \texttt{DnD\_rolling.xlsx}.

First, I have calculated the outcomes of the advantaged rolling.
For this, I created a table that displays the maximum from each rolled pair of numbers.
It is clear (and relatively straightforward to calculate by hand) that the probability of getting $n$ out of this scenario is

\begin{equation*}
P_{A}(n)=\frac{(2n-1)}{400}.
\end{equation*}

This probability is used in the DOA calculations.
Although not relevant to the riddle, the average roll in this case is now 13.825.

Next, I calculated the outcomes for the disadvantaged rolling using the minimum from each rolled pair.
Now, the probability becomes

\begin{equation*}
P_{D}(n)=\frac{(41-2n)}{400},
\end{equation*}

with an average roll of 7.175.
Similarly, this probability is used in the AOD calculations.
For completeness, I note that

\begin{equation*}
\sum_{i=1}^{20}P_{A}(n)=\sum_{i=1}^{20}P_{D}(n)=1.
\end{equation*}

I never actually figured out an equation for the probabilities using the AOD and DOA strategies ($P_{AOD}(n)$ and $P_{DOA}(n)$).
Instead I just created an extra set of tables, and made a few large Excel calculations to get to the solutions.
The second two tables calculate the probabilites of rolling a given pair of numbers based on two advantaged or two disadvantaged rolls.
I then calculated the average roll using those tables and the original maximum and minimum tables.
The end result is that the AOD situation has an average roll of 9.833\dots, while the DOA gives 11.166\dots\,.
Thus, the best strategy is
\fcolorbox{red}{white}{\bf disadvantage-of-advantage}\,.

I continued the calculations from the AOD and DOA tables to get at the extra credit question.
The summed probability of getting at least $n$ is shown below these tables for each of the single roll, AOD, and DOA.
Of course, for a roll of at least 1, all strategies give a probability of 1, since that includes all the numbers on the die.
Up through 13, the DOA strategy gives the highest probability, while for 14 and above, the single roll is best.
So the extra credit solution is
\fcolorbox{red}{white}{\bf use any for at least 1, use DOA for at least 2--13, and use a single roll for 14--20}\,.





\end{document}