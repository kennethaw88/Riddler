\documentclass{article}

\usepackage{amsmath}
\usepackage{amssymb}
\usepackage[T1]{fontenc}
\usepackage[margin=0.8in]{geometry}
\usepackage{lipsum}
\usepackage{lmodern}
%\usepackage{setspace}
\usepackage{titling}
%\usepackage{url}
\usepackage{xcolor} % colors and color boxes
\usepackage[colorlinks,urlcolor=blue]{hyperref}

\setlength{\droptitle}{-4em}

%\title{Riddler Express}
%\author{}
%\date{13 December 2019}

\begin{document}

\pagestyle{empty}

%\maketitle
\begin{center}
{\LARGE Riddler Express}

\vspace{0.2in}

{\Large 13 December 2019}
\end{center}


\section*{Riddle:}

The infamous 1984 World Chess Championship match between the reigning world champion Anatoly Karpov and 21-year-old Garry Kasparov was supposed to have been played until either player had won six games.
Instead, it went on for 48 games: Karpov won five, Kasparov won 3, and the other 40 games each ended in a draw.
Alas, the match was controversially terminated without a winner.

We can deduce from the games Karpov and Kasparov played that, independently of other games, Karpov’s chances of winning each game were 5/48, Kasparov’s chances were 3/48, and the chances of a draw were 40/48.
Had the match been allowed to continue indefinitely, what would have been Kasparov’s chances of eventually winning the match?



\section*{Solution:}

There are two interpretations to this riddle.

First I assumed that the match was started from scratch (i.e., from a score of 0-0-0).
Then I just need to calculate what are the chances to win at least 6 games outright.
It is clear that the probability of drawing has no impact on the calculations; a draw is essentially a do-over.
Thus, the problem is reduced to starting a match at 0-0, with probabilities of winning 5/8 and 3/8 for Karpov and Kasparaov, respectively.
The match is now best of 6, out of 11 matches.
The total probability is a binomial sum:
\begin{equation*}
P(\mathrm{Kasparov}) = \sum_{n=6}^{11}\binom{11}{n}\left(\frac{3}{8}\right)^{n}\left(\frac{5}{8}\right)^{11-n}
\end{equation*}
This gives a solution of 0.19434\dots, or \fcolorbox{red}{white}{\bf 19.4\%}.

The second interpretation is that the match continued from 3-5-48.
Again, draws have no impact, so this is essentially a match starting at 3-5, with the same probabilities for the remaining games.
The math is much simpler, since Kasparov must win three games in a row, with probability $(3/8)^{3}$.
This solution is then 0.052734375, or \fcolorbox{red}{white}{\bf 5.3\%}.

\end{document}