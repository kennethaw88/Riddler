\documentclass{article}

\usepackage{amsmath} % math stuff
\usepackage{amssymb} % math stuff
\usepackage{array} % equations and stuff
\usepackage{bm} % bold math
%\usepackage{caption} % suppressed table numbering; incompatible with revtex, and longtable, I think
\usepackage{comment} % comment environment
%\usepackage{enumitem} % customization of enumeration, itemize, and description
\usepackage[T1]{fontenc} % font encoding for special characters, must also use scalable font package
\usepackage[margin=0.8in]{geometry} % paper sizes and margins (but be careful not to mess up pre-defined pages)
\usepackage{graphicx} % for graphics
%\usepackage{helvet} % default font is the helvetica postscript font
\usepackage{lipsum} % lorem ipsum filler text
\usepackage{lmodern} % scalable font?
\usepackage{longtable} % multi-page tables
\usepackage{mathrsfs} % math script font
\usepackage{mhchem} % easier chemical formula
\usepackage{microtype} % allows disabling of ligatures
%\usepackage{newcent} % new century schoolbook font
\usepackage{nicefrac}
\usepackage{parskip} % removes paragraph indentation, and adjusts paragraph skip, as well as list items
%\usepackage{setspace} % adjust text spacing and indents
\usepackage{siunitx} % decimal alignment
\usepackage{subfigure} % divided figures
%\usepackage{tabu} % extra table options
\usepackage{textcomp} % symbols
\usepackage{threeparttablex} % better footnotes with longtable
\usepackage{titling} % title placement
\usepackage{ulem} % strikethrough text
%\usepackage{url} % superceded by hyperref
\usepackage{verbatim} % verbatim environment
\usepackage{xcolor} % colors and color boxes
\usepackage{xspace} % commands that don't eat up white space
\usepackage{hyperref} % links and page setup; should always come last

\hypersetup{
	bookmarks=true,
	colorlinks=true,
	citecolor=blue,
	linkcolor=blue,
	urlcolor=blue,
	pdfstartview={XYZ null null 1.0} % default open view is 100%
}

\DisableLigatures[f]{encoding = *, family = * } % disable ff, fi, fl ligatures, without f option, it also disables -- = endash
\renewcommand{\arraystretch}{1.1} % extra vertical space in tables

\begin{document}

\pagestyle{empty} % don't number pages

% custom title
\begin{center}
{\LARGE Classic Riddler}

\vspace{0.15in}

{\Large 11 September 2020}
\end{center}


\section*{Riddle:}

This month, the Tour de France is back, and so is the Tour de FiveThirtyEight!

For every mountain in the Tour de FiveThirtyEight, the first few riders to reach the summit are awarded points.
The rider with the most such points at the end of the Tour is named ``King of the Mountains'' and gets to wear a special polka dot jersey.

At the moment, you are racing against three other riders up one of the mountains.
The first rider over the top gets 5 points, the second rider gets 3, the third rider gets 2, and the fourth rider gets 1.

All four of you are of equal ability---that is, under normal circumstances, you all have an equal chance of reaching the summit first.
But there's a catch---two of your competitors are on the same team.
Teammates are able to work together, drafting and setting a tempo up the mountain.
Whichever teammate happens to be slower on the climb will get a boost from their faster teammate, and the two of them will both reach the summit at the faster teammate's time.

As a lone rider, the odds may be stacked against you.
In your quest for the polka dot jersey, how many points can you expect to win on this mountain, on average?

\section*{Solution:}

This problem is small enough that every possibility can be listed out.
First, because every rider individually has the same probability to be in any position, your own probability of being the fastest (and therefore coming in first) is simply \nicefrac{1}{4}, so you win 5 points with probability \nicefrac{1}{4}.
Similarly, you are the slowest, and win 1 point with probability \nicefrac{1}{4}.
These are independent of the team configurations, because the teamed riders are either both behind or ahead of you together.

If you are the second-fastest rider (which happens with probability \nicefrac{1}{4}), then you have to consider the team.
There are three possibilities for the team: the fastest and third-fastest riders, the fastest and slowest riders, or the third fastest and slowest riders.
In both of the possibilities with the fastest rider, you are bumped down a position, coming in third.
In the other possibility, you stay in second place.
So you get 2 points with probability \nicefrac{2}{12} and 3 points with probability \nicefrac{1}{12}.

If you are the third-fastest rider (probability \nicefrac{1}{4}), you must again consider the team.
In one of the three possibilities, the teammates are both ahead of you, and your position does not change.
In the remaining two possibilites, one teammate is ahead of you, and one teammate is behind you, in which case you are bumped down to last place.
So you win 2 points with probability \nicefrac{1}{12} and 1 point with probability \nicefrac{2}{12}.

Putting these together leads to an average score of
\fcolorbox{red}{white}{\bf \nicefrac{29}{12}}\,.


\end{document}