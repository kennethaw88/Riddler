\documentclass{article}


\usepackage{amsmath} % math stuff
\usepackage{amssymb} % math stuff
\usepackage{array} % equations and stuff
\usepackage{bm} % bold math
%\usepackage{caption} % suppressed table numbering; incompatible with revtex, and longtable, I think
\usepackage{comment} % comment environment
%\usepackage{enumitem} % customization of enumeration, itemize, and description
\usepackage[T1]{fontenc} % font encoding for special characters, must also use scalable font package
\usepackage[margin=0.8in]{geometry} % paper sizes and margins (but be careful not to mess up pre-defined pages)
\usepackage{graphicx} % for graphics
%\usepackage{helvet} % default font is the helvetica postscript font
\usepackage{lipsum} % lorem ipsum filler text
\usepackage{lmodern} % scalable font?
\usepackage{longtable} % multi-page tables
\usepackage{mathrsfs} % math script font
\usepackage{mhchem} % easier chemical formula
\usepackage{microtype} % allows disabling of ligatures
%\usepackage{newcent} % new century schoolbook font
\usepackage{nicefrac}
\usepackage{parskip} % removes paragraph indentation, and adjusts paragraph skip, as well as list items
%\usepackage{setspace} % adjust text spacing and indents
\usepackage{siunitx} % decimal alignment
\usepackage{subfigure} % divided figures
%\usepackage{tabu} % extra table options
\usepackage{textcomp} % symbols
\usepackage{threeparttablex} % better footnotes with longtable
\usepackage{titling} % title placement
\usepackage{ulem} % strikethrough text
%\usepackage{url} % superceded by hyperref
\usepackage{verbatim} % verbatim environment
\usepackage{xcolor} % colors and color boxes
\usepackage{xspace} % commands that don't eat up white space
\usepackage{hyperref} % links and page setup; should always come last

\hypersetup{
	bookmarks=true,
	colorlinks=true,
	citecolor=blue,
	linkcolor=blue,
	urlcolor=blue,
	pdfstartview={XYZ null null 1.0} % default open view is 100%
}

\DisableLigatures[f]{encoding = *, family = * } % disable ff, fi, fl ligatures, without f option, it also disables -- = endash
\renewcommand{\arraystretch}{1} % extra vertical space in tables

\begin{document}

\pagestyle{empty} % don't number pages

% custom title
\begin{center}
{\LARGE Classic Riddler}

\vspace{0.15in}

{\Large 17 July 2020}
\end{center}


\section*{Riddle:}

The tortoise and the hare are about to begin a 10-mile race along a ``stretch'' of road.
The tortoise is driving a car that travels 60 miles per hour, while the hare is driving a car that travels 75 miles per hour.
(For the purposes of this problem, assume that both cars accelerate from 0 miles per hour to their cruising speed instantaneously.)

The hare does a quick mental calculation and realizes if it waits until two minutes have passed, they’ll cross the finish line at the exact same moment.
And so, when the race begins, the tortoise drives off while the hare patiently waits.

But one minute into the race, after the tortoise has driven 1 mile, something extraordinary happens.
The road turns out to be magical and instantaneously stretches by 10 miles!
As a result of this stretching, the tortoise is now 2 miles ahead of the hare, who remains at the starting line.

At the end of every subsequent minute, the road stretches by 10 miles.
With this in mind, the hare does some more mental math.

How long after the race has begun should the hare wait so that both the tortoise and the hare will cross the finish line at the same exact moment?

\section*{Solution:}

At first glance, this problem seems impossible, since the road stretches at a rate of 10 miles per minute, ten times the speed of the tortoise.
However, because the tortoise is pulled along with the stretching, it does eventually reach the end.
The tortoise's distance becomes a harmonic sum $H_{n}$:

\[
H_{n}=\sum_{i=1}^{n} \frac{1}{i}
\]

This sum does not converge, and is larger than any real number for some large enough $n$.
There isn't an analytical way to find how large $n$ must be to exceed a given number; it must be solved numerically, especially since the tortoise will arrive between discrete stretches (because the harmonic number is constantly adding new primes to the denominator).

I decided to solve this in Excel, and the calculations can be found in \texttt{tortoise\_hare\_harmonic\_race.xlsx}.
First I had to calculate how long it would take the tortoise to finish.
I did this by calculating each one-minute section of driving and instant stretching step by step.
In the step-by-step section, the tortoise crosses the finish line sometime between minutes 12,366 and 12,367, at which time the road is 123,670 miles long.
After the stretching to 123,670 miles, the tortoise's position is 123,669.5319 miles (approximately), and it takes an additional 0.4681 minutes to get to the finish.

Next I used this finish time and distance for the hare, and made the same step-by-step calculation with the hare.
This time, the order of the road ``compression'' and movement of the hare are reversed.
Ultimately, the hare made it back to the start line between minutes 3 and 4, when the road is 4 miles long.
At minute 4, before the stretch (after the compression), the hare's position is 0.4167 miles.
Traveling that distance takes 0.3333 minutes, so the hare must leave after 2.6667 minutes.

I'm not sure if the solution should be exactly 2~\nicefrac{2}{3} minutes, or if it is just an artifact of the truncation of decimals and adding/subtracting small numbers.
So I'll leave the solution as an approximate decimal, and the hare should leave after
\fcolorbox{red}{white}{\bf 2.6667~minutes}\,.


\end{document}