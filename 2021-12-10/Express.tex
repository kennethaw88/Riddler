\documentclass{article}

\usepackage{amsmath} % math stuff
\usepackage{amssymb} % math stuff
\usepackage{array} % equations and stuff
\usepackage{bm} % bold math
%\usepackage{booktabs} % extra table rule options
%\usepackage{caption} % suppressed table numbering; incompatible with revtex, and longtable, I think
\usepackage{comment} % comment environment
%\usepackage{enumitem} % customization of enumeration, itemize, and description
\usepackage[T1]{fontenc} % font encoding for special characters, must also use scalable font package
\usepackage[margin=0.8in]{geometry} % paper sizes and margins (but be careful not to mess up pre-defined pages)
\usepackage{graphicx} % for graphics
%\usepackage{helvet} % default font is the helvetica postscript font
\usepackage[utf8]{inputenc} % special characters in tex input
\usepackage{layouts} % print units like widths
\usepackage{lipsum} % lorem ipsum filler text
\usepackage{lmodern} % scalable font?
\usepackage{longtable} % multi-page tables
\usepackage{makecell} % specify line-breaks in table cells
\usepackage{mathrsfs} % math script font
\usepackage{mhchem} % easier chemical formula
\usepackage{microtype} % allows disabling of ligatures
\usepackage{multicol} % multicolumns
%\usepackage{newcent} % new century schoolbook font
\usepackage{nicefrac}
\usepackage{numprint} % print and format (large) numbers
\usepackage{parskip} % removes paragraph indentation, and adjusts paragraph skip, as well as list items
\usepackage{pdfpages} % add pdf files as pages
%\usepackage{setspace} % adjust text spacing and indents
\usepackage{siunitx} % decimal alignment
\usepackage{subfigure} % divided figures
%\usepackage{tabu} % extra table options
\usepackage{textcomp} % symbols
\usepackage{threeparttablex} % better footnotes with longtable
\usepackage{titling} % title placement
\usepackage{ulem} % strikethrough text
\usepackage{upgreek} % upright Greek letters
%\usepackage{url} % superceded by hyperref
\usepackage{verbatim} % verbatim environment
\usepackage{xcolor} % colors and color boxes
\usepackage{xspace} % commands that don't eat up white space
\usepackage{hyperref} % links and page setup; should always come last

\hypersetup{
 bookmarks=true,
 colorlinks=true,
 citecolor=blue,
 linkcolor=blue,
 urlcolor=blue,
 pdfstartview={XYZ null null 1.0} % default open view is 100%
}

\DisableLigatures[f,t]{encoding = T1} % disable ff, fi, fl, tt ligatures; without options, it also disables -- = endash
\renewcommand{\arraystretch}{1.0} % extra vertical (and horizontal?) space in tables

% define centered, left- and right-aligned columns with specified widths
\newcommand{\PreserveBackslash}[1]{\let\temp=\\#1\let\\=\temp}
\newcolumntype{C}[1]{>{\PreserveBackslash\centering}p{#1}}
\newcolumntype{L}[1]{>{\PreserveBackslash\raggedright}p{#1}}
\newcolumntype{R}[1]{>{\PreserveBackslash\raggedleft}p{#1}}

\begin{document}

\pagestyle{empty} % don't number pages

% custom title
\begin{center}
{\LARGE Express Riddler}

\vspace{0.15in}

{\Large 10 December 2021}
\end{center}


\section*{Riddle:}

My condo complex has a single elevator that serves four stories: the garage (G), the first floor (1), the second floor (2) and the third floor (3).
Unfortunately, the elevator is malfunctioning and stopping at every single floor, no matter what.
The elevator always goes G, 1, 2, 3, 2, 1, G, 1, 2, etc.

I want to board the elevator on a random floor (with all four floors being equally likely).
As I round the corner to approach the elevator, I hear that its doors have closed, but I have no further information about which floor it’s on or whether the elevator is going up or down.
The doors might have just closed on my floor, for all I know.

On average, how many stops will the elevator make until it opens on my floor (including the stop on your floor)?
For example, if I am waiting on the second floor, and I heard the doors closing on the garage level, then the elevator would open on my floor in two stops.

\textit{Extra credit}: Instead of four floors, suppose my condo had $N$ floors.
On average, how many stops will the elevator make until it opens on my floor?


\section*{Solution:}

For the four-floor path (G-1-2-3-2-1-G...), there are six possible elevator starting points.
If you are on the garage floor, the starting points are 6, 5, 4, 3, 2, and 1 stop away.
Those have an average of \nicefrac{3}{2} stops.
Similary, if you are on the third floor, the average number of stops is also \nicefrac{3}{2}.
If you are on the first (or second) floor, the starting points are 1, 4, 3, 2, 1, and 2 stops away.
Those have an average of \nicefrac{13}{6} stops.
Averaging these numbers of stops over all four of your possible locations, the solution is \fcolorbox{red}{white}{\textbf{\nicefrac{17}{6}}}\,.


\end{document}