\documentclass{article}

\usepackage{amsmath} % math stuff
\usepackage{amssymb} % math stuff
\usepackage{array} % equations and stuff
\usepackage{bm} % bold math
%\usepackage{booktabs} % extra table rule options
%\usepackage{caption} % suppressed table numbering; incompatible with revtex, and longtable, I think
\usepackage{comment} % comment environment
%\usepackage{enumitem} % customization of enumeration, itemize, and description
\usepackage[T1]{fontenc} % font encoding for special characters, must also use scalable font package
\usepackage[margin=0.8in]{geometry} % paper sizes and margins (but be careful not to mess up pre-defined pages)
\usepackage{graphicx} % for graphics
%\usepackage{helvet} % default font is the helvetica postscript font
\usepackage[utf8]{inputenc} % special characters in tex input
\usepackage{layouts} % print units like widths
\usepackage{lipsum} % lorem ipsum filler text
\usepackage{lmodern} % scalable font?
\usepackage{longtable} % multi-page tables
\usepackage{makecell} % specify line-breaks in table cells
\usepackage{mathrsfs} % math script font
\usepackage{mhchem} % easier chemical formula
\usepackage{microtype} % allows disabling of ligatures
\usepackage{multicol} % multicolumns
%\usepackage{newcent} % new century schoolbook font
\usepackage{nicefrac}
\usepackage{numprint} % print and format (large) numbers
\usepackage{parskip} % removes paragraph indentation, and adjusts paragraph skip, as well as list items
\usepackage{pdfpages} % add pdf files as pages
%\usepackage{setspace} % adjust text spacing and indents
\usepackage{siunitx} % decimal alignment
\usepackage{subfigure} % divided figures
%\usepackage{tabu} % extra table options
\usepackage{textcomp} % symbols
\usepackage{threeparttablex} % better footnotes with longtable
\usepackage{titling} % title placement
\usepackage{ulem} % strikethrough text
\usepackage{upgreek} % upright Greek letters
%\usepackage{url} % superceded by hyperref
\usepackage{verbatim} % verbatim environment
\usepackage{xcolor} % colors and color boxes
\usepackage{xspace} % commands that don't eat up white space
\usepackage{hyperref} % links and page setup; should always come last

\hypersetup{
 bookmarks=true,
 colorlinks=true,
 citecolor=blue,
 linkcolor=blue,
 urlcolor=blue,
 pdfstartview={XYZ null null 1.0} % default open view is 100%
}

\DisableLigatures[f,t]{encoding = T1} % disable ff, fi, fl, tt ligatures; without options, it also disables -- = endash
\renewcommand{\arraystretch}{1.0} % extra vertical (and horizontal?) space in tables

% define centered, left- and right-aligned columns with specified widths
\newcommand{\PreserveBackslash}[1]{\let\temp=\\#1\let\\=\temp}
\newcolumntype{C}[1]{>{\PreserveBackslash\centering}p{#1}}
\newcolumntype{L}[1]{>{\PreserveBackslash\raggedright}p{#1}}
\newcolumntype{R}[1]{>{\PreserveBackslash\raggedleft}p{#1}}

\begin{document}

\pagestyle{empty} % don't number pages

% custom title
\begin{center}
{\LARGE Classic Riddler}

\vspace{0.15in}

{\Large 10 December 2021}
\end{center}


\section*{Riddle:}

You are the coach at Riddler Fencing Academy, where your three students are squaring off against a neighboring squad.
Each of your students has a different probability of winning any given point in a match.
The strongest fencer has a 75 percent chance of winning each point.
The weakest has only a 25 percent chance of winning each point.
The remaining fencer has a 50 percent probability of winning each point.

The match will be a relay.
First, one of your students will face off against an opponent.
As soon as one of them reaches a score of 15, they are both swapped out.
Then, a different student of yours faces a different opponent, continuing from wherever the score left off.
When one team reaches 30 (not necessarily from the same team that first reached 15), both fencers are swapped out.
The remaining two fencers continue the relay until one team reaches 45 points.

As the coach, you can choose the order in which your three students occupy the three positions in the relay: going first, second or third.
How will you order them?
And then what will be your team's chances of winning the relay?


\section*{Solution:}

I wrote code to simulate this match at \texttt{fencing.C}.
It repeatedly generates random numbers to determine which team wins a given point.
For the 50\%\ player, it performs modulo 2 on a random number, while for the other players, it performs modulo 2 on two random numbers to get 25\%\ and 75\%\ using an \texttt{and} or \texttt{or} condition, respectively.

By manually switching around the code blocks, I could individually calculate the probability of winning the match with different orders of students.
The six results are as follows:

\vspace{0.1in}
\begin{center}
\begin{tabular}{cc}
Order & Probability \\
\hline
25-50-75 & 0.932 \\
25-75-50 & 0.826 \\
50-25-75 & 0.925 \\
50-75-25 & 0.075 \\
75-25-50 & 0.174 \\
75-50-25 & 0.068
\end{tabular}
\end{center}
\vspace{0.1in}

So the highest probability of winning comes from the order \fcolorbox{red}{white}{\textbf{25-50-75}}\,, which gave an overall probability of approximately \fcolorbox{red}{white}{\textbf{93.2\%}}\,.


\end{document}