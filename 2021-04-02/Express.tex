\documentclass{article}

\usepackage{amsmath} % math stuff
\usepackage{amssymb} % math stuff
\usepackage{array} % equations and stuff
\usepackage{bm} % bold math
%\usepackage{booktabs} % extra table rule options
%\usepackage{caption} % suppressed table numbering; incompatible with revtex, and longtable, I think
\usepackage{comment} % comment environment
%\usepackage{enumitem} % customization of enumeration, itemize, and description
\usepackage[T1]{fontenc} % font encoding for special characters, must also use scalable font package
\usepackage[margin=0.8in]{geometry} % paper sizes and margins (but be careful not to mess up pre-defined pages)
\usepackage{graphicx} % for graphics
%\usepackage{helvet} % default font is the helvetica postscript font
\usepackage{layouts} % print units like widths
\usepackage{lipsum} % lorem ipsum filler text
\usepackage{lmodern} % scalable font?
\usepackage{longtable} % multi-page tables
\usepackage{makecell} % specify line-breaks in table cells
\usepackage{mathrsfs} % math script font
\usepackage{mhchem} % easier chemical formula
\usepackage{microtype} % allows disabling of ligatures
%\usepackage{newcent} % new century schoolbook font
\usepackage{nicefrac}
\usepackage{numprint} % print and format (large) numbers
\usepackage{parskip} % removes paragraph indentation, and adjusts paragraph skip, as well as list items
\usepackage{pdfpages} % add pdf files as pages
%\usepackage{setspace} % adjust text spacing and indents
\usepackage{siunitx} % decimal alignment
\usepackage{subfigure} % divided figures
%\usepackage{tabu} % extra table options
\usepackage{textcomp} % symbols
\usepackage{threeparttablex} % better footnotes with longtable
\usepackage{titling} % title placement
\usepackage{ulem} % strikethrough text
%\usepackage{url} % superceded by hyperref
\usepackage{verbatim} % verbatim environment
\usepackage{xcolor} % colors and color boxes
\usepackage{xspace} % commands that don't eat up white space
\usepackage{hyperref} % links and page setup; should always come last

\hypersetup{
 bookmarks=true,
 colorlinks=true,
 citecolor=blue,
 linkcolor=blue,
 urlcolor=blue,
 pdfstartview={XYZ null null 1.0} % default open view is 100%
}

\DisableLigatures[f,t]{encoding = T1} % disable ff, fi, fl, tt ligatures; without options, it also disables -- = endash
\renewcommand{\arraystretch}{1.0} % extra vertical (and horizontal?) space in tables

% define centered, left- and right-aligned columns with specified widths
\newcommand{\PreserveBackslash}[1]{\let\temp=\\#1\let\\=\temp}
\newcolumntype{C}[1]{>{\PreserveBackslash\centering}p{#1}}
\newcolumntype{L}[1]{>{\PreserveBackslash\raggedright}p{#1}}
\newcolumntype{R}[1]{>{\PreserveBackslash\raggedleft}p{#1}}

\begin{document}

\pagestyle{empty} % don't number pages

% custom title
\begin{center}
{\LARGE Express Riddler}

\vspace{0.15in}

{\Large 2 April 2021}
\end{center}


\section*{Riddle:}

You and Wenjun are playing a game in which you alternate taking turns, removing pennies from a pile.
On your turn, you can remove either one or two pennies from the pile.
On Wenjun's turn, he can remove either two or three pennies.
Whoever takes the last penny \textit{loses}.
(If there is only one penny left and it's Wenjun's turn, then he skips his turn, which means you will lose).
Suppose both you and Wenjun play optimally.

1) If you go first, then what initial numbers of pennies mean \textit{you} will win the game?

2) If Wenjun goes first, then what initial numbers of pennies mean \textit{he} will win the game?



\section*{Solution:}

Like other NIM games, the way to solve is starting from the end of the game when somebody loses.
In order for Wenjun to lose, he has to start a turn with exactly 2 pennies.
Since he can't remove 3 coins, he must remove both of the remaining pennies.
If there is 1 penny left, he gets to pass, and you lose by removing the last penny.
(If there are 0 pennies left, that means you have already lost.)
With 3 pennies left, he can choose to remove only 2, leaving you to remove the last penny.
Similarly, with 4 pennies left, he can choose to remove 3, also leaving you to remove the last penny.

So in order to make sure that Wenjun loses, you must finish a turn with 2 pennies left.
This can be done if you start your turn with 3 (and remove 1) or start your turn with 4 (and remove 2).
Thus, if Wenjun finishes a turn with 3 or 4 pennies left, you can win.
If Wenjun starts a turn with 6 pennies, then he can only finish his turn with 3 or 4 pennies, and he will lose in the end.

This is where the pattern starts to repeat.
Wenjun will lose (and you will win) if his turn starts with 2, 6, 10, 14\dots pennies, or more generally, any number of the form $4n+2$.
Following the same pattern, Wenjun will also lose if your turn starts with 3, 4, 7, 8, 11, 12\dots, which are numbers of the form $4n$ and $4n+3$.

So to answer the specific questions in the riddle:
When you go first, you will win if the game starts with
\fcolorbox{red}{white}{\bf $\bm{4n}$ or $\bm{4n+3}$ pennies}\,.
When Wenjun goes first, Wenjun will win if the game starts with\\
\fcolorbox{red}{white}{\bf $\bm{4n}$, $\bm{4n+1}$, or $\bm{4n+3}$ pennies}\,.



\end{document}