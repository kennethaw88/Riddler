\documentclass{article}

\usepackage{amsmath} % math stuff
\usepackage{amssymb} % math stuff
\usepackage{array} % equations and stuff
\usepackage{bm} % bold math
%\usepackage{caption} % suppressed table numbering; incompatible with revtex, and longtable, I think
\usepackage{comment} % comment environment
%\usepackage{enumitem} % customization of enumeration, itemize, and description
\usepackage[T1]{fontenc} % font encoding for special characters, must also use scalable font package
\usepackage[margin=0.8in]{geometry} % paper sizes and margins (but be careful not to mess up pre-defined pages)
\usepackage{graphicx} % for graphics
%\usepackage{helvet} % default font is the helvetica postscript font
\usepackage{layouts} % print units like widths
\usepackage{lipsum} % lorem ipsum filler text
\usepackage{lmodern} % scalable font?
\usepackage{longtable} % multi-page tables
\usepackage{mathrsfs} % math script font
\usepackage{mhchem} % easier chemical formula
\usepackage{microtype} % allows disabling of ligatures
%\usepackage{newcent} % new century schoolbook font
\usepackage{nicefrac}
\usepackage{parskip} % removes paragraph indentation, and adjusts paragraph skip, as well as list items
\usepackage{pdfpages} % add pdf files as pages
%\usepackage{setspace} % adjust text spacing and indents
\usepackage{siunitx} % decimal alignment
\usepackage{subfigure} % divided figures
%\usepackage{tabu} % extra table options
\usepackage{textcomp} % symbols
\usepackage{threeparttablex} % better footnotes with longtable
\usepackage{titling} % title placement
\usepackage{ulem} % strikethrough text
%\usepackage{url} % superceded by hyperref
\usepackage{verbatim} % verbatim environment
\usepackage{xcolor} % colors and color boxes
\usepackage{xspace} % commands that don't eat up white space
\usepackage{hyperref} % links and page setup; should always come last

\hypersetup{
	bookmarks=true,
	colorlinks=true,
	citecolor=blue,
	linkcolor=blue,
	urlcolor=blue,
	pdfstartview={XYZ null null 1.0} % default open view is 100%
}

\DisableLigatures[f,t]{encoding = T1} % disable ff, fi, fl, tt ligatures, without f option, it also disables -- = endash
\renewcommand{\arraystretch}{1.1} % extra vertical space in tables

\begin{document}

\pagestyle{empty} % don't number pages

% custom title
\begin{center}
{\LARGE Express Riddler}

\vspace{0.15in}

{\Large 30 October 2020}
\end{center}


\section*{Riddle:}

While waiting in line to vote early last week, I overheard a discussion between election officials.
Apparently, there may have been a political sign that was within 100~feet of the polling place, against New York State law.

This got me thinking.
Suppose a polling site is a square building whose sides are 100 feet in length.
An election official takes a string that is also 100 feet long and ties one end to a door located in the middle of one of the building’s sides.
She then holds the other end of the string in her hand.

What's the area of the region outside of the building she can reach while holding the string?

\section*{Solution:}

The path traced out by the end of the string is a circular arc.
When the string hits the edge of the building, it continues in a smaller arc along the sides of the building.
The shape is shown below:

\vspace{0.1in}
\begin{center}
\includegraphics[width=3in]{polling_radius.png}
\end{center}
\vspace{0.1in}

The region traced out by the string forms a semicircle and two quarter circles.
The radius of the large semicircle is of course 100~ft, and the radius of the quarter circles is 50~ft.
So the area $A$ is

\[
A=\frac{1}{2}\pi 100^2+2\frac{1}{4}\pi 50^2
\]

The solution is therefore (approximately)
\fcolorbox{red}{white}{\bf 19,635~sq~ft}\,.




\end{document}