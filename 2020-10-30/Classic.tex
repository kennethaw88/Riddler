\documentclass{article}

\usepackage{amsmath} % math stuff
\usepackage{amssymb} % math stuff
\usepackage{array} % equations and stuff
\usepackage{bm} % bold math
%\usepackage{caption} % suppressed table numbering; incompatible with revtex, and longtable, I think
\usepackage{comment} % comment environment
%\usepackage{enumitem} % customization of enumeration, itemize, and description
\usepackage[T1]{fontenc} % font encoding for special characters, must also use scalable font package
\usepackage[margin=0.8in]{geometry} % paper sizes and margins (but be careful not to mess up pre-defined pages)
\usepackage{graphicx} % for graphics
%\usepackage{helvet} % default font is the helvetica postscript font
\usepackage{layouts} % print units like widths
\usepackage{lipsum} % lorem ipsum filler text
\usepackage{lmodern} % scalable font?
\usepackage{longtable} % multi-page tables
\usepackage{mathrsfs} % math script font
\usepackage{mhchem} % easier chemical formula
\usepackage{microtype} % allows disabling of ligatures
%\usepackage{newcent} % new century schoolbook font
\usepackage{nicefrac}
\usepackage{parskip} % removes paragraph indentation, and adjusts paragraph skip, as well as list items
\usepackage{pdfpages} % add pdf files as pages
%\usepackage{setspace} % adjust text spacing and indents
\usepackage{siunitx} % decimal alignment
\usepackage{subfigure} % divided figures
%\usepackage{tabu} % extra table options
\usepackage{textcomp} % symbols
\usepackage{threeparttablex} % better footnotes with longtable
\usepackage{titling} % title placement
\usepackage{ulem} % strikethrough text
%\usepackage{url} % superceded by hyperref
\usepackage{verbatim} % verbatim environment
\usepackage{xcolor} % colors and color boxes
\usepackage{xspace} % commands that don't eat up white space
\usepackage{hyperref} % links and page setup; should always come last

\hypersetup{
 bookmarks=true,
 colorlinks=true,
 citecolor=blue,
 linkcolor=blue,
 urlcolor=blue,
 pdfstartview={XYZ null null 1.0} % default open view is 100%
}

\DisableLigatures[f,t]{encoding = T1} % disable ff, fi, fl, tt ligatures, without f option, it also disables -- = endash
\renewcommand{\arraystretch}{1.1} % extra vertical space in tables

\begin{document}

\pagestyle{empty} % don't number pages

% custom title
\begin{center}
{\LARGE Classic Riddler}

\vspace{0.15in}

{\Large 30 October 2020}
\end{center}


\section*{Riddle:}

Instead of playing hot potato, you and 60 of your closest friends decide to play a socially distanced game of hot pumpkin.

Before the game starts, you all sit in a circle and agree on a positive integer $N$.
Once the number has been chosen, you (the leader of the group) start the game by counting ``1'' and passing the pumpkin to the person sitting directly to your left.
She then declares ``2'' and passes the pumpkin one space to \textit{her} left.
This continues with each player saying the next number in the sequence, wrapping around the circle as many times as necessary, until the group has collectively counted up to $N$.
At that point, the player who counted ``$N$'' is eliminated, and the player directly to his or her left starts the next round, again proceeding to the same value of $N$.
The game continues until just one player remains, who is declared the victor.

In the game's first round, the player 18 spaces to your left is the first to be eliminated.
Ricky, the next player in the sequence, begins the next round.
The second round sees the elimination of the player 31 spaces to Ricky's left.
Zach begins the third round, only to find himself eliminated in a cruel twist of fate.
(Woe is Zach.)

What was the smallest value of $N$ the group could have used for this game?

\textit{Extra credit}: Suppose the players were numbered from 1 to 61, with you as Player No. 1, the player to your left as Player No. 2 and so on.
Which player won the game?

\textit{Extra extra credit}: What's the smallest $N$ that would have made you the winner?

\section*{Solution:}

I went about this problem in a few different ways.
To answer the first question, I set up a script that searches for and checks possible solutions.
Based on the description, the number $N$ must satisfy the following equations:

\begin{align*}
N&=19 \mod 61 \\
N&=32 \mod 60 \\
N&=1 \mod 59
\end{align*}

The script I set up can be found at the beginning of \texttt{hot\_pumpkin.C}\,.
It starts counting integers $i$ and calculates $N=61i+19$.
Then it checks the modulus for both 60 and 59.
If it's a match, then it outputs the results.
The first few results are 136,232; 352,172; 568,112; and 999,992.
The solutions are all of the form $N=136,232+215,940n$ for any non-negative integer $n$.
Notably, $215,940=61*60*59$.
So the solution to the first part is
\fcolorbox{red}{white}{\bf 136,232}\,.

For the second part, I decided to play out the game by hand.
My work can be seen in \texttt{Hot\_pumpkin.xlsx}\,.
The calculation was made much easier, because it's only necessary to pass the pumpkin based on the modulus for each round.
The modulus ($N\mod\rm{round \#}$) is calculated in the file.
I also set up the entire game in my C++ script.
In both cases, the player that won (and the solution) is
\fcolorbox{red}{white}{\bf player 58}\,.

For the third part, I extended my work from part two in the C++ script.
It starts counting $N$ from 1, and calculates the winner for a game with 61 players.
For fun, I outputted all the winners up through a win for player 1.
The solution is that player 1 wins for
\fcolorbox{red}{white}{$\bm{ N=120}$}\,.
All the other winners for a 61-player game are shown in the table below.
Interestingly, the original game could have been shortened by a whole lot by choosing $N=84$, since player 58 would have won anyway.

\begin{center}
\begin{tabular*}{0.8\textwidth}{@{\extracolsep{\fill}} cccccccc}
$N$ & Winner & $N$ & Winner & $N$ & Winner & $N$ & Winner \\
\cline{1-2} \cline{3-4} \cline{5-6} \cline{7-8}
1  & 61 & 31 & 12 & 61 & 33 & 91  & 60 \\
2  & 59 & 32 & 59 & 62 & 19 & 92  & 22 \\
3  & 10 & 33 & 4  & 63 & 14 & 93  & 25 \\
4  & 27 & 34 & 24 & 64 & 6  & 94  & 19 \\
5  & 49 & 35 & 28 & 65 & 13 & 95  & 23 \\
6  & 7  & 36 & 22 & 66 & 7  & 96  & 33 \\
7  & 35 & 37 & 57 & 67 & 39 & 97  & 18 \\
8  & 46 & 38 & 20 & 68 & 39 & 98  & 23 \\
9  & 48 & 39 & 35 & 69 & 30 & 99  & 50 \\
10 & 22 & 40 & 56 & 70 & 44 & 100 & 50 \\
11 & 14 & 41 & 30 & 71 & 30 & 101 & 7  \\
12 & 6  & 42 & 31 & 72 & 5  & 102 & 12 \\
13 & 61 & 43 & 12 & 73 & 31 & 103 & 33 \\
14 & 10 & 44 & 30 & 74 & 28 & 104 & 40 \\
15 & 15 & 45 & 41 & 75 & 20 & 105 & 41 \\
16 & 53 & 46 & 59 & 76 & 26 & 106 & 48 \\
17 & 19 & 47 & 4  & 77 & 48 & 107 & 18 \\
18 & 25 & 48 & 42 & 78 & 20 & 108 & 24 \\
19 & 18 & 49 & 12 & 79 & 19 & 109 & 12 \\
20 & 20 & 50 & 51 & 80 & 7  & 110 & 31 \\
21 & 13 & 51 & 2  & 81 & 40 & 111 & 51 \\
22 & 53 & 52 & 41 & 82 & 15 & 112 & 20 \\
23 & 56 & 53 & 24 & 83 & 59 & 113 & 42 \\
24 & 16 & 54 & 34 & 84 & 58 & 114 & 25 \\
25 & 35 & 55 & 44 & 85 & 47 & 115 & 27 \\
26 & 16 & 56 & 46 & 86 & 13 & 116 & 37 \\
27 & 47 & 57 & 51 & 87 & 11 & 117 & 17 \\
28 & 49 & 58 & 49 & 88 & 21 & 118 & 53 \\
29 & 49 & 59 & 2  & 89 & 20 & 119 & 3  \\
30 & 3  & 60 & 37 & 90 & 5  & 120 & 1  \\
\end{tabular*}
\end{center}





\end{document}