\documentclass{article}

\usepackage{amsmath} % math stuff
\usepackage{amssymb} % math stuff
\usepackage{array} % equations and stuff
\usepackage{bm} % bold math
%\usepackage{caption} % suppressed table numbering; incompatible with revtex, and longtable, I think
\usepackage{comment} % comment environment
%\usepackage{enumitem} % customization of enumeration, itemize, and description
\usepackage[T1]{fontenc} % font encoding for special characters, must also use scalable font package
\usepackage[margin=0.8in]{geometry} % paper sizes and margins (but be careful not to mess up pre-defined pages)
\usepackage{graphicx} % for graphics
%\usepackage{helvet} % default font is the helvetica postscript font
\usepackage{layouts} % print units like widths
\usepackage{lipsum} % lorem ipsum filler text
\usepackage{lmodern} % scalable font?
\usepackage{longtable} % multi-page tables
\usepackage{makecell} % specify line-breaks in table cells
\usepackage{mathrsfs} % math script font
\usepackage{mhchem} % easier chemical formula
\usepackage{microtype} % allows disabling of ligatures
%\usepackage{newcent} % new century schoolbook font
\usepackage{nicefrac}
\usepackage{numprint} % print and format (large) numbers
\usepackage{parskip} % removes paragraph indentation, and adjusts paragraph skip, as well as list items
\usepackage{pdfpages} % add pdf files as pages
%\usepackage{setspace} % adjust text spacing and indents
\usepackage{siunitx} % decimal alignment
\usepackage{subfigure} % divided figures
%\usepackage{tabu} % extra table options
\usepackage{textcomp} % symbols
\usepackage{threeparttablex} % better footnotes with longtable
\usepackage{titling} % title placement
\usepackage{ulem} % strikethrough text
%\usepackage{url} % superceded by hyperref
\usepackage{verbatim} % verbatim environment
\usepackage{xcolor} % colors and color boxes
\usepackage{xspace} % commands that don't eat up white space
\usepackage{hyperref} % links and page setup; should always come last

\hypersetup{
 bookmarks=true,
 colorlinks=true,
 citecolor=blue,
 linkcolor=blue,
 urlcolor=blue,
 pdfstartview={XYZ null null 1.0} % default open view is 100%
}

\DisableLigatures[f,t]{encoding = T1} % disable ff, fi, fl, tt ligatures, without f option, it also disables -- = endash
\renewcommand{\arraystretch}{2.0} % extra vertical space in tables

\begin{document}

\pagestyle{empty} % don't number pages

% custom title
\begin{center}
{\LARGE Express Riddler}

\vspace{0.15in}

{\Large 6 September 2019}
\end{center}


\section*{Riddle:}

You and your friend are playing a game of ``Acchi, Muite, Hoi'' (or what some on Twitter have been calling the ``lookaway challenge'').
In the first round, you point in one of four directions: up, down, left or right.
At the exact same time, your friend also looks in one of those four directions.
If your friend looks in the same direction you're pointing, you win!
Otherwise, you switch roles as the game continues to the next round---now your friend points in a direction and you try to look away.
As long as no one wins, you keep switching off who points and who looks.

It \textit{seems} like your chances of winning should be 50 percent, since there are exactly two players.
But surely it's not that simple.
If both you and your friend choose your directions randomly in each round, what are your chances of winning?


\section*{Solution:}

In the first round, you have a \nicefrac{1}{4} chance of winning.
The other \nicefrac{3}{4} of the time, the game continues with your opponent now having the same \nicefrac{1}{4} chance.
If no one wins after two rounds, the game essentially starts over from the beginning.
So the entire game can be represented by the first two rounds.
You have a \nicefrac{1}{4} chance of winning, and your opponent has a $\nicefrac{3}{4}*\nicefrac{1}{4}=\nicefrac{3}{16}$ chance of winning.
That gives you a 4:3 advantage over your opponent, and normalizing these probabilities gives you a chance of wininnging of
\fcolorbox{red}{white}{\bf\nicefrac{4}{7}}\,.
With increasingly more than four options to choose, this chance of winning would approach \nicefrac{1}{2}.



\end{document}