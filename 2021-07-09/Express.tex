\documentclass{article}

\usepackage{amsmath} % math stuff
\usepackage{amssymb} % math stuff
\usepackage{array} % equations and stuff
\usepackage{bm} % bold math
%\usepackage{booktabs} % extra table rule options
%\usepackage{caption} % suppressed table numbering; incompatible with revtex, and longtable, I think
\usepackage{comment} % comment environment
%\usepackage{enumitem} % customization of enumeration, itemize, and description
\usepackage[T1]{fontenc} % font encoding for special characters, must also use scalable font package
\usepackage[margin=0.8in]{geometry} % paper sizes and margins (but be careful not to mess up pre-defined pages)
\usepackage{graphicx} % for graphics
%\usepackage{helvet} % default font is the helvetica postscript font
\usepackage[utf8]{inputenc} % special characters in tex input
\usepackage{layouts} % print units like widths
\usepackage{lipsum} % lorem ipsum filler text
\usepackage{lmodern} % scalable font?
\usepackage{longtable} % multi-page tables
\usepackage{makecell} % specify line-breaks in table cells
\usepackage{mathrsfs} % math script font
\usepackage{mhchem} % easier chemical formula
\usepackage{microtype} % allows disabling of ligatures
\usepackage{multicol} % multicolumns
%\usepackage{newcent} % new century schoolbook font
\usepackage{nicefrac}
\usepackage{numprint} % print and format (large) numbers
\usepackage{parskip} % removes paragraph indentation, and adjusts paragraph skip, as well as list items
\usepackage{pdfpages} % add pdf files as pages
%\usepackage{setspace} % adjust text spacing and indents
\usepackage{siunitx} % decimal alignment
\usepackage{subfigure} % divided figures
%\usepackage{tabu} % extra table options
\usepackage{textcomp} % symbols
\usepackage{threeparttablex} % better footnotes with longtable
\usepackage{titling} % title placement
\usepackage{ulem} % strikethrough text
%\usepackage{url} % superceded by hyperref
\usepackage{verbatim} % verbatim environment
\usepackage{xcolor} % colors and color boxes
\usepackage{xspace} % commands that don't eat up white space
\usepackage{hyperref} % links and page setup; should always come last

\hypersetup{
 bookmarks=true,
 colorlinks=true,
 citecolor=blue,
 linkcolor=blue,
 urlcolor=blue,
 pdfstartview={XYZ null null 1.0} % default open view is 100%
}

\DisableLigatures[f,t]{encoding = T1} % disable ff, fi, fl, tt ligatures; without options, it also disables -- = endash
\renewcommand{\arraystretch}{1.0} % extra vertical (and horizontal?) space in tables

% define centered, left- and right-aligned columns with specified widths
\newcommand{\PreserveBackslash}[1]{\let\temp=\\#1\let\\=\temp}
\newcolumntype{C}[1]{>{\PreserveBackslash\centering}p{#1}}
\newcolumntype{L}[1]{>{\PreserveBackslash\raggedright}p{#1}}
\newcolumntype{R}[1]{>{\PreserveBackslash\raggedleft}p{#1}}

\begin{document}

\pagestyle{empty} % don't number pages

% custom title
\begin{center}
{\LARGE Express Riddler}

\vspace{0.15in}

{\Large 9 July 2021}
\end{center}


\section*{Riddle:}

Earlier this year, a new generation of Brood X cicadas had emerged in many parts of the U.S.
This particular brood emerges every 17 years, while some other cicada broods emerge every 13 years.
Both 13 and 17 are prime numbers---and relatively prime with one another---which means these broods are rarely in phase with other predators or each other.
In fact, cicadas following a 13-year cycle and cicadas following a 17-year cycle will only emerge in the same season once every 221 (i.e., 13 times 17) years!

Now, suppose there are two broods of cicadas, with periods of $A$ and $B$ years, that have just emerged in the same season.
However, these two broods can also interfere with each other one year \textit{after} they emerge due to a resulting lack of available food.
For example, if $A$ is 5 and $B$ is 7, then $B$'s emergence in year 14 (i.e., 2 times 7) means that when $A$ emerges in year 15 (i.e., 3 times 5) there won't be enough food to go around.

If both $A$ and $B$ are relatively prime and are both less than or equal to 20, what is the longest stretch these two broods can go without interfering with one another's cycle?
(Remember, both broods just emerged this year.)
For example, if $A$ is 5 and $B$ is 7, then the interference would occur in year 15.


\section*{Solution:}

I basically solved this problem by hand.
I listed out all of the multiples of the numbers 11 through 20, up to 400 ($20^{2}$).
Then I just looked for interfering multiples for pairs of distinct numbers.
Numbers less than 11 would have the same (or earlier) interferences because they already divide at least one number in the range 11--20, so it wasn't necessary to consider them.

It wasn't actually necessary to look for these multiples for all pairs of numbers.
For pairs of adjacent numbers of the form $n$-$(n+1)$, the first interference will always be after $n+1$ years, of which the maximum for this problem is 20 (for the pair 19-20).
For pairs of numbers that differ by two (or four, six, or eight), pairs of even numbers didn't need to be considered, since they are not co-prime.
Likewise for pairs of numbers (in the range 11--20) which are both multiples of three, four, five, and six.

It turns out the largest possible interference year occurs for the pair 17-19.
This interference year is 153, which is $17\times9$, and occurs immediately after $19\times8=152$.
So the solution is
\fcolorbox{red}{white}{\bf 153}\,.




\end{document}