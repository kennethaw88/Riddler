\documentclass{article}

\usepackage{amsmath} % math stuff
\usepackage{amssymb} % math stuff
\usepackage{array} % equations and stuff
\usepackage{bm} % bold math
%\usepackage{booktabs} % extra table rule options
%\usepackage{caption} % suppressed table numbering; incompatible with revtex, and longtable, I think
\usepackage{comment} % comment environment
%\usepackage{enumitem} % customization of enumeration, itemize, and description
\usepackage[T1]{fontenc} % font encoding for special characters, must also use scalable font package
\usepackage[margin=0.8in]{geometry} % paper sizes and margins (but be careful not to mess up pre-defined pages)
\usepackage{graphicx} % for graphics
%\usepackage{helvet} % default font is the helvetica postscript font
\usepackage[utf8]{inputenc} % special characters in tex input
\usepackage{layouts} % print units like widths
\usepackage{lipsum} % lorem ipsum filler text
\usepackage{lmodern} % scalable font?
\usepackage{longtable} % multi-page tables
\usepackage{makecell} % specify line-breaks in table cells
\usepackage{mathrsfs} % math script font
\usepackage{mhchem} % easier chemical formula
\usepackage{microtype} % allows disabling of ligatures
\usepackage{multicol} % multicolumns
%\usepackage{newcent} % new century schoolbook font
\usepackage{nicefrac}
\usepackage{numprint} % print and format (large) numbers
\usepackage{parskip} % removes paragraph indentation, and adjusts paragraph skip, as well as list items
\usepackage{pdfpages} % add pdf files as pages
%\usepackage{setspace} % adjust text spacing and indents
\usepackage{siunitx} % decimal alignment
\usepackage{subfigure} % divided figures
%\usepackage{tabu} % extra table options
\usepackage{textcomp} % symbols
\usepackage{threeparttablex} % better footnotes with longtable
\usepackage{titling} % title placement
\usepackage{ulem} % strikethrough text
%\usepackage{url} % superceded by hyperref
\usepackage{verbatim} % verbatim environment
\usepackage{xcolor} % colors and color boxes
\usepackage{xspace} % commands that don't eat up white space
\usepackage{hyperref} % links and page setup; should always come last

\hypersetup{
 bookmarks=true,
 colorlinks=true,
 citecolor=blue,
 linkcolor=blue,
 urlcolor=blue,
 pdfstartview={XYZ null null 1.0} % default open view is 100%
}

\DisableLigatures[f,t]{encoding = T1} % disable ff, fi, fl, tt ligatures; without options, it also disables -- = endash
\renewcommand{\arraystretch}{1.0} % extra vertical (and horizontal?) space in tables

% define centered, left- and right-aligned columns with specified widths
\newcommand{\PreserveBackslash}[1]{\let\temp=\\#1\let\\=\temp}
\newcolumntype{C}[1]{>{\PreserveBackslash\centering}p{#1}}
\newcolumntype{L}[1]{>{\PreserveBackslash\raggedright}p{#1}}
\newcolumntype{R}[1]{>{\PreserveBackslash\raggedleft}p{#1}}

\begin{document}

\pagestyle{empty} % don't number pages

% custom title
\begin{center}
{\LARGE Express Riddler}

\vspace{0.15in}

{\Large 9 July 2021}
\end{center}


\section*{Riddle:}

The astronomers of Planet Xiddler are back!

This time, they have identified three planets that circularly orbit a neighboring star.
Planet A is three astronomical units away from its star and completes its orbit in three years.
Planet B is four astronomical units away from the star and completes its orbit in four years.
Finally, Planet C is five astronomical units away from the star and completes its orbit in five years.
They report their findings to Xiddler's Grand Minister, along with the auspicious news that all three planets are currently lined up (i.e., they are collinear) with their star.
However, the Grand Minister is far more interested in the three planets than the star and wants to know how long it will be until the planets are next aligned.

How many years will it be until the three planets are again collinear (not necessarily including the star)?


\section*{Solution:}

The angular frequencies of the three planets are

\[
\omega_{A}=2\pi\left(\frac{1}{3}\right)
\]
\[
\omega_{B}=2\pi\left(\frac{1}{4}\right)
\]
\[
\omega_{C}=2\pi\left(\frac{1}{5}\right)
\]

However, because it isn't specified in the problem, the planets aren't necessarily orbiting in the same direction.
That means that each of the frequencies could be negative or positive, depending on whether each planet is moving in a clockwise or counterclockwise direction.
There are eight possible combinations of orbits.
Without loss of generality, I will assume that planet C moves counterclockwise, leaving four possibilities for A and B.
The other four cases with C moving clockwise will result in the same four final answers.

To simplify the problem, I will consider the positions of A and B relative to a fixed position for C.
This means that A and B will orbit with modified frequencies while C does not move.
Specifically, the modified frequencies are

\[
\omega_{A}'=2\pi\left(\pm\frac{1}{3}-\frac{1}{5}\right)
\]
\[
\omega_{B}'=2\pi\left(\pm\frac{1}{4}-\frac{1}{5}\right)
\]

If the circular orbits are centered at the origin (and the x-axis directed at C), then the positions of the planets are

\begin{align*}
\left(x_{A}(t),y_{A}(t)\right)&=\left(3\cos(\omega_{A}'t),3\sin(\omega_{A}'t)\right) \\
\left(x_{B}(t),y_{B}(t)\right)&=\left(4\cos(\omega_{B}'t),4\sin(\omega_{B}'t)\right) \\
\left(x_{C}(t),y_{C}(t)\right)&=(5,0)
\end{align*}

With these equations the planets are aligned along the x-axis at time $t=0$, with $t$ measured in years.
From the equations I can determine the slope of the lines passing through C and either A or B.
If the slopes of these lines are the same, then they are the same line, and the planets are collinear.
To determine the slope $m$ from two points, I use the equation

\[
m=\frac{y_{1}-y_{2}}{x_{1}-x{2}}
\]

so that the two slopes between A and C and B and C are

\begin{align*}
m_{AC}&=\frac{-3\sin(\omega_{A}'t)}{5-3\cos(\omega_{A}'t)} \\
m_{BC}&=\frac{-4\sin(\omega_{B}'t)}{5-4\cos(\omega_{B}'t)}
\end{align*}

Setting these two equations equal to each other allows solving for $t$, which I did using Wolfram Alpha.
As mentioned before, there are four possible solutions depending on the orbits of A and B.
If both A and B are moving clockwise ($\omega{A}$ and $\omega_{B}$ are negative), then the first solution for $t$ after 0 is (approximately)
\fcolorbox{red}{white}{\bf 1.6336 years}\,.
If A is moving clockwise and B is moving counterclockwise ($\omega{A}$ is negative and $\omega_{B}$ is positive), then the first solution is
\fcolorbox{red}{white}{\bf 5.3061 years}\,.
If A is moving counterclockwise and B is moving clockwise ($\omega{A}$ is positive and $\omega_{B}$ is negative), then the first solution is
\fcolorbox{red}{white}{\bf 1.5996 years}\,.
Finally, if both A and B are moving counterclockwise ($\omega{A}$ and $\omega_{B}$ are positive), then the first solution is
\fcolorbox{red}{white}{\bf 7.7668 years}\,.
The solutions are periodic, since all the planets will line up on the same axis with the sun again every 60 years, and there are multiple solutions within a period.
But these are simply the first times that the planets line up in any orientation.



\end{document}