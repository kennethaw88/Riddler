\documentclass{article}


\usepackage{amsmath} % math stuff
\usepackage{amssymb} % math stuff
\usepackage{array} % equations and stuff
\usepackage{bm} % bold math
%\usepackage{caption} % suppressed table numbering; incompatible with revtex, and longtable, I think
\usepackage{comment} % comment environment
%\usepackage{enumitem} % customization of enumeration, itemize, and description
\usepackage[T1]{fontenc} % font encoding for special characters, must also use scalable font package
\usepackage[margin=0.8in]{geometry} % paper sizes and margins (but be careful not to mess up pre-defined pages)
\usepackage{graphicx} % for graphics
%\usepackage{helvet} % default font is the helvetica postscript font
\usepackage{lipsum} % lorem ipsum filler text
\usepackage{lmodern} % scalable font?
\usepackage{longtable} % multi-page tables
\usepackage{mathrsfs} % math script font
\usepackage{mhchem} % easier chemical formula
\usepackage{microtype} % allows disabling of ligatures
%\usepackage{newcent} % new century schoolbook font
\usepackage{nicefrac}
\usepackage{parskip} % removes paragraph indentation, and adjusts paragraph skip, as well as list items
%\usepackage{setspace} % adjust text spacing and indents
\usepackage{siunitx} % decimal alignment
\usepackage{subfigure} % divided figures
%\usepackage{tabu} % extra table options
\usepackage{textcomp} % symbols
\usepackage{threeparttablex} % better footnotes with longtable
\usepackage{titling} % title placement
\usepackage{ulem} % strikethrough text
%\usepackage{url} % superceded by hyperref
\usepackage{verbatim} % verbatim environment
\usepackage{xcolor} % colors and color boxes
\usepackage{xspace} % commands that don't eat up white space
\usepackage{hyperref} % links and page setup; should always come last

\hypersetup{
	bookmarks=true,
	colorlinks=true,
	citecolor=blue,
	linkcolor=blue,
	urlcolor=blue,
	pdfstartview={XYZ null null 1.0} % default open view is 100%
}

\DisableLigatures[f]{encoding = *, family = * } % disable ff, fi, fl ligatures, without f option, it also disables -- = endash
\renewcommand{\arraystretch}{1} % extra vertical space in tables

\begin{document}

\pagestyle{empty} % don't number pages

% custom title
\begin{center}
{\LARGE Express Riddler}

\vspace{0.15in}

{\Large 26 June 2020}
\end{center}


\section*{Riddle:}

In Riddler City, the city streets follow a grid layout, running north-south and east-west. You're driving north when you decide to play a little game.
Every time you reach an intersection, you randomly turn left or right, each with a 50 percent chance.

After driving through 10 intersections, what is the probability that you are still driving north?

\textit{Extra credit}: Now suppose that at every intersection, there's a one-third chance you turn left, a one-third chance you turn right and a one-third chance you drive straight.
After driving through 10 intersections, \textit{now} what's the probability that you are still driving north?

\section*{Solution:}

This riddle is really straightforward.
The outcome of turning from north is the same as turning from south; both lead to driving east or west, each with 50\% probability.
Similarly, turning from east or west both result in driving north or south with 50\% probability.
Therefore, the total trip is just a pattern, alternating between north/south and east/west each turn.
So if you start driving north, after 10 or any number of even intersections, you are driving north or south, each still with 50\% probability.
So the solution is
\fcolorbox{red}{white}{\bf 50\%}\,.

The extra credit requires actual math.
From each direction, we can easily figure out the probabilities to go in any direction.
From north, each of west, north, and east has a \nicefrac{1}{3} probability, and so on.
We can turn that into a matrix, with each row corresponding to an incoming direction, and each column an outgoing direction.
The value at a given position is the probability of turning to an outgoing direction given an incoming direction.
Calling the matrix $A$, and letting the rows and columns be arranged in the order north, east, south, and west, we have:

\[
A=
\begin{bmatrix}
\nicefrac{1}{3} & \nicefrac{1}{3} & 0 & \nicefrac{1}{3} \\
\nicefrac{1}{3} & \nicefrac{1}{3} & \nicefrac{1}{3} & 0 \\
0 & \nicefrac{1}{3} & \nicefrac{1}{3} & \nicefrac{1}{3} \\
\nicefrac{1}{3} & 0 & \nicefrac{1}{3} & \nicefrac{1}{3} 
\end{bmatrix}
= \frac{1}{3}
\begin{bmatrix}
1 & 1 & 0 & 1 \\
1 & 1 & 1 & 0 \\
0 & 1 & 1 & 1 \\
1 & 0 & 1 & 1
\end{bmatrix}
\]

To find the probabilities for a final direction, we need to multiply this by a $4\times1$ vector that describes the initial distributions of driving directions.
In this case, the direction is only north, so the vector $v$ becomes

\[
v=
\begin{bmatrix}
1 \\
0 \\
0 \\
0
\end{bmatrix}
\]

Multiplying this by the matrix ($Av$) gives us another $4\times1$ vector that lists the probabilities for each direction after a single turn.
The probability distrubtion after $n$ turns just means $A$ gets multiplied $n$ times, so the calculation becomes $A^{n}v$.
We can separate these terms and calculate $A^{n}$ before applying it to $v$.
Luckily, there is an easy general form for $A^{n}$ in this case.

\[
A^{n}=
\begin{cases}
\frac{1}{3^{n}}
\begin{bmatrix}
a & b & b & b \\
b & a & b & b \\
b & b & a & b \\
b & b & b & a
\end{bmatrix} & n\ \mathrm{even} \\
\frac{1}{3^{n}}
\begin{bmatrix}
a & a & b & a \\
a & a & a & b \\
b & a & a & a \\
a & b & a & a
\end{bmatrix} & n\ \mathrm{odd}
\end{cases}
\]

where
\[
a=\left\lceil \frac{3^{n}}{4}\right\rceil
\]
\[
b=\left\lfloor \frac{3^{n}}{4}\right\rfloor
\]

For $n=10$, the final probability for starting north and ending north ($\nicefrac{a(10)}{3^{10}}$) is $\nicefrac{14763}{59049}\approx0.25001$, which is very close to \nicefrac{1}{4}.
Of course, as the number of turns gets large the probabilities all start to smear together and approach \nicefrac{1}{4}, but the initial direction always keeps a very slight advantage.
So the solution to the extra credit is
\fcolorbox{red}{white}{\bf \nicefrac{14763}{59049}}\,.


\end{document}