\documentclass{article}

\usepackage{amsmath} % math stuff
\usepackage{amssymb} % math stuff
\usepackage{array} % equations and stuff
\usepackage{bm} % bold math
%\usepackage{booktabs} % extra table rule options
%\usepackage{caption} % suppressed table numbering; incompatible with revtex, and longtable, I think
\usepackage{comment} % comment environment
%\usepackage{enumitem} % customization of enumeration, itemize, and description
\usepackage[T1]{fontenc} % font encoding for special characters, must also use scalable font package
\usepackage[margin=0.8in]{geometry} % paper sizes and margins (but be careful not to mess up pre-defined pages)
\usepackage{graphicx} % for graphics
%\usepackage{helvet} % default font is the helvetica postscript font
\usepackage[utf8]{inputenc} % special characters in tex input
\usepackage{layouts} % print units like widths
\usepackage{lipsum} % lorem ipsum filler text
\usepackage{lmodern} % scalable font?
\usepackage{longtable} % multi-page tables
\usepackage{makecell} % specify line-breaks in table cells
\usepackage{mathrsfs} % math script font
\usepackage{mhchem} % easier chemical formula
\usepackage{microtype} % allows disabling of ligatures
\usepackage{multicol} % multicolumns
%\usepackage{newcent} % new century schoolbook font
\usepackage{nicefrac}
\usepackage{numprint} % print and format (large) numbers
\usepackage{parskip} % removes paragraph indentation, and adjusts paragraph skip, as well as list items
\usepackage{pdfpages} % add pdf files as pages
%\usepackage{setspace} % adjust text spacing and indents
\usepackage{siunitx} % decimal alignment
\usepackage{subfigure} % divided figures
%\usepackage{tabu} % extra table options
\usepackage{textcomp} % symbols
\usepackage{threeparttablex} % better footnotes with longtable
\usepackage{titling} % title placement
\usepackage{ulem} % strikethrough text
%\usepackage{url} % superceded by hyperref
\usepackage{verbatim} % verbatim environment
\usepackage{xcolor} % colors and color boxes
\usepackage{xspace} % commands that don't eat up white space
\usepackage{hyperref} % links and page setup; should always come last

\hypersetup{
 bookmarks=true,
 colorlinks=true,
 citecolor=blue,
 linkcolor=blue,
 urlcolor=blue,
 pdfstartview={XYZ null null 1.0} % default open view is 100%
}

\DisableLigatures[f,t]{encoding = T1} % disable ff, fi, fl, tt ligatures; without options, it also disables -- = endash
\renewcommand{\arraystretch}{1.0} % extra vertical (and horizontal?) space in tables

% define centered, left- and right-aligned columns with specified widths
\newcommand{\PreserveBackslash}[1]{\let\temp=\\#1\let\\=\temp}
\newcolumntype{C}[1]{>{\PreserveBackslash\centering}p{#1}}
\newcolumntype{L}[1]{>{\PreserveBackslash\raggedright}p{#1}}
\newcolumntype{R}[1]{>{\PreserveBackslash\raggedleft}p{#1}}

\begin{document}

\pagestyle{empty} % don't number pages

% custom title
\begin{center}
{\LARGE Classic Riddler}

\vspace{0.15in}

{\Large 25 June 2021}
\end{center}


\section*{Riddle:}

Riddler solitaire is played with 11 cards: an ace, a two, a three, a four, a five, a six, a seven, an eight, a nine, a 10 and a joker.
Each card is worth its face value in points, while the ace counts for 1 point.
To play a game, you shuffle the cards so they are randomly ordered, and then turn them over one by one.
You start with 0 points, and as you flip over each card your score increases by that card's points---as long as the joker hasn't shown up.
The moment the joker appears, the game is over and your score is 0.
The key is that you can stop any moment and walk away with a nonzero score.

What strategy maximizes your expected number of points?

\textit{Extra credit}: With an optimal strategy, how many points would you earn on average in a game of Riddler solitaire?


\section*{Solution:}

The way to solve this problem is by calculating the expected change in score by drawing a single card.
For a given current score $N$ and number of cards $C$ still left in the deck, the expected change in score can be relatively easily calculated.
Because $N$ and $C$ are the only known quantities during the game, the answer must only depend on these two values.
If the expected change in score $E(N,C)$ is positive for a given $N$ and $C$, then the best strategy is to draw another card.
Otherwise, if the expected change in score is negative, the best strategy is to stop.

I calculated $E(N,C)$ for all possible pairs of $N$ and $C$ in the game.
The value of $N$ ranges from 0 (start of game, no cards drawn yet) to 55 (all ten numbered cards drawn, with only the joker left).
Several combinations of $N$ and $C$ are not possible; for example, if only one card has been drawn, any score above 10 is not possible.
To calculate $E(N,C)$, first the difference of the current score ($N$) from 55 is calculated and divided by $C-1$, which gives the average score added by drawing one of the remaining numbered cards.
This is then weighted by a factor of \nicefrac{(C-1)}{C}, which is the probability of drawing a numbered card.
Then the current score is divided by $C$ and subtracted, giving the weighted average chance of drawing the joker and dropping the current to 0.
The overal result for $E$ is

\[
E(N,C)=\frac{55-2N}{C}
\]

The results are located in \texttt{Poker.xlsx}.
All possible values are checked if they are positive or negative.
(Luckily, there was no result with $E=0$, which would be ambiguous strategy-wise.)
It turns out, the best strategy is to continue taking cards as long as the score is below 28.
Interestingly, this strategy turns out not to depend on the number of cards left.
So the solution is to
\fcolorbox{red}{white}{\bf draw cards as long as your score is below 28, and stop at 28 or above}\,.




\end{document}