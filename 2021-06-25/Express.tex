\documentclass{article}

\usepackage{amsmath} % math stuff
\usepackage{amssymb} % math stuff
\usepackage{array} % equations and stuff
\usepackage{bm} % bold math
%\usepackage{booktabs} % extra table rule options
%\usepackage{caption} % suppressed table numbering; incompatible with revtex, and longtable, I think
\usepackage{comment} % comment environment
%\usepackage{enumitem} % customization of enumeration, itemize, and description
\usepackage[T1]{fontenc} % font encoding for special characters, must also use scalable font package
\usepackage[margin=0.8in]{geometry} % paper sizes and margins (but be careful not to mess up pre-defined pages)
\usepackage{graphicx} % for graphics
%\usepackage{helvet} % default font is the helvetica postscript font
\usepackage[utf8]{inputenc} % special characters in tex input
\usepackage{layouts} % print units like widths
\usepackage{lipsum} % lorem ipsum filler text
\usepackage{lmodern} % scalable font?
\usepackage{longtable} % multi-page tables
\usepackage{makecell} % specify line-breaks in table cells
\usepackage{mathrsfs} % math script font
\usepackage{mhchem} % easier chemical formula
\usepackage{microtype} % allows disabling of ligatures
\usepackage{multicol} % multicolumns
%\usepackage{newcent} % new century schoolbook font
\usepackage{nicefrac}
\usepackage{numprint} % print and format (large) numbers
\usepackage{parskip} % removes paragraph indentation, and adjusts paragraph skip, as well as list items
\usepackage{pdfpages} % add pdf files as pages
%\usepackage{setspace} % adjust text spacing and indents
\usepackage{siunitx} % decimal alignment
\usepackage{subfigure} % divided figures
%\usepackage{tabu} % extra table options
\usepackage{textcomp} % symbols
\usepackage{threeparttablex} % better footnotes with longtable
\usepackage{titling} % title placement
\usepackage{ulem} % strikethrough text
%\usepackage{url} % superceded by hyperref
\usepackage{verbatim} % verbatim environment
\usepackage{xcolor} % colors and color boxes
\usepackage{xspace} % commands that don't eat up white space
\usepackage{hyperref} % links and page setup; should always come last

\hypersetup{
 bookmarks=true,
 colorlinks=true,
 citecolor=blue,
 linkcolor=blue,
 urlcolor=blue,
 pdfstartview={XYZ null null 1.0} % default open view is 100%
}

\DisableLigatures[f,t]{encoding = T1} % disable ff, fi, fl, tt ligatures; without options, it also disables -- = endash
\renewcommand{\arraystretch}{1.0} % extra vertical (and horizontal?) space in tables

% define centered, left- and right-aligned columns with specified widths
\newcommand{\PreserveBackslash}[1]{\let\temp=\\#1\let\\=\temp}
\newcolumntype{C}[1]{>{\PreserveBackslash\centering}p{#1}}
\newcolumntype{L}[1]{>{\PreserveBackslash\raggedright}p{#1}}
\newcolumntype{R}[1]{>{\PreserveBackslash\raggedleft}p{#1}}

\begin{document}

\pagestyle{empty} % don't number pages

% custom title
\begin{center}
{\LARGE Express Riddler}

\vspace{0.15in}

{\Large 25 June 2021}
\end{center}


\section*{Riddle:}

A bag contains 100 marbles, and each marble is one of three different colors.
If you were to draw three marbles at random, the probability that you would get one of each color is \textit{exactly} 20 percent.

How many marbles of each color are in the bag?


\section*{Solution:}

I will label the number of marbles of each color A, B, and C.
Since there are three unknowns, in principle, there must be three distinct equations that lead to a solution.
The first equation is simply the total number of marbles:

\[
A+B+C=100
\]

The second equation comes from the total probability described in the problem.
Suppose that the three marbles are drawn one at a time (without replacing them in the bag).
The chance of drawing A followed by B, then C is

\[
\frac{A}{100}\cdot\frac{B}{99}\cdot\frac{C}{98}=\frac{ABC}{970200}
\]

But of course, the three marbles could have been drawn in any order, say A-C-B, or B-A-C.
There are $3!=6$ ways to draw the three marbles, independent of order.
So the above value needs to be multiplied by 6 to get the final probability:

\[
6\cdot\frac{ABC}{970200}=\frac{1}{5}
\]

Rearranging this gives the second equation used to solve the riddle:

\[
ABC=32,340
\]

These are the only two equations that can be directly realized from the problem.
However, the other implicit condition of the problem is that A, B, and C are all integers between 0 and 100.
This reduces the available solutions and makes it uniquely solvable.

To solve the riddle, I factored 32,340 to determine what prime factors must be distributed between A, B, and C.
The prime factorization is $2\cdot2\cdot3\cdot5\cdot7\cdot7\cdot11$.
I solved the riddle by considering what multiple of 11 would work for one of the numbers.
Without loss of generality, I let A have the factor of 11.
The possible values of A are then 11, 22, 33, 44, 55, 66, and 77.
88 has too many factors of 2, and 99 has too many factors of 3.

A could not be 11, because either B or C would have to be $7^{2}=49$ ($2\cdot7^{2}=98$ is too large a sum), leaving the other with a too-large value of 60, or B and C would both be multiples of 7, which could not sum to 100.
A could not be 22, because either B or C would be 49, leaving the other with 30, or else B and C would both be multiples of 7, which could not sum to 100.
A could not be 33 for the same reason.
A could not be 55, because then B or C would have to be 49 (which is too large a sum), or
B and C would both be multiples of 7, which could not sum to 100.
A could not be 66, because 49 is too large, and no multiples of 7 could sum to 100.
Finally, A could not be 77, because 35 is too large, and no other multiple of 7 could be summed with a multiple of 5 to get 100.

That leaves A to be 44.
Removing the factors of $2\cdot2\cdot11$ leaves the remaining equations:

\[
B+C=56
\]
\[
BC=735=3\cdot5\cdot7\cdot7
\]

Simple guessing and checking reveals that (without loss of generality) B=25 and C=21.
So the solution is that A, B, and C are, in any order,
\fcolorbox{red}{white}{\bf 44, 35, and 21}\,.




\end{document}