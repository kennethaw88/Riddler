\documentclass{article}

\usepackage{amsmath} % math stuff
\usepackage{amssymb} % math stuff
\usepackage{array} % equations and stuff
\usepackage{bm} % bold math
%\usepackage{booktabs} % extra table rule options
%\usepackage{caption} % suppressed table numbering; incompatible with revtex, and longtable, I think
\usepackage{comment} % comment environment
%\usepackage{enumitem} % customization of enumeration, itemize, and description
\usepackage[T1]{fontenc} % font encoding for special characters, must also use scalable font package
\usepackage[margin=0.8in]{geometry} % paper sizes and margins (but be careful not to mess up pre-defined pages)
\usepackage{graphicx} % for graphics
%\usepackage{helvet} % default font is the helvetica postscript font
\usepackage[utf8]{inputenc} % special characters in tex input
\usepackage{layouts} % print units like widths
\usepackage{lipsum} % lorem ipsum filler text
\usepackage{lmodern} % scalable font?
\usepackage{longtable} % multi-page tables
\usepackage{makecell} % specify line-breaks in table cells
\usepackage{mathrsfs} % math script font
\usepackage{mhchem} % easier chemical formula
\usepackage{microtype} % allows disabling of ligatures
\usepackage{multicol} % multicolumns
%\usepackage{newcent} % new century schoolbook font
\usepackage{nicefrac}
\usepackage{numprint} % print and format (large) numbers
\usepackage{parskip} % removes paragraph indentation, and adjusts paragraph skip, as well as list items
\usepackage{pdfpages} % add pdf files as pages
%\usepackage{setspace} % adjust text spacing and indents
\usepackage{siunitx} % decimal alignment
\usepackage{subfigure} % divided figures
%\usepackage{tabu} % extra table options
\usepackage{textcomp} % symbols
\usepackage{threeparttablex} % better footnotes with longtable
\usepackage{titling} % title placement
\usepackage{ulem} % strikethrough text
%\usepackage{url} % superceded by hyperref
\usepackage{verbatim} % verbatim environment
\usepackage{xcolor} % colors and color boxes
\usepackage{xspace} % commands that don't eat up white space
\usepackage{hyperref} % links and page setup; should always come last

\hypersetup{
 bookmarks=true,
 colorlinks=true,
 citecolor=blue,
 linkcolor=blue,
 urlcolor=blue,
 pdfstartview={XYZ null null 1.0} % default open view is 100%
}

\DisableLigatures[f,t]{encoding = T1} % disable ff, fi, fl, tt ligatures; without options, it also disables -- = endash
\renewcommand{\arraystretch}{1.0} % extra vertical (and horizontal?) space in tables

% define centered, left- and right-aligned columns with specified widths
\newcommand{\PreserveBackslash}[1]{\let\temp=\\#1\let\\=\temp}
\newcolumntype{C}[1]{>{\PreserveBackslash\centering}p{#1}}
\newcolumntype{L}[1]{>{\PreserveBackslash\raggedright}p{#1}}
\newcolumntype{R}[1]{>{\PreserveBackslash\raggedleft}p{#1}}

\begin{document}

\pagestyle{empty} % don't number pages

% custom title
\begin{center}
{\LARGE Express Riddler}

\vspace{0.15in}

{\Large 27 August 2021}
\end{center}


\section*{Riddle:}

Football season is almost here, which means fantasy football season is already here.

You and your two best friends are in a three-person league, drafting just three positions each for your teams: quarterback, running back and wide receiver.
Yes, this is a simplified version of fantasy football.

The following table shows the top three athletes in each position, as well as the number of fantasy points they are expected to earn over the course of the season.
You and your friends must each select exactly one player from each position.

\vspace{0.1in}
\begin{center}
\begin{tabular}{|l|c|l|c|l|c|}
\hline
Quarterback & Points & Running Back & Points & Wide Receiver & Points \\
\hline
Matrick Pahomes & 400 & Caffrey McChristian & 300 & Avante Dadams & 250 \\
\hline
Osh Jallen & 350 & Calvin Dook & 225 & Hyreek Till & 225 \\
\hline
Myler Kurray & 300 & Herrick Denry & 200 & Defon Stiggs & 175 \\
\hline
\end{tabular}
\end{center}
\vspace{0.1in}

The draft is a ``snake draft.''
If person A drafts first, B drafts second and C drafts third, then the order of the picks is as follows: A-B-C-C-B-A-A-B-C.

Your friends---being the kind people that they are---agree that you can choose your pick number.
Which draft position should you choose to maximize your expected fantasy score?


\section*{Solution:}

When each person makes the first draft selection, there are three positions available to be filled.
During the second selection, there are two positions, and during the third selection, there is one position remaining.
I assume that for any given position, the drafter picks the best-available player in that position; this means that the drafters are only picking among positions, and not among all players.
Each of drafters A, B, and C has $3\times2\times1=6$ possible ways to fill their team.
With three drafters, there are $6^{3}=216$ possible ``paths'' toward choosing the teams.

But some of these paths are redundant.
Specifically, A can reverse their second and third picks, leading to the same final result; similarly C can reverse the first and second picks.
This means there are only 54 unique paths for drafting.

I manually listed each of the 54 possible paths and calculated the expected scores for each drafter.
From there, I determined the optimal choice that each player would make, starting backwards from B's second pick.
(Interestingly, A only has a single free choice; the second and third picks are whatever is leftover after B and C's second picks.)
If two choices led to paths with the same expected score, I assumed that the drafter would choose either with 50\%\ probability.
This probability could factor into earlier choices by other players.

Working back all the way to A's first choice, I determine that the best drafter is \fcolorbox{red}{white}{\textbf{B}}\,.
With the following draft choices, A and C each score 800, while B scores 825.

\vspace{0.1in}
\begin{center}
\begin{tabular}{ll}
A: & Matrick Pahomes \\
B: & Caffrey McChristian \\
C: & Osh Jallen/Avante Dadams \\
B: & Hyreek Till \\
A: & Calvin Dook/Defon Stiggs \\
B: & Myler Kurray \\
C: & Herrick Denry \\
\end{tabular}
\end{center}
\vspace{0.1in}

\end{document}