\documentclass{article}


\usepackage{amsmath} % math stuff
\usepackage{amssymb} % math stuff
\usepackage{array} % equations and stuff
\usepackage{bm} % bold math
%\usepackage{caption} % suppressed table numbering; incompatible with revtex, and longtable, I think
\usepackage{comment} % comment environment
%\usepackage{enumitem} % customization of enumeration, itemize, and description
\usepackage[T1]{fontenc} % font encoding for special characters, must also use scalable font package
\usepackage[margin=0.8in]{geometry} % paper sizes and margins (but be careful not to mess up pre-defined pages)
\usepackage{graphicx} % for graphics
%\usepackage{helvet} % default font is the helvetica postscript font
\usepackage{lipsum} % lorem ipsum filler text
\usepackage{lmodern} % scalable font?
\usepackage{longtable} % multi-page tables
\usepackage{mathrsfs} % math script font
\usepackage{microtype} % allows disabling of ligatures
%\usepackage{newcent} % new century schoolbook font
\usepackage{parskip} % removes paragraph indentation, and adjusts paragraph skip, as well as list items
%\usepackage{setspace} % adjust text spacing and indents
\usepackage{siunitx} % decimal alignment
\usepackage{subfigure} % divided figures
\usepackage{tabu} % extra table options
\usepackage{textcomp} % symbols
\usepackage{threeparttablex} % better footnotes with longtable
\usepackage{titling} % title placement
%\usepackage{url} % superceded by hyperref
\usepackage{verbatim} % verbatim environment
\usepackage{xcolor} % colors and color boxes
\usepackage{xspace} % commands that don't eat up white space
\usepackage{hyperref} % links and page setup; should always come last

\hypersetup{
	bookmarks=true,
	colorlinks=true,
	citecolor=blue,
	linkcolor=blue,
	urlcolor=blue,
	pdfstartview={XYZ null null 1.0} % default open view is 100%
}

\DisableLigatures[f]{encoding = *, family = * } % disable ff, fi, fl ligatures, without f option, it also disables -- = endash

%\setlength{\droptitle}{-4em} % remove huge white space above title

%\title{Riddler Express}
%\author{}
%\date{13 December 2019}

\begin{document}

\pagestyle{empty} % don't number pages

%\maketitle

% custom title
\begin{center}
{\LARGE Classic Riddler}

\vspace{0.15in}

{\Large 6 December 2019}
\end{center}


\section*{Riddle:}

You have a playlist with exactly 100 tracks (i.e., songs), numbered 1 to 100.
To go to another track, there are two buttons you can press: (1) ``Next,'' which will take you to the next track in the list or back to song 1 if you are currently on track 100, and (2) ``Random,'' which will take you to a track chosen uniformly from among the 100 tracks.
Pressing ``Random'' can restart the track you’re already listening to — this will happen 1 percent of the time you press the “Random” button.

For example, if you started on track 73, and you pressed the buttons in the sequence ``Random, Next, Random, Random, Next, Next, Random, Next,'' you might get the following sequence of track numbers: 73, 30, 31, 67, 12, 13, 14, 89, 90.
You always know the number of the track you’re currently listening to.

Your goal is to get to your favorite song (on track 42, of course) with as few button presses as possible.
What should your general strategy be?
Assuming you start on a random track, what is the average number of button presses you would need to make to reach your favorite song?



\section*{Solution:}

I will first translate the problem, so that the numbers make a bit more sense.
The desired track is now 1, and the ``Next'' button becomes ``Previous''.
Now it is a problem of counting down to 1.

It is clear that the strategy must be pressing each button only for a pre-specified range (and of course, not doing anything on track 1).
Suppose you start on track $k$.
You should always press ``Previous'' for $k\leq n$ for some specified threshold track $n$.
Thus the ``Previous'' range becomes 2--$n$.
For all other tracks, you should press ``Random''.

Within the range 2--$n$ it will always take $k-1$ clicks to get to 1.
Outside that range, the probability of clicking into the range is $n/100$, so the average numbers of clicks to get there is $100/n$.
Letting $M(k,n)$ be the expected number of clicks starting from track $k$ with threshold $n$, we see that
\begin{equation*}
M(k,n)=
\begin{cases}
k-1 & k\leq n \\
\frac{100}{n}+\frac{1}{n}\sum\limits_{m=1}^{n}(m-1) & k>n
\end{cases}
\end{equation*}

Furthur let $M(n)$ be the expected number of clicks with threshold $n$:
\begin{equation*}
M(n)=\frac{1}{100}\sum\limits_{k=1}^{100}M(k,n)
\end{equation*}

The problem is now to find the smallest value of $M(n)$.
I have set up the calculations in the spreadsheet \texttt{Song\_skipping.xlsx}.
The apparent solution is $n=14$, which gives an average of 12.6 clicks to get to track 1.
Translating this back to the original problem means the 13 tracks prior to 42 become our range, and the solution is
\fcolorbox{red}{white}{\bf Press ``Next'' for tracks 29--41, and press ``Random'' for all others.}



\end{document}