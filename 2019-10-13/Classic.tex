\documentclass{article}


\usepackage{amsmath} % math stuff
\usepackage{amssymb} % math stuff
\usepackage{array} % equations and stuff
\usepackage{bm} % bold math
%\usepackage{caption} % suppressed table numbering; incompatible with revtex, and longtable, I think
\usepackage{comment} % comment environment
%\usepackage{enumitem} % customization of enumeration, itemize, and description
\usepackage[T1]{fontenc} % font encoding for special characters, must also use scalable font package
\usepackage[margin=0.8in]{geometry} % paper sizes and margins (but be careful not to mess up pre-defined pages)
\usepackage{graphicx} % for graphics
%\usepackage{helvet} % default font is the helvetica postscript font
\usepackage{lipsum} % lorem ipsum filler text
\usepackage{lmodern} % scalable font?
\usepackage{longtable} % multi-page tables
\usepackage{mathrsfs} % math script font
\usepackage{mhchem} % easier chemical formula
\usepackage{microtype} % allows disabling of ligatures
%\usepackage{newcent} % new century schoolbook font
\usepackage{parskip} % removes paragraph indentation, and adjusts paragraph skip, as well as list items
%\usepackage{setspace} % adjust text spacing and indents
\usepackage{siunitx} % decimal alignment
\usepackage{subfigure} % divided figures
%\usepackage{tabu} % extra table options
\usepackage{textcomp} % symbols
\usepackage{threeparttablex} % better footnotes with longtable
\usepackage{titling} % title placement
%\usepackage{url} % superceded by hyperref
\usepackage{verbatim} % verbatim environment
\usepackage{xcolor} % colors and color boxes
\usepackage{xspace} % commands that don't eat up white space
\usepackage{hyperref} % links and page setup; should always come last

\hypersetup{
	bookmarks=true,
	colorlinks=true,
	citecolor=blue,
	linkcolor=blue,
	urlcolor=blue,
	pdfstartview={XYZ null null 1.0} % default open view is 100%
}

\DisableLigatures[f]{encoding = *, family = * } % disable ff, fi, fl ligatures, without f option, it also disables -- = endash

%\setlength{\droptitle}{-4em} % remove huge white space above title

%\title{Riddler Express}
%\author{}
%\date{13 December 2019}

\begin{document}

\pagestyle{empty} % don't number pages

%\maketitle

% custom title
\begin{center}
{\LARGE Classic Riddler}

\vspace{0.15in}

{\Large 13 December 2019}
\end{center}


\section*{Riddle:}

Suppose I have a rectangle whose side lengths are each a whole number, and whose area (in square units) is the same as its perimeter (in units of length). What are the possible dimensions for this rectangle?

Alas, that’s not the riddle — that’s just the appetizer. The rectangle could be 4 by 4 or 3 by 6. You can check both of these: 4 · 4 = 16 and 4 + 4 + 4 + 4 = 16, while 3 · 6 = 18 and 3 + 6 + 3 + 6 = 18. These are the only two whole number dimensions the rectangle could have. (One way to see this is to call the rectangle’s length a and its width b. You’re looking for whole number solutions to the equation ab = 2a + 2b.)

On to the main course! Instead of rectangles, let’s give rectangular prisms a try. What whole number dimensions can rectangular prisms have so that their volume (in cubic units) is the same as their surface area (in square units)?

To get you started, Steve notes that 6 by 6 by 6 is one such solution. How many others can you find?


\section*{Solution:}

This is a fairly straightforward riddle.
It amounts to solving the equation
\begin{equation*}
abc=2(ab+ac+bc)
\end{equation*}
for side lengths $a$, $b$, and $c$.
One method is to just guess and check, since the answers will all be relatively small integers; the $6\times6\times6$ example shows that the maximum possible value of the smallest side is 6.
This is because the third-order term on the left-hand side of the equation will grow faster than the second-order terms on the right-hand side.
The guess and check method is made slightly easier by the observation that at least one side must have even length, reducing the number of guesses.

Instead of guessing and checking, I substituted values of the smallest side (since it is limited to 1--6) and found the relation for the remaining two sides.
Without loss of generality, I let $a$ be the smallest side to get 6 different relations for $b$ and $c$.

\vspace{0.2in}

$a=1$:
\begin{equation*}
b=-\frac{2c}{c+2}
\end{equation*}
This clearly has no solutions in which both $b$ and $c$ are positive.

\vspace{0.2in}

$a=2$:
\begin{equation*}
b=-c
\end{equation*}
Again, this has no postive solutions.

\vspace{0.2in}

$a=3$:
\begin{equation*}
b=\frac{6c}{c-6}
\end{equation*}
Here it is clear that $c>6$ for positive solutions.
Checking all results up to $b=c$ gives
\fcolorbox{red}{white}{$c=7$, $b=42$}\,;
\fcolorbox{red}{white}{$c=8$, $b=24$}\,;
\fcolorbox{red}{white}{$c=9$, $b=18$}\,;
\fcolorbox{red}{white}{$c=10$, $b=15$}\,; and
\fcolorbox{red}{white}{$c=12$, $b=12$}\,.

\vspace{0.2in}

$a=4$:
\begin{equation*}
b=\frac{8c}{c-8}
\end{equation*}
Here, $c>4$, with solutions
\fcolorbox{red}{white}{$c=5$, $b=20$}\,;
\fcolorbox{red}{white}{$c=6$, $b=12$}\,; and
\fcolorbox{red}{white}{$c=8$, $b=8$}\,.


\vspace{0.2in}

$a=5$:
\begin{equation*}
b=\frac{10c}{3c-10}
\end{equation*}
Here, $c>3$, but I have already found solutions with sides $<5$, so this has the single solution
\fcolorbox{red}{white}{$c=5$, $b=10$}\,.

\vspace{0.2in}

$a=6$:
\begin{equation*}
b=\frac{12c}{4c-12}
\end{equation*}
Here, $c>3$, but I have already found solutions with sides $<6$, so this recovers the example solution
\fcolorbox{red}{white}{$c=6$, $b=6$}\,.

\vspace{0.2in}

Up to permutations, there are ten unique solutions to this problem.
Letting $a\leq b\leq c$, the final solutions are

\begin{center}
\begin{tabular}{| >{\centering}p{0.5in} | >{\centering}p{0.5in} | >{\centering}p{0.5in} | >{\centering}p{1.75in} |}
\hline
{\bf a} & {\bf b} & {\bf c} & {\bf Volume/Surface area} \tabularnewline \hline
3 & 7  & 42 & 882 \tabularnewline \hline
3 & 8  & 24 & 576 \tabularnewline \hline
3 & 9  & 18 & 486 \tabularnewline \hline
3 & 10 & 15 & 450 \tabularnewline \hline
3 & 12 & 12 & 432 \tabularnewline \hline
4 & 5  & 20 & 400 \tabularnewline \hline
4 & 6  & 12 & 288 \tabularnewline \hline
4 & 8  & 8  & 256 \tabularnewline \hline
5 & 5  & 10 & 250 \tabularnewline \hline
6 & 6  & 6  & 216 \tabularnewline \hline
\end{tabular}
\end{center}



\end{document}