\documentclass{article}

\usepackage{amsmath} % math stuff
\usepackage{amssymb} % math stuff
\usepackage{array} % equations and stuff
\usepackage{bm} % bold math
%\usepackage{booktabs} % extra table rule options
%\usepackage{caption} % suppressed table numbering; incompatible with revtex, and longtable, I think
\usepackage{comment} % comment environment
%\usepackage{enumitem} % customization of enumeration, itemize, and description
\usepackage[T1]{fontenc} % font encoding for special characters, must also use scalable font package
\usepackage[margin=0.8in]{geometry} % paper sizes and margins (but be careful not to mess up pre-defined pages)
\usepackage{graphicx} % for graphics
%\usepackage{helvet} % default font is the helvetica postscript font
\usepackage{layouts} % print units like widths
\usepackage{lipsum} % lorem ipsum filler text
\usepackage{lmodern} % scalable font?
\usepackage{longtable} % multi-page tables
\usepackage{makecell} % specify line-breaks in table cells
\usepackage{mathrsfs} % math script font
\usepackage{mhchem} % easier chemical formula
\usepackage{microtype} % allows disabling of ligatures
%\usepackage{newcent} % new century schoolbook font
\usepackage{nicefrac}
\usepackage{numprint} % print and format (large) numbers
\usepackage{parskip} % removes paragraph indentation, and adjusts paragraph skip, as well as list items
\usepackage{pdfpages} % add pdf files as pages
%\usepackage{setspace} % adjust text spacing and indents
\usepackage{siunitx} % decimal alignment
\usepackage{subfigure} % divided figures
%\usepackage{tabu} % extra table options
\usepackage{textcomp} % symbols
\usepackage{threeparttablex} % better footnotes with longtable
\usepackage{titling} % title placement
\usepackage{ulem} % strikethrough text
%\usepackage{url} % superceded by hyperref
\usepackage{verbatim} % verbatim environment
\usepackage{xcolor} % colors and color boxes
\usepackage{xspace} % commands that don't eat up white space
\usepackage{hyperref} % links and page setup; should always come last

\hypersetup{
 bookmarks=true,
 colorlinks=true,
 citecolor=blue,
 linkcolor=blue,
 urlcolor=blue,
 pdfstartview={XYZ null null 1.0} % default open view is 100%
}

\DisableLigatures[f,t]{encoding = T1} % disable ff, fi, fl, tt ligatures; without options, it also disables -- = endash
\renewcommand{\arraystretch}{1.0} % extra vertical (and horizontal?) space in tables

% define centered, left- and right-aligned columns with specified widths
\newcommand{\PreserveBackslash}[1]{\let\temp=\\#1\let\\=\temp}
\newcolumntype{C}[1]{>{\PreserveBackslash\centering}p{#1}}
\newcolumntype{L}[1]{>{\PreserveBackslash\raggedright}p{#1}}
\newcolumntype{R}[1]{>{\PreserveBackslash\raggedleft}p{#1}}

\begin{document}

\pagestyle{empty} % don't number pages

% custom title
\begin{center}
{\LARGE Express Riddler}

\vspace{0.15in}

{\Large 11 June 2021}
\end{center}


\section*{Riddle:}

You've been hired to design a new logo for Riddler City.
The mayor is a little eccentric and has requested that the logo have at least two lines of symmetry that intersect at an angle of precisely 1 radian, or 180/$\pi$ (approximately 57.3) degrees.

What sorts of logos meet the mayor's requirements?
(You can give one example or describe what all possible logos have in common.)


\section*{Solution:}

Logos that meet this requirement must have circular symmetry.
In other words, the logos only consist of concentric rings or circles, and a rotation of any arbitrary angle about the center results in the same logo.
One example of such a logo is the Target logo:

\vspace{0.1in}
\begin{center}
\includegraphics[width=2.5in]{target.png}
\end{center}
\vspace{0.1in}

The reason is that the demanded angle between the initial lines of symmetry is not a rational fraction of the whole circle.
Instead, the angle is transcendental because of the inclusion of $\pi$.
Once the first line of symmetry is established, creating a second line of symmetry necessitates a third line of symmetry which is a reflection of the second line across the first.
Similarly, the initial two lines create two other lines reflected across the third.
This sequence of necessary lines of symmetry continues infinitely, and each new line of symmetry is unique, never aligning with a previous line.
It is infinite because of the transcendental angle in the beginning; if the initial angle was a rational fraction of the whole circle, say 60\textdegree, or even 180\textdegree/13, then the lines would eventually line up, creating a finite set of symmetry lines.
Because there are infinitely many lines of symmetry, distributed evenly across the logo, the logo must have perfect circular symmetry.


\end{document}