\documentclass{article}

\usepackage{amsmath} % math stuff
\usepackage{amssymb} % math stuff
\usepackage{array} % equations and stuff
\usepackage{bm} % bold math
%\usepackage{caption} % suppressed table numbering; incompatible with revtex, and longtable, I think
\usepackage{comment} % comment environment
%\usepackage{enumitem} % customization of enumeration, itemize, and description
\usepackage[T1]{fontenc} % font encoding for special characters, must also use scalable font package
\usepackage[margin=0.8in]{geometry} % paper sizes and margins (but be careful not to mess up pre-defined pages)
\usepackage{graphicx} % for graphics
%\usepackage{helvet} % default font is the helvetica postscript font
\usepackage{layouts} % print units like widths
\usepackage{lipsum} % lorem ipsum filler text
\usepackage{lmodern} % scalable font?
\usepackage{longtable} % multi-page tables
\usepackage{makecell} % specify line-breaks in table cells
\usepackage{mathrsfs} % math script font
\usepackage{mhchem} % easier chemical formula
\usepackage{microtype} % allows disabling of ligatures
%\usepackage{newcent} % new century schoolbook font
\usepackage{nicefrac}
\usepackage{parskip} % removes paragraph indentation, and adjusts paragraph skip, as well as list items
\usepackage{pdfpages} % add pdf files as pages
%\usepackage{setspace} % adjust text spacing and indents
\usepackage{siunitx} % decimal alignment
\usepackage{subfigure} % divided figures
%\usepackage{tabu} % extra table options
\usepackage{textcomp} % symbols
\usepackage{threeparttablex} % better footnotes with longtable
\usepackage{titling} % title placement
\usepackage{ulem} % strikethrough text
%\usepackage{url} % superceded by hyperref
\usepackage{verbatim} % verbatim environment
\usepackage{xcolor} % colors and color boxes
\usepackage{xspace} % commands that don't eat up white space
\usepackage{hyperref} % links and page setup; should always come last

\hypersetup{
 bookmarks=true,
 colorlinks=true,
 citecolor=blue,
 linkcolor=blue,
 urlcolor=blue,
 pdfstartview={XYZ null null 1.0} % default open view is 100%
}

\DisableLigatures[f,t]{encoding = T1} % disable ff, fi, fl, tt ligatures, without f option, it also disables -- = endash
\renewcommand{\arraystretch}{2.0} % extra vertical space in tables

\begin{document}

\pagestyle{empty} % don't number pages

% custom title
\begin{center}
{\LARGE Express Riddler}

\vspace{0.15in}

{\Large 8 January 2021}
\end{center}


\section*{Riddle:}

There are many ways to slice a big square into smaller squares (not necessarily of equal size), so that the smaller squares don’t overlap, while still making up the entire area of the big square.

For example, you can slice the big square into four smaller squares, each a quarter of the area of the big square.
Or you could slice it into seven squares, if you take one of those four squares and slice it into four yet smaller squares.

What whole numbers of squares can you \textit{not} slice the big square into?


\section*{Solution:}

Each slicing operation changes a single square into another number of squares, which is always a square number (of the form $n^{2}$).
In other words, each operation adds $n^{2}-1$ squares to the total.
The final number will be obtained by a sum of consecutive slicing operations.

The first few numbers of the form $n^{2}-1$ for $n\geq2$ are 3, 8, 15, 24, and so on.
I will call these the slice numbers.
So the riddle is essentially asking for what numbers cannot be written as a sum of slice numbers.
In fact, many larger numbers have multiple ways to be written as a sum of those numbers, but only the numbers that have no sum contribute to the solution.
This question is essentially the Frobenius problem, which asks specifically for the largest number which cannot be written as a sum of numbers.
In this case, just using the first two slice numbers (3 and 8), the solution to the Frobenius problem is 13 ($=3\times8-3-8$).
So the 15, 24, and higher slice numbers don't even matter.

Besides 13, the other (positive) numbers that aren't a sum of slice numbers are 1, 2, 4, 5, 7, and 10.
Taking into account the original square as part of the riddle and shifting these numbers by 1, the solution is
\fcolorbox{red}{white}{\bf 2, 3, 5, 6, 8, 11, and 14}\,.





\end{document}