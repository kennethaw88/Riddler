\documentclass{article}

\usepackage{amsmath} % math stuff
\usepackage{amssymb} % math stuff
\usepackage{array} % equations and stuff
\usepackage{bm} % bold math
%\usepackage{caption} % suppressed table numbering; incompatible with revtex, and longtable, I think
\usepackage{comment} % comment environment
%\usepackage{enumitem} % customization of enumeration, itemize, and description
\usepackage[T1]{fontenc} % font encoding for special characters, must also use scalable font package
\usepackage[margin=0.8in]{geometry} % paper sizes and margins (but be careful not to mess up pre-defined pages)
\usepackage{graphicx} % for graphics
%\usepackage{helvet} % default font is the helvetica postscript font
\usepackage{lipsum} % lorem ipsum filler text
\usepackage{lmodern} % scalable font?
\usepackage{longtable} % multi-page tables
\usepackage{mathrsfs} % math script font
\usepackage{mhchem} % easier chemical formula
\usepackage{microtype} % allows disabling of ligatures
%\usepackage{newcent} % new century schoolbook font
\usepackage{nicefrac}
\usepackage{parskip} % removes paragraph indentation, and adjusts paragraph skip, as well as list items
%\usepackage{setspace} % adjust text spacing and indents
\usepackage{siunitx} % decimal alignment
\usepackage{subfigure} % divided figures
%\usepackage{tabu} % extra table options
\usepackage{textcomp} % symbols
\usepackage{threeparttablex} % better footnotes with longtable
\usepackage{titling} % title placement
\usepackage{ulem} % strikethrough text
%\usepackage{url} % superceded by hyperref
\usepackage{verbatim} % verbatim environment
\usepackage{xcolor} % colors and color boxes
\usepackage{xspace} % commands that don't eat up white space
\usepackage{hyperref} % links and page setup; should always come last

\hypersetup{
	bookmarks=true,
	colorlinks=true,
	citecolor=blue,
	linkcolor=blue,
	urlcolor=blue,
	pdfstartview={XYZ null null 1.0} % default open view is 100%
}

\DisableLigatures[f]{encoding = *, family = * } % disable ff, fi, fl ligatures, without f option, it also disables -- = endash
\renewcommand{\arraystretch}{1.1} % extra vertical space in tables

\begin{document}

\pagestyle{empty} % don't number pages

% custom title
\begin{center}
{\LARGE Classic Riddler}

\vspace{0.15in}

{\Large 2 October 2020}
\end{center}


\section*{Riddle:}

I have 10 chocolates in a bag: Two are milk chocolate, while the other eight are dark chocolate.
One at a time, I randomly pull chocolates from the bag and eat them---that is, until I pick a chocolate of the other kind.
When I get to the other type of chocolate, I put it back in the bag and start drawing again with the remaining chocolates.
I keep going until I have eaten all 10 chocolates.

For example, if I first pull out a dark chocolate, I will eat it.
(I'll always eat the first chocolate I pull out.)
If I pull out a second dark chocolate, I will eat that as well.
If the third one is milk chocolate, I will not eat it (yet), and instead place it back in the bag.
Then I will start again, eating the first chocolate I pull out.

What are the chances that the \textit{last} chocolate I eat is milk chocolate?

\section*{Solution:}

To solve this, I will define the probability $P(n_{d},n_{m},last)$ of eating milk chocolate last given that there are $n_{d}$ remaining dark pieces, $n_{m}$ remaining milk pieces, and that I most recently ate type $last$.
For my notation, I let $last$ be $m$ (milk), $d$ (dark), or $x$ (none/replacement step).

I can define $P$ for the possible final states, and build up the probabilities up to the case for $P(8,2,x)$.
I can start with the end cases:

\[
P(0,0,d)=0
\]
\[
P(0,0,m)=1
\]

Even more generally, I can write

\[
P(n_{d}>0,0,x)=P(n_{d}>0,0,d)=P(n_{d}>0,0,m)=0
\]
\[
P(0,n_{m}>0,s)=P(0,n_{m}>0,d)=P(0,n_{m}>0,m)=1
\]

From here, the rules for the probabilities are as follows (with $n_{d},n_{m}>0$):

\[
P(n_{d},n_{m},x)=\left(\frac{n_{d}}{n_{d}+n_{m}}\right)P(n_{d}-1,n_{m},d)+\left(\frac{n_{m}}{n_{d}+n_{m}}\right)P(n_{d},n_{m}-1,m)
\]
\[
P(n_{d},n_{m},d)=\left(\frac{n_{d}}{n_{d}+n_{m}}\right)P(n_{d}-1,n_{m},d)+\left(\frac{n_{m}}{n_{d}+n_{m}}\right)P(n_{d},n_{m},x)
\]
\[
P(n_{d},n_{m},m)=\left(\frac{n_{d}}{n_{d}+n_{m}}\right)P(n_{d},n_{m},x)+\left(\frac{n_{m}}{n_{d}+n_{m}}\right)P(n_{d},n_{m}-1,m)
\]

Carrying these calculations out gives a rather surprising result.
For any combinations of dark and milk chocolates (with both being more than zero), the probability of finishing with either chocolate is always \nicefrac{1}{2}.
So the answer to the specific riddle with $n_{d}=8$ and $n_{m}=2$ (i.e., $P(8,2,x)$) is
\fcolorbox{red}{white}{\bf \nicefrac{1}{2}}\,.


\end{document}