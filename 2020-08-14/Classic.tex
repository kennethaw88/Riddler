\documentclass{article}


\usepackage{amsmath} % math stuff
\usepackage{amssymb} % math stuff
\usepackage{array} % equations and stuff
\usepackage{bm} % bold math
%\usepackage{caption} % suppressed table numbering; incompatible with revtex, and longtable, I think
\usepackage{comment} % comment environment
%\usepackage{enumitem} % customization of enumeration, itemize, and description
\usepackage[T1]{fontenc} % font encoding for special characters, must also use scalable font package
\usepackage[margin=0.8in]{geometry} % paper sizes and margins (but be careful not to mess up pre-defined pages)
\usepackage{graphicx} % for graphics
%\usepackage{helvet} % default font is the helvetica postscript font
\usepackage{lipsum} % lorem ipsum filler text
\usepackage{lmodern} % scalable font?
\usepackage{longtable} % multi-page tables
\usepackage{mathrsfs} % math script font
\usepackage{mhchem} % easier chemical formula
\usepackage{microtype} % allows disabling of ligatures
%\usepackage{newcent} % new century schoolbook font
\usepackage{nicefrac}
\usepackage{parskip} % removes paragraph indentation, and adjusts paragraph skip, as well as list items
%\usepackage{setspace} % adjust text spacing and indents
\usepackage{siunitx} % decimal alignment
\usepackage{subfigure} % divided figures
%\usepackage{tabu} % extra table options
\usepackage{textcomp} % symbols
\usepackage{threeparttablex} % better footnotes with longtable
\usepackage{titling} % title placement
\usepackage{ulem} % strikethrough text
%\usepackage{url} % superceded by hyperref
\usepackage{verbatim} % verbatim environment
\usepackage{xcolor} % colors and color boxes
\usepackage{xspace} % commands that don't eat up white space
\usepackage{hyperref} % links and page setup; should always come last

\hypersetup{
	bookmarks=true,
	colorlinks=true,
	citecolor=blue,
	linkcolor=blue,
	urlcolor=blue,
	pdfstartview={XYZ null null 1.0} % default open view is 100%
}

\DisableLigatures[f]{encoding = *, family = * } % disable ff, fi, fl ligatures, without f option, it also disables -- = endash
\renewcommand{\arraystretch}{1.1} % extra vertical space in tables

\begin{document}

\pagestyle{empty} % don't number pages

% custom title
\begin{center}
{\LARGE Classic Riddler}

\vspace{0.15in}

{\Large 14 August 2020}
\end{center}


\section*{Riddle:}

The Riddler Manufacturing Company makes all sorts of mathematical tools: compasses, protractors, slide rules---you name it!

Recently, there was an issue with the production of foot-long rulers.
It seems that each ruler was accidentally sliced at three random points along the ruler, resulting in four pieces.
Looking on the bright side, that means there are now four times as many rulers---they just happen to have different lengths.

On average, how long are the pieces that contain the 6-inch mark?

\section*{Solution:}

There are two ways to make the cuts around the halfway mark.
Either all three cuts are on the same half, or there are two on one half and one on the other.
Considering the case that the cuts are on the same half, there are actually two ways to make these cuts; either on the bottom half or top half.
These are symmetric, though, and they produce the same average cut sizes with the same probabilities.
Similarly, the case of two and one cuts on either side produce the same average cut sizes with the same probabilities (though distinct from the same-side cuts).

For the case of three cuts on the same half, the total probability is \nicefrac{1}{4}.
The cuts are distributed uniformly across the half length, so the expected distance from the halfway mark to the closest cut is \nicefrac{1}{8}.
(Either the lowest cut is at \nicefrac{5}{8} or the highest cut is at \nicefrac{3}{8}, on average.)
The average length of the section in question is therefore \nicefrac{5}{8}.

For the other case, the total probability is \nicefrac{3}{4}.
Here, the position of two different cuts must be considered.
With one cut on the bottom half, its average position is \nicefrac{1}{4}.
The average position of the lower cut in the top half is \nicefrac{2}{3}.
Reversed, these positions are \nicefrac{1}{3} and \nicefrac{3}{4}.
The average length is then \nicefrac{5}{12}.

Putting these two together yields an average length of \nicefrac{15}{32} (as a percentage), or
\fcolorbox{red}{white}{\bf 5.625 inches}\,.


\end{document}