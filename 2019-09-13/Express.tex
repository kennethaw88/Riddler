\documentclass{article}

\usepackage{amsmath} % math stuff
\usepackage{amssymb} % math stuff
\usepackage{array} % equations and stuff
\usepackage{bm} % bold math
%\usepackage{caption} % suppressed table numbering; incompatible with revtex, and longtable, I think
\usepackage{comment} % comment environment
%\usepackage{enumitem} % customization of enumeration, itemize, and description
\usepackage[T1]{fontenc} % font encoding for special characters, must also use scalable font package
\usepackage[margin=0.8in]{geometry} % paper sizes and margins (but be careful not to mess up pre-defined pages)
\usepackage{graphicx} % for graphics
%\usepackage{helvet} % default font is the helvetica postscript font
\usepackage{layouts} % print units like widths
\usepackage{lipsum} % lorem ipsum filler text
\usepackage{lmodern} % scalable font?
\usepackage{longtable} % multi-page tables
\usepackage{makecell} % specify line-breaks in table cells
\usepackage{mathrsfs} % math script font
\usepackage{mhchem} % easier chemical formula
\usepackage{microtype} % allows disabling of ligatures
%\usepackage{newcent} % new century schoolbook font
\usepackage{nicefrac}
\usepackage{numprint} % print and format (large) numbers
\usepackage{parskip} % removes paragraph indentation, and adjusts paragraph skip, as well as list items
\usepackage{pdfpages} % add pdf files as pages
%\usepackage{setspace} % adjust text spacing and indents
\usepackage{siunitx} % decimal alignment
\usepackage{subfigure} % divided figures
%\usepackage{tabu} % extra table options
\usepackage{textcomp} % symbols
\usepackage{threeparttablex} % better footnotes with longtable
\usepackage{titling} % title placement
\usepackage{ulem} % strikethrough text
%\usepackage{url} % superceded by hyperref
\usepackage{verbatim} % verbatim environment
\usepackage{xcolor} % colors and color boxes
\usepackage{xspace} % commands that don't eat up white space
\usepackage{hyperref} % links and page setup; should always come last

\hypersetup{
 bookmarks=true,
 colorlinks=true,
 citecolor=blue,
 linkcolor=blue,
 urlcolor=blue,
 pdfstartview={XYZ null null 1.0} % default open view is 100%
}

\DisableLigatures[f,t]{encoding = T1} % disable ff, fi, fl, tt ligatures, without f option, it also disables -- = endash
\renewcommand{\arraystretch}{2.0} % extra vertical space in tables

\begin{document}

\pagestyle{empty} % don't number pages

% custom title
\begin{center}
{\LARGE Express Riddler}

\vspace{0.15in}

{\Large 13 September 2019}
\end{center}


\section*{Riddle:}

At long last, Dakota Jones is close to finding the Lost Arc, a geometric antiquity buried deep in the sands of Egypt.
Along the way, she discovered what she described as a ``highly symmetric crystal'' that's needed to precisely locate the Arc.
Dakota measured the crystal using her laser scanner and relayed the results to you.
But nefarious agents have gotten wind of her plans, and Dakota and the crystal are nowhere to be found.

Locating the Arc is now up to you.
To do that, you must recreate the crystal using the data from Dakota's laser scanner.
The scanner takes a 3D object, and records 2D cross-sectional slices along the third dimension.
Here's the looping animation file the scanner produced for the crystal: \href{https://fivethirtyeight.com/wp-content/uploads/2019/09/jones_538.gif}{Link}

What sort of three-dimensional shape is the crystal?
No pressure---Dakota Jones, nay, the entire world, is counting on you to locate the Lost Arc and ensure its place in a museum!


\section*{Solution:}

The shape is a
\fcolorbox{red}{white}{\bf cube}\,.
The cross-sectional planes are normal to a diagonal which connects any two opposite vertices of the cube.
The shape is not unique, though, and the cube could be stretched or compressed along that diagonal.
But the riddle hints that it is highly symmetric, and a cube itself is more symmetric than the stretched or compressed shapes.


\end{document}