\documentclass{article}

\usepackage{amsmath} % math stuff
\usepackage{amssymb} % math stuff
\usepackage{array} % equations and stuff
\usepackage{bm} % bold math
%\usepackage{caption} % suppressed table numbering; incompatible with revtex, and longtable, I think
\usepackage{comment} % comment environment
%\usepackage{enumitem} % customization of enumeration, itemize, and description
\usepackage[T1]{fontenc} % font encoding for special characters, must also use scalable font package
\usepackage[margin=0.8in]{geometry} % paper sizes and margins (but be careful not to mess up pre-defined pages)
\usepackage{graphicx} % for graphics
%\usepackage{helvet} % default font is the helvetica postscript font
\usepackage{layouts} % print units like widths
\usepackage{lipsum} % lorem ipsum filler text
\usepackage{lmodern} % scalable font?
\usepackage{longtable} % multi-page tables
\usepackage{makecell} % specify line-breaks in table cells
\usepackage{mathrsfs} % math script font
\usepackage{mhchem} % easier chemical formula
\usepackage{microtype} % allows disabling of ligatures
%\usepackage{newcent} % new century schoolbook font
\usepackage{nicefrac}
\usepackage{numprint} % print and format (large) numbers
\usepackage{parskip} % removes paragraph indentation, and adjusts paragraph skip, as well as list items
\usepackage{pdfpages} % add pdf files as pages
%\usepackage{setspace} % adjust text spacing and indents
\usepackage{siunitx} % decimal alignment
\usepackage{subfigure} % divided figures
%\usepackage{tabu} % extra table options
\usepackage{textcomp} % symbols
\usepackage{threeparttablex} % better footnotes with longtable
\usepackage{titling} % title placement
\usepackage{ulem} % strikethrough text
%\usepackage{url} % superceded by hyperref
\usepackage{verbatim} % verbatim environment
\usepackage{xcolor} % colors and color boxes
\usepackage{xspace} % commands that don't eat up white space
\usepackage{hyperref} % links and page setup; should always come last

\hypersetup{
 bookmarks=true,
 colorlinks=true,
 citecolor=blue,
 linkcolor=blue,
 urlcolor=blue,
 pdfstartview={XYZ null null 1.0} % default open view is 100%
}

\DisableLigatures[f,t]{encoding = T1} % disable ff, fi, fl, tt ligatures, without f option, it also disables -- = endash
\renewcommand{\arraystretch}{2.0} % extra vertical space in tables

\begin{document}

\pagestyle{empty} % don't number pages

% custom title
\begin{center}
{\LARGE Express Riddler}

\vspace{0.15in}

{\Large 27 September 2019}
\end{center}


\section*{Riddle:}

Riddler Nation's Jibriel Taha, an avid baseball fan, saw the following tweet from the Milwaukee Brewers' beat writer Adam McCalvy:

\begin{quote}
The Brewers are...

\quad5-5 last 10g \\
10-10 last 20g \\
15-15 last 30g \\
20-20 last 40g \\
25-25 last 50g \\
30-30 last 60g

— Adam McCalvy (@AdamMcCalvy) \href{https://twitter.com/AdamMcCalvy/status/1170170706983866368}{September 6, 2019}
\end{quote}

Inspired by the Brewers' apparent mediocrity (they've since gone on a roll to clinch a playoff spot) Jibriel asks the following:

If a baseball team is truly .500, meaning it has a 50 percent chance of winning each game, what's the probability that it has won two of its last four games and four of its last eight games?

\section*{Solution:}

With eight games, there are $2^{8}=256$ possible outcomes, and each occurs with equal probability.
The best way to calculate the solution is to consider the first four games and last four games separately.
For each set of four games, there are $\binom{4}{2}=6$ ways to win exactly two games, in any order.
Since the two sets of four games are independent, there are $6\times6=36$ ways to win two games in each set.
So the solution is \nicefrac{36}{256}, or
\fcolorbox{red}{white}{\bf 14.0625\%}\,.




\end{document}