\documentclass{article}

\usepackage{amsmath} % math stuff
\usepackage{amssymb} % math stuff
\usepackage{array} % equations and stuff
\usepackage{bm} % bold math
%\usepackage{booktabs} % extra table rule options
%\usepackage{caption} % suppressed table numbering; incompatible with revtex, and longtable, I think
\usepackage{comment} % comment environment
%\usepackage{enumitem} % customization of enumeration, itemize, and description
\usepackage[T1]{fontenc} % font encoding for special characters, must also use scalable font package
\usepackage[margin=0.8in]{geometry} % paper sizes and margins (but be careful not to mess up pre-defined pages)
\usepackage{graphicx} % for graphics
%\usepackage{helvet} % default font is the helvetica postscript font
\usepackage{layouts} % print units like widths
\usepackage{lipsum} % lorem ipsum filler text
\usepackage{lmodern} % scalable font?
\usepackage{longtable} % multi-page tables
\usepackage{makecell} % specify line-breaks in table cells
\usepackage{mathrsfs} % math script font
\usepackage{mhchem} % easier chemical formula
\usepackage{microtype} % allows disabling of ligatures
%\usepackage{newcent} % new century schoolbook font
\usepackage{nicefrac}
\usepackage{numprint} % print and format (large) numbers
\usepackage{parskip} % removes paragraph indentation, and adjusts paragraph skip, as well as list items
\usepackage{pdfpages} % add pdf files as pages
%\usepackage{setspace} % adjust text spacing and indents
\usepackage{siunitx} % decimal alignment
\usepackage{subfigure} % divided figures
%\usepackage{tabu} % extra table options
\usepackage{textcomp} % symbols
\usepackage{threeparttablex} % better footnotes with longtable
\usepackage{titling} % title placement
\usepackage{ulem} % strikethrough text
%\usepackage{url} % superceded by hyperref
\usepackage{verbatim} % verbatim environment
\usepackage{xcolor} % colors and color boxes
\usepackage{xspace} % commands that don't eat up white space
\usepackage{hyperref} % links and page setup; should always come last

\hypersetup{
 bookmarks=true,
 colorlinks=true,
 citecolor=blue,
 linkcolor=blue,
 urlcolor=blue,
 pdfstartview={XYZ null null 1.0} % default open view is 100%
}

\DisableLigatures[f,t]{encoding = T1} % disable ff, fi, fl, tt ligatures, without f option, it also disables -- = endash
\renewcommand{\arraystretch}{2.0} % extra vertical space in tables

% define centered, left- and right-aligned columns with specified widths
\newcommand{\PreserveBackslash}[1]{\let\temp=\\#1\let\\=\temp}
\newcolumntype{C}[1]{>{\PreserveBackslash\centering}p{#1}}
\newcolumntype{L}[1]{>{\PreserveBackslash\raggedright}p{#1}}
\newcolumntype{R}[1]{>{\PreserveBackslash\raggedleft}p{#1}}

\begin{document}

\pagestyle{empty} % don't number pages

% custom title
\begin{center}
{\LARGE Express Riddler}

\vspace{0.15in}

{\Large 26 February 2021}
\end{center}


\section*{Riddle:}

This is the fourth and final week of CrossProduct™ puzzles---for now.
This time, there are \textit{four} four-digit numbers---each belongs in a row of the table below, with one digit per cell.
The products of the four digits of each number are shown in the rightmost column.
Meanwhile, the products of the digits in the thousands, hundreds, tens and ones places, respectively, are shown in the bottom row.

\begin{center}
\begin{tabular}{|C{0.7in}|C{0.7in}|C{0.7in}|C{0.7in}!{\vrule width 2pt}R{0.7in}|}
\hline
 & & & & 1,458 \\
\hline
 & & & & 128 \\
\hline
 & & & & 2,688 \\
\hline
 & & & & 360 \\
\hline
 & & & & 125 \\
\noalign{\hrule height 2pt}
960 & 384 & 630 & 270 & \\
\hline
\end{tabular}
\end{center}

Can you find all four four-digit numbers and complete the table?

\section*{Solution:}

The riddle is essentially asking for 16 digits to be placed in a $4\times4$ table.
Generally, the first step is to decompose each number into either four (not-necessarily-prime) single-digit factors.
For this riddle, though, I only decomposed the rows into factors.
I list all possible sets of factors for each row below:

\begin{center}
\setlength{\tabcolsep}{3pt}
\begin{tabular}{lllll}
1458 & $=2\cdot9\cdot9\cdot9$ & $=3\cdot6\cdot9\cdot9$ \\
128  & $=1\cdot2\cdot8\cdot8$ & $=1\cdot4\cdot4\cdot8$ & $=2\cdot2\cdot4\cdot8$ & $=2\cdot4\cdot4\cdot4$ \\
2688 & $=6\cdot7\cdot8\cdot8$ \\
125  & $=1\cdot5\cdot5\cdot5$
\end{tabular}
\end{center}

My first step is that 384 is not divisible by 5, so the bottom row must have a 1 in the second column.
The rest of the row is 5s.
Therefore the bottom row is 5,1,5,5.

The only products which have a factor of 7 are 2,688 and 630, so the third row, third column must be 7.
Since the 630 already has factors of $5\cdot7=35$, the remaining product is 18.
The only factor in common with 128 is 2, so the second row of the third column must be 2, leaving 9 for the top row.

The 270 has a factor of 5, so its remaining product is 54.
The 2,688 only has remaining factors of $6\cdot8\cdot8$, and the only factor in common with 54 is 6, which goes in the last column of the third row.
This leaves 8s for the first and second column.
The third row is therefore 8,8,7,6.

The 960 has factors of 5 and 8, leaving it with a remaining product of 24.
The 384 has factors of 1 and 8, leaving it with a remaining product of 48.
The 270 has factors of 5 and 6, leaving it with a remaining product of 9.
With a 9 already placed in the top row, there is only one remaining product with a factor of 9, so the factorization of $2\cdot9\cdot9\cdot9$ is ruled out.
The 9 must go in the last column, leaving factors of 3 and 6.
The 6 must go in the second column, leaving 3 in the first column.
The first row is therefore 3,6,9,9.

The only remaining factors for the columns are 8, 8, and 1.
Therefore the second row is 8,8,2,1.

The final solution is

\begin{center}
\begin{tabular}{|C{0.7in}|C{0.7in}|C{0.7in}|C{0.7in}|}
\hline
3 & 6 & 9 & 9 \\
\hline
8 & 8 & 2 & 1 \\
\hline
8 & 8 & 7 & 6 \\
\hline
5 & 1 & 5 & 5 \\
\hline
\end{tabular}
\end{center}


\end{document}