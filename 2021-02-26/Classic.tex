\documentclass{article}

\usepackage{amsmath} % math stuff
\usepackage{amssymb} % math stuff
\usepackage{array} % equations and stuff
\usepackage{bm} % bold math
%\usepackage{booktabs} % extra table rule options
%\usepackage{caption} % suppressed table numbering; incompatible with revtex, and longtable, I think
\usepackage{comment} % comment environment
%\usepackage{enumitem} % customization of enumeration, itemize, and description
\usepackage[T1]{fontenc} % font encoding for special characters, must also use scalable font package
\usepackage[margin=0.8in]{geometry} % paper sizes and margins (but be careful not to mess up pre-defined pages)
\usepackage{graphicx} % for graphics
%\usepackage{helvet} % default font is the helvetica postscript font
\usepackage{layouts} % print units like widths
\usepackage{lipsum} % lorem ipsum filler text
\usepackage{lmodern} % scalable font?
\usepackage{longtable} % multi-page tables
\usepackage{makecell} % specify line-breaks in table cells
\usepackage{mathrsfs} % math script font
\usepackage{mhchem} % easier chemical formula
\usepackage{microtype} % allows disabling of ligatures
%\usepackage{newcent} % new century schoolbook font
\usepackage{nicefrac}
\usepackage{numprint} % print and format (large) numbers
\usepackage{parskip} % removes paragraph indentation, and adjusts paragraph skip, as well as list items
\usepackage{pdfpages} % add pdf files as pages
%\usepackage{setspace} % adjust text spacing and indents
\usepackage{siunitx} % decimal alignment
\usepackage{subfigure} % divided figures
%\usepackage{tabu} % extra table options
\usepackage{textcomp} % symbols
\usepackage{threeparttablex} % better footnotes with longtable
\usepackage{titling} % title placement
\usepackage{ulem} % strikethrough text
%\usepackage{url} % superceded by hyperref
\usepackage{verbatim} % verbatim environment
\usepackage{xcolor} % colors and color boxes
\usepackage{xspace} % commands that don't eat up white space
\usepackage{hyperref} % links and page setup; should always come last

\hypersetup{
 bookmarks=true,
 colorlinks=true,
 citecolor=blue,
 linkcolor=blue,
 urlcolor=blue,
 pdfstartview={XYZ null null 1.0} % default open view is 100%
}

\DisableLigatures[f,t]{encoding = T1} % disable ff, fi, fl, tt ligatures, without f option, it also disables -- = endash
\renewcommand{\arraystretch}{1.0} % extra vertical space in tables

% define centered, left- and right-aligned columns with specified widths
\newcommand{\PreserveBackslash}[1]{\let\temp=\\#1\let\\=\temp}
\newcolumntype{C}[1]{>{\PreserveBackslash\centering}p{#1}}
\newcolumntype{L}[1]{>{\PreserveBackslash\raggedright}p{#1}}
\newcolumntype{R}[1]{>{\PreserveBackslash\raggedleft}p{#1}}

\begin{document}

\pagestyle{empty} % don't number pages

% custom title
\begin{center}
{\LARGE Classic Riddler}

\vspace{0.15in}

{\Large 26 February 2021}
\end{center}


\section*{Riddle:}

You have 10 blocks with which to build four steps against a wall.
The first step is one block high, the second is two blocks high, the third is three blocks high and the fourth is four blocks high.

However, the ground ever-so-slightly slopes down toward the wall, and both the floor and the blocks are a little bit slippery.
As a result, whenever you place a block at ground level, it slides toward the wall until it hits the wall or another block.
And when you place a block atop another block, it will similarly slide toward the wall until it hits the wall or another block.

Suppose the four blocks in the bottom row are labeled A, the three blocks in the second row are labeled B, the two blocks in the next row are labeled C and the topmost block is labeled D.
One way to build the steps would be to place the blocks in the following order, one row at a time: A-A-A-A-B-B-B-C-C-D.
You could alternatively place the blocks one column at a time: A-B-C-D-A-B-C-A-B-A.
But you could \textit{not} place them in the order A-B-B-A-A-A-B-C-C-D because that would mean at one point you have more blocks in the second row, B, than in the bottom row, A---a physical impossibility!

How many distinct ways are there to build these four steps using the 10 blocks?

\textit{Extra credit}: Suppose you have precisely enough blocks to build a staircase with $N$ stairs.
How many distinct ways are there to build this staircase?

\section*{Solution:}

I first note that the number of blocks required to make a simple staircase with $N$ steps is the $N$th triangular number, $T_{N}$:

\[
T_{N}=\frac{N(N+1)}{2}
\]

This is because the blocks form a (right) triangle, with two sides of length $N$ in the horizontal and vertical direction.

I decided to solve this problem by first counting the number of ways to stack the blocks for one, two, and three steps.
For one step, there is trivially just a single block, and a single way to place it, since there is only one action total.
For two steps, there are three blocks, and two ways to stack: A-A-B and A-B-A.

For three steps, I had to write out and count all the possibilities:

\begin{center}
\begin{tabular}{cccc}
A-A-A-B-B-C & A-A-A-B-C-B & A-A-B-A-B-C & A-A-B-A-C-B \\
A-A-B-B-A-C & A-A-B-B-C-A & A-A-B-C-A-B & A-A-B-C-B-A \\
A-B-A-A-B-C & A-B-A-A-C-B & A-B-A-B-A-C & A-B-A-B-C-A \\
A-B-A-C-A-B & A-B-A-C-B-A & A-B-C-A-A-B & A-B-C-A-B-A
\end{tabular}
\end{center}

This is a total of 16.

For 4 steps, I did not take the time to write out all the possibilities.
Instead, I first counted out the possibilities which reduce to the three-step solution.
With $T_{4}=10$ blocks, if the first four blocks are placed as either A-A-A-A or A-B-C-D, the remaining six blocks are essentially a three-step staircase.
So there are $2\times16=32$ ways to stack the blocks.
For the other ways to stack, I only considered how the first four were placed, for a simple comparison with the previous two cases.

Stacking the first four blocks in the order A-A-A-B produces the same (or mirrored) arrangement as A-A-B-A, A-B-A-A, A-A-B-C, A-B-A-C, and A-B-C-A.
So I only listed the remaining six blocks in the A-A-A-B case, since the others would have the same number of ways to fill in the rest of the blocks.
I list those here:

\begin{center}
\begin{tabular}{cccccc}
A-B-B-C-C-D & A-A-B-C-D-C & A-B-C-B-C-D & A-B-C-B-D-C & A-B-C-C-B-D & A-B-C-C-D-B \\
A-B-C-D-B-C & A-B-C-D-C-B & A-C-B-B-C-D & A-C-B-B-D-C & A-C-B-C-B-D & A-C-B-C-D-B \\
A-C-B-D-B-C & A-C-B-D-C-B & A-C-D-B-B-C & A-C-D-B-C-B & B-A-B-C-C-D & B-A-B-C-D-C \\
B-A-C-B-C-D & B-A-C-B-D-C & B-A-C-C-B-D & B-A-C-C-D-B & B-A-C-D-B-C & B-A-C-D-C-B \\
B-B-A-C-C-D & B-B-A-C-D-C & B-B-C-A-C-D & B-B-C-A-D-C & B-B-C-C-A-D & B-B-C-C-D-A \\
B-B-C-D-A-C & B-B-C-D-C-A & B-C-A-B-C-D & B-C-A-B-D-C & B-C-A-C-B-D & B-C-A-C-D-B \\
B-C-A-D-B-C & B-C-A-D-C-B & B-C-B-A-C-D & B-C-B-A-D-C & B-C-B-C-A-D & B-C-B-C-D-A \\
B-C-B-D-A-C & B-C-B-D-C-A & B-C-C-A-B-D & B-C-C-A-D-B & B-C-C-B-A-D & B-C-C-B-D-A \\
B-C-C-D-A-B & B-C-C-D-B-A & B-C-D-A-B-C & B-C-D-A-C-B & B-C-D-B-A-C & B-C-D-B-C-A \\
B-C-D-C-A-B & B-C-D-C-B-A & C-A-B-B-C-D & C-A-B-B-D-C & C-A-B-C-B-D & C-A-B-C-D-B \\
C-A-B-D-B-C & C-A-B-D-C-B & C-A-D-B-B-C & C-A-D-B-C-B & C-B-A-B-C-D & C-B-A-B-D-C \\
C-B-A-C-B-D & C-B-A-C-D-B & C-B-A-D-B-C & C-B-A-D-C-A & C-B-B-A-C-D & C-B-B-A-D-C \\
C-B-B-C-A-D & C-B-B-C-D-A & C-B-B-D-A-C & C-B-B-D-C-A & C-B-C-A-B-D & C-B-C-A-D-B \\
C-B-C-B-A-D & C-B-C-B-D-A & C-B-C-D-A-B & C-B-C-D-B-A & C-B-D-A-B-C & C-B-D-A-C-B \\
C-B-D-B-A-C & C-B-D-B-C-A & C-B-D-C-A-B & C-B-D-C-B-A & C-D-A-B-B-C & C-D-A-B-C-B \\
C-D-B-A-B-C & C-D-B-A-C-B & C-D-B-B-A-C & C-D-B-B-C-A & C-D-B-C-A-B & C-D-B-C-B-A
\end{tabular}
\end{center}

That is 96 possibilities, adding $6\times96=576$ to the solution.
There are only two other ways to stack the first four blocks: A-A-B-B and A-B-A-B.
Starting with either of these, the remaining six blocks can be arranged as follows:

\begin{center}
\begin{tabular}{ccccc}
A-A-B-C-C-D & A-A-B-C-D-C & A-A-C-B-C-D & A-A-C-B-D-C & A-A-C-C-B-D \\
A-A-C-C-D-B & A-A-C-D-B-C & A-A-C-D-C-B & A-B-A-C-C-D & A-B-A-C-D-C \\
A-B-C-A-C-D & A-B-C-A-D-C & A-B-C-C-A-D & A-B-C-C-D-A & A-B-C-D-A-C \\
A-B-C-D-C-A & A-C-A-B-C-D & A-C-A-B-D-C & A-C-A-C-B-D & A-C-A-C-D-B \\
A-C-A-D-B-C & A-C-A-D-C-B & A-C-B-A-C-D & A-C-B-A-D-C & A-C-B-C-A-D \\
A-C-B-C-D-A & A-C-B-D-A-C & A-C-B-D-C-A & A-C-C-A-B-D & A-C-C-A-D-B \\
A-C-C-B-A-D & A-C-C-B-D-A & A-C-C-D-A-B & A-C-C-D-B-A & A-C-D-A-B-C \\
A-C-D-A-C-B & A-C-D-B-A-C & A-C-D-B-C-A & A-C-D-C-A-B & A-C-D-C-B-A \\
C-A-A-B-C-D & C-A-A-B-D-C & C-A-A-C-B-D & C-A-A-C-D-B & C-A-A-D-B-C \\
C-A-A-D-C-B & C-A-B-A-C-D & C-A-B-A-D-C & C-A-B-C-A-D & C-A-B-C-D-A \\
C-A-B-D-A-C & C-A-B-D-C-A & C-A-C-A-B-D & C-A-C-A-D-B & C-A-C-B-A-D \\
C-A-C-B-D-A & C-A-C-D-A-B & C-A-C-D-B-A & C-A-D-A-B-C & C-A-D-A-C-B \\
C-A-D-B-A-C & C-A-D-B-C-A & C-A-D-C-A-B & C-A-D-C-B-A & C-C-A-A-B-D \\
C-C-A-A-D-B & C-C-A-B-A-D & C-C-A-B-D-A & C-C-A-D-A-B & C-C-A-D-B-A \\
C-C-D-A-A-B & C-C-D-A-B-A & C-D-A-A-B-C & C-D-A-A-C-B & C-D-A-B-A-C \\
C-D-A-B-C-A & C-D-A-C-A-B & C-D-A-C-B-A & C-D-C-A-A-B & C-D-C-A-B-A
\end{tabular}
\end{center}

That is 80 possibilties, adding $2\times80=160$ to the solution.
In total, I have counted
\fcolorbox{red}{white}{\bf 768} ways to stack the blocks.

So for one, two, three, and four blocks, the sequence seems to be 1, 2, 16, and 768.
Plugging this into the OEIS gives three known sequences.
The most promising seems to be A108400, which is simply listed as the product:

\[
a(n)=\prod_{k=0}^{n}k!\cdot2^{k}
\]

Compared to the current riddle, there is an offset of 1, so that my proposed solution for the number of ways to build $b(N)$ for $N\geq1$ blocks is

\[
b(N)=\prod_{k=0}^{N-1}k!\cdot2^{k}
\]





\end{document}