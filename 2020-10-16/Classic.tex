\documentclass{article}

\usepackage{amsmath} % math stuff
\usepackage{amssymb} % math stuff
\usepackage{array} % equations and stuff
\usepackage{bm} % bold math
%\usepackage{caption} % suppressed table numbering; incompatible with revtex, and longtable, I think
\usepackage{comment} % comment environment
%\usepackage{enumitem} % customization of enumeration, itemize, and description
\usepackage[T1]{fontenc} % font encoding for special characters, must also use scalable font package
\usepackage[margin=0.8in]{geometry} % paper sizes and margins (but be careful not to mess up pre-defined pages)
\usepackage{graphicx} % for graphics
%\usepackage{helvet} % default font is the helvetica postscript font
\usepackage{layouts} % print units like widths
\usepackage{lipsum} % lorem ipsum filler text
\usepackage{lmodern} % scalable font?
\usepackage{longtable} % multi-page tables
\usepackage{mathrsfs} % math script font
\usepackage{mhchem} % easier chemical formula
\usepackage{microtype} % allows disabling of ligatures
%\usepackage{newcent} % new century schoolbook font
\usepackage{nicefrac}
\usepackage{parskip} % removes paragraph indentation, and adjusts paragraph skip, as well as list items
%\usepackage{setspace} % adjust text spacing and indents
\usepackage{siunitx} % decimal alignment
\usepackage{subfigure} % divided figures
%\usepackage{tabu} % extra table options
\usepackage{textcomp} % symbols
\usepackage{threeparttablex} % better footnotes with longtable
\usepackage{titling} % title placement
\usepackage{ulem} % strikethrough text
%\usepackage{url} % superceded by hyperref
\usepackage{verbatim} % verbatim environment
\usepackage{xcolor} % colors and color boxes
\usepackage{xspace} % commands that don't eat up white space
\usepackage{hyperref} % links and page setup; should always come last

\hypersetup{
	bookmarks=true,
	colorlinks=true,
	citecolor=blue,
	linkcolor=blue,
	urlcolor=blue,
	pdfstartview={XYZ null null 1.0} % default open view is 100%
}

\DisableLigatures[f,t]{encoding = T1} % disable ff, fi, fl, tt ligatures, without f option, it also disables -- = endash
\renewcommand{\arraystretch}{1.1} % extra vertical space in tables

\begin{document}

\pagestyle{empty} % don't number pages

% custom title
\begin{center}
{\LARGE Classic Riddler}

\vspace{0.15in}

{\Large 16 October 2020}
\end{center}


\section*{Riddle:}

This week, we return to the brilliant and ageless game show, ``The Price is Right.''
In a modified version of the bidding round, you and two (not three) other contestants must guess the price of an item, one at a time.

Assume the true price of this item is a randomly selected value between 0 and 100.
(Note: The value is a real number and does not have to be an integer.)
Among the three contestants, the winner is whoever guesses the closest price \textit{without going over}.
For example, if the true price is 29 and I guess 30, while another contestant guesses 20, then they would be the winner even though my guess was technically closer.

In the event all three guesses exceed the actual price, the contestant who made the lowest (and therefore closest) guess is declared the winner.
I mean, \textit{someone} has to win, right?

If you are the first to guess, and all contestants play optimally (taking full advantage of the guesses of those who went before them), what are your chances of winning?

\section*{Solution:}

This solution can be built up from the two-contestant case.
In that case, the best solution is for the first contestant to guess 50 (i.e., \nicefrac{1}{2}*100).
If the second player guesses any amount above 50, even by some infinitessimal $\varepsilon$, then that player's chance of winning is less than 50\%, (specifically $\nicefrac{(50-\varepsilon)}{100}$).
The second player's best guess then is any number below 50, so that there is a win in both cases where the number is either between the two guesses or less than both guesses.
In this case, the second player's chance of winning is 50\%.
So the best that the last player can do is to have a chance of winning of \nicefrac{1}{2}.
If the first player guesses anything below 50 (call the guess $x$), the second player could seize the opportunity and guess \nicefrac{(50+x)}{2}, giving the second player an advantage; the chance of winning is now $\nicefrac{(150-x)}{200}>\nicefrac{1}{2}$.
If $x$ is above 50, the second player guesses anything below $x$, with a chance of winning of $x>\nicefrac{1}{2}$.
So the only strategy for the first player which gives the best chance of winning is 100*\nicefrac{1}{2}.
No strategy for the first player leads to a chance of winning over \nicefrac{1}{2}.

The case for any other number (say $N$) of players is similar.
The best any contestant can do, given ideal play from all previous players is to end with a chance of winning of \nicefrac{1}{2}.
The first player should guess \nicefrac{(N-1)}{N}*100.
No subsequent player will guess higher, because it would reduce that player's chance of winning.
The next player should treat the remaining space like the first player: try to get a \nicefrac{1}{(N-1)} chance of winning out of the \nicefrac{(N-1)}{N} that is left.
This player's guess should be \nicefrac{(N-2)}{N}.

In general, the $i$th player should always guess \nicefrac{(N-$i$)}{N}.
This leaves each player with the same \nicefrac{1}{N} chance of winning, so nobody can do better.

For the specific case of three contestants, the first player should guess \nicefrac{2}{3}, which leaves that player with a chance of winning of
\fcolorbox{red}{white}{\bf{\nicefrac{1}{3}}}\,.


\end{document}