\documentclass{article}

\usepackage{amsmath} % math stuff
\usepackage{amssymb} % math stuff
\usepackage{array} % equations and stuff
\usepackage{bm} % bold math
%\usepackage{booktabs} % extra table rule options
%\usepackage{caption} % suppressed table numbering; incompatible with revtex, and longtable, I think
\usepackage{comment} % comment environment
%\usepackage{enumitem} % customization of enumeration, itemize, and description
\usepackage[T1]{fontenc} % font encoding for special characters, must also use scalable font package
\usepackage[margin=0.8in]{geometry} % paper sizes and margins (but be careful not to mess up pre-defined pages)
\usepackage{graphicx} % for graphics
%\usepackage{helvet} % default font is the helvetica postscript font
\usepackage{layouts} % print units like widths
\usepackage{lipsum} % lorem ipsum filler text
\usepackage{lmodern} % scalable font?
\usepackage{longtable} % multi-page tables
\usepackage{makecell} % specify line-breaks in table cells
\usepackage{mathrsfs} % math script font
\usepackage{mhchem} % easier chemical formula
\usepackage{microtype} % allows disabling of ligatures
%\usepackage{newcent} % new century schoolbook font
\usepackage{nicefrac}
\usepackage{numprint} % print and format (large) numbers
\usepackage{parskip} % removes paragraph indentation, and adjusts paragraph skip, as well as list items
\usepackage{pdfpages} % add pdf files as pages
%\usepackage{setspace} % adjust text spacing and indents
\usepackage{siunitx} % decimal alignment
\usepackage{subfigure} % divided figures
%\usepackage{tabu} % extra table options
\usepackage{textcomp} % symbols
\usepackage{threeparttablex} % better footnotes with longtable
\usepackage{titling} % title placement
\usepackage{ulem} % strikethrough text
%\usepackage{url} % superceded by hyperref
\usepackage{verbatim} % verbatim environment
\usepackage{xcolor} % colors and color boxes
\usepackage{xspace} % commands that don't eat up white space
\usepackage{hyperref} % links and page setup; should always come last

\hypersetup{
 bookmarks=true,
 colorlinks=true,
 citecolor=blue,
 linkcolor=blue,
 urlcolor=blue,
 pdfstartview={XYZ null null 1.0} % default open view is 100%
}

\DisableLigatures[f,t]{encoding = T1} % disable ff, fi, fl, tt ligatures; without options, it also disables -- = endash
\renewcommand{\arraystretch}{1.0} % extra vertical (and horizontal?) space in tables

% define centered, left- and right-aligned columns with specified widths
\newcommand{\PreserveBackslash}[1]{\let\temp=\\#1\let\\=\temp}
\newcolumntype{C}[1]{>{\PreserveBackslash\centering}p{#1}}
\newcolumntype{L}[1]{>{\PreserveBackslash\raggedright}p{#1}}
\newcolumntype{R}[1]{>{\PreserveBackslash\raggedleft}p{#1}}

\begin{document}

\pagestyle{empty} % don't number pages

% custom title
\begin{center}
{\LARGE Classic Riddler}

\vspace{0.15in}

{\Large 19 March 2021}
\end{center}


\section*{Riddle:}

A few weeks ago, Scott Matlick reached out to me with observations about the relative likelihood that a positive integer with a given number of digits would be a perfect square.
And that got us both wondering.
For some perfect squares, when you remove the last digit, you get another perfect square.
For example, when you remove the last digit from 256 ($16^{2}$), you get 25 ($5^{2}$).

The first few squares for which this happens are 16, 49, 169, 256 and 361.
What are the \textit{next} three squares for which you can remove the last digit and get a different perfect square?
How many more can you find?
(Bonus points for not looking this up online or writing code to solve it for you!
There are interesting ways to do this by hand, I swear.)

\textit{Extra credit}: Did you look up the sequence and spoil the puzzle for yourself?
Good news, there’s more!
In the list above, 169 ($13^{2}$) is a little different from the other numbers.
Not only when you remove the last digit do you get a perfect square, 16 ($4^{2}$), but when you remove the last two digits, you again get a perfect square: 1 ($1^{2}$).
Can you find another square with \textit{both} of these properties?



\section*{Solution:}

To solve this, I just started listing out square numbers and trying to match them up.
The next three squares for which this occurs are 1444 ($38^{2}$), 3249 ($57^{2}$), and 18,496 ($136^{2}$).
After removing the last digit, these correspond to 144 ($12^{2}$), 324 ($18^{2}$), and 1849 ($43^{2}$).
So the solution is
\fcolorbox{red}{white}{\bf 1,444; 3,249; and 18,496}\,.


\end{document}