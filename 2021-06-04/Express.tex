\documentclass{article}

\usepackage{amsmath} % math stuff
\usepackage{amssymb} % math stuff
\usepackage{array} % equations and stuff
\usepackage{bm} % bold math
%\usepackage{booktabs} % extra table rule options
%\usepackage{caption} % suppressed table numbering; incompatible with revtex, and longtable, I think
\usepackage{comment} % comment environment
%\usepackage{enumitem} % customization of enumeration, itemize, and description
\usepackage[T1]{fontenc} % font encoding for special characters, must also use scalable font package
\usepackage[margin=0.8in]{geometry} % paper sizes and margins (but be careful not to mess up pre-defined pages)
\usepackage{graphicx} % for graphics
%\usepackage{helvet} % default font is the helvetica postscript font
\usepackage{layouts} % print units like widths
\usepackage{lipsum} % lorem ipsum filler text
\usepackage{lmodern} % scalable font?
\usepackage{longtable} % multi-page tables
\usepackage{makecell} % specify line-breaks in table cells
\usepackage{mathrsfs} % math script font
\usepackage{mhchem} % easier chemical formula
\usepackage{microtype} % allows disabling of ligatures
%\usepackage{newcent} % new century schoolbook font
\usepackage{nicefrac}
\usepackage{numprint} % print and format (large) numbers
\usepackage{parskip} % removes paragraph indentation, and adjusts paragraph skip, as well as list items
\usepackage{pdfpages} % add pdf files as pages
%\usepackage{setspace} % adjust text spacing and indents
\usepackage{siunitx} % decimal alignment
\usepackage{subfigure} % divided figures
%\usepackage{tabu} % extra table options
\usepackage{textcomp} % symbols
\usepackage{threeparttablex} % better footnotes with longtable
\usepackage{titling} % title placement
\usepackage{ulem} % strikethrough text
%\usepackage{url} % superceded by hyperref
\usepackage{verbatim} % verbatim environment
\usepackage{xcolor} % colors and color boxes
\usepackage{xspace} % commands that don't eat up white space
\usepackage{hyperref} % links and page setup; should always come last

\hypersetup{
 bookmarks=true,
 colorlinks=true,
 citecolor=blue,
 linkcolor=blue,
 urlcolor=blue,
 pdfstartview={XYZ null null 1.0} % default open view is 100%
}

\DisableLigatures[f,t]{encoding = T1} % disable ff, fi, fl, tt ligatures; without options, it also disables -- = endash
\renewcommand{\arraystretch}{1.1} % extra vertical (and horizontal?) space in tables

% define centered, left- and right-aligned columns with specified widths
\newcommand{\PreserveBackslash}[1]{\let\temp=\\#1\let\\=\temp}
\newcolumntype{C}[1]{>{\PreserveBackslash\centering}p{#1}}
\newcolumntype{L}[1]{>{\PreserveBackslash\raggedright}p{#1}}
\newcolumntype{R}[1]{>{\PreserveBackslash\raggedleft}p{#1}}

\begin{document}

\pagestyle{empty} % don't number pages

% custom title
\begin{center}
{\LARGE Express Riddler}

\vspace{0.15in}

{\Large 4 June 2021}
\end{center}


\section*{Riddle:}

Max the Mathemagician is calling for volunteers.
He has a magic wand of length 10 that can be broken anywhere along its length (fractional and decimal lengths are allowed).
After the volunteer chooses these breakpoints, Max will multiply the lengths of the resulting pieces.
For example, if they break the wand near its midpoint and nowhere else, the resulting product is $5\times5$, or 25.
If the product is the largest possible, they will win a free backstage pass to his next show.
(Amazing, right?)

You raise your hand to volunteer, and you and Max briefly make eye contact.
As he calls you up to the stage, you know you have this in the bag.
What is the maximum product you can achieve?

\textit{Extra credit}: Zax the Mathemagician (no relation to Max) has the same routine in his show, only the wand has a length of 100.
What is the maximum product now?


\section*{Solution:}

It is trivial to show that for two numbers with a given sum, the maximum product occurs when the two numbers are equal to each other, and equal to half of the sum.
That is, if A+B=C for some constant C, then, the product AB is maximized if A=B=\nicefrac{C}{2}.
This can be generalized to more numbers with a constant sum.
For any two numbers among a larger set to have a maximum product, they must be equal.
Applying this logic to the current problem, the overall product (for any number of wand pieces) will be maximized if all the pieces have equal length.
So if the wand is broken into $n$ pieces, then each piece will have length $\nicefrac{10}{n}$.
Multiplying each of the $n$ lengths will give the total product $(\nicefrac{10}{n})^{n}$.
I list the first few products below:

\vspace{0.1in}
\begin{center}
\begin{tabular}{cc}
$n$ & $\left(\frac{10}{n}\right)^{n}$ \\
\hline
1 & 10 \\
2 & 25 \\
3 & 37.037$\dots$ \\
4 & 39.0625 \\
5 & 32
\end{tabular}
\qquad\qquad
\begin{tabular}{cc}
$n$ & $\left(\frac{10}{n}\right)^{n}$ \\
\hline
6 & 21.433$\dots$ \\
7 & 12.142$\dots$ \\
8 & 5.960$\dots$ \\
9 & 2.581$\dots$ \\
10 & 1
\end{tabular}
\end{center}
\vspace{0.1in}

Within this limited set of values, the largest product is obtained for 4 pieces.
Beyond 4 it seems the product decreases, but it's not obvious whether it continues to decrease for numbers larger than 10.

The answer can be found by considering the continuous function $\nicefrac{k}{n}$.
This function has a maximum at $n=\nicefrac{k}{e}$, and decreases toward 0 for all larger values of $n$.
In the current case, $k=10$, so the maximum occurs at $n\approx3.68$.
So for any integer value of $n$ above 4, the product will be less than 39.0625, and this is indeed the maximum product.
So the solution is
\fcolorbox{red}{white}{\bf 39.0625}\,.

Using the same continuous function with $k=100$ gives the maximum at $n\approx36.8$, so both 36 and 37 must be considered as solutions.
Using $n=36$ gives a product of $9.40\times10^{15}$, and 37 gives $9.47\times10^{15}$.
So the solution to the extra credit is (approximately)
\fcolorbox{red}{white}{$\bm{9.47\times10^{15}}$}\, with 37 pieces.



\end{document}