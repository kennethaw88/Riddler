\documentclass{article}

\usepackage{amsmath} % math stuff
\usepackage{amssymb} % math stuff
\usepackage{array} % equations and stuff
\usepackage{bm} % bold math
%\usepackage{caption} % suppressed table numbering; incompatible with revtex, and longtable, I think
\usepackage{comment} % comment environment
%\usepackage{enumitem} % customization of enumeration, itemize, and description
\usepackage[T1]{fontenc} % font encoding for special characters, must also use scalable font package
\usepackage[margin=0.8in]{geometry} % paper sizes and margins (but be careful not to mess up pre-defined pages)
\usepackage{graphicx} % for graphics
%\usepackage{helvet} % default font is the helvetica postscript font
\usepackage{layouts} % print units like widths
\usepackage{lipsum} % lorem ipsum filler text
\usepackage{lmodern} % scalable font?
\usepackage{longtable} % multi-page tables
\usepackage{makecell} % specify line-breaks in table cells
\usepackage{mathrsfs} % math script font
\usepackage{mhchem} % easier chemical formula
\usepackage{microtype} % allows disabling of ligatures
%\usepackage{newcent} % new century schoolbook font
\usepackage{nicefrac}
\usepackage{numprint} % print and format (large) numbers
\usepackage{parskip} % removes paragraph indentation, and adjusts paragraph skip, as well as list items
\usepackage{pdfpages} % add pdf files as pages
%\usepackage{setspace} % adjust text spacing and indents
\usepackage{siunitx} % decimal alignment
\usepackage{subfigure} % divided figures
%\usepackage{tabu} % extra table options
\usepackage{textcomp} % symbols
\usepackage{threeparttablex} % better footnotes with longtable
\usepackage{titling} % title placement
\usepackage{ulem} % strikethrough text
%\usepackage{url} % superceded by hyperref
\usepackage{verbatim} % verbatim environment
\usepackage{xcolor} % colors and color boxes
\usepackage{xspace} % commands that don't eat up white space
\usepackage{hyperref} % links and page setup; should always come last

\hypersetup{
 bookmarks=true,
 colorlinks=true,
 citecolor=blue,
 linkcolor=blue,
 urlcolor=blue,
 pdfstartview={XYZ null null 1.0} % default open view is 100%
}

\DisableLigatures[f,t]{encoding = T1} % disable ff, fi, fl, tt ligatures, without f option, it also disables -- = endash
\renewcommand{\arraystretch}{2.0} % extra vertical space in tables

\npthousandsep{,\allowbreak} % allow breaks at any thousands separator for numprint values

\begin{document}

\pagestyle{empty} % don't number pages

% custom title
\begin{center}
{\LARGE Express Riddler}

\vspace{0.15in}

{\Large 11 October 2019}
\end{center}


\section*{Riddle:}

An auditorium with 200 seats, numbered from 1 to 200, is filled to capacity.
A speaker, who happens to be a mathematician, steps up to the podium overlooking the audience and pauses for a moment.
``You know,'' she says, ``I'm thinking of a rather large whole number.
Every seat number in this auditorium evenly divides my number, except for two of them---and those two seats happen to be next to each other.''

As you'd expect, adjacent seats in the auditorium have consecutive numbers.
Which two numbers was the speaker referring to?


\section*{Solution:}

The two adjacent numbers which are excluded must be powers of prime numbers.
Otherwise, their factorization will still show up in the final product.
For example, if 35 ($=5\times7$) is excluded, the factors of 5 and 7 would still show up because 5 and 7 themselves are not excluded.
Further, the excluded numbers must be the highest possible powers that are less than 200.
For example, if the number 27 ($=3^{3}$) is excluded, all the 3s would still show up because 81 ($=3^{4}$) is not excluded.

All of the highest prime powers below 200 are listed below.

\begin{center}
\begin{tabular*}{5.5in}{@{\extracolsep{\fill}}cccccccccc}
128 ($2^{7}$)  & 169 ($13^{2}$) & 31 & 53 & 73 & 101 & 127 & 151 & 179 & 199 \\
81 ($3^{4}$)   & 17             & 37 & 59 & 79 & 103 & 131 & 157 & 181 & \\
125 ($5^{3}$)  & 19             & 41 & 61 & 83 & 107 & 137 & 163 & 191 & \\
49 ($7^{2}$)   & 23             & 43 & 67 & 89 & 109 & 139 & 167 & 193 & \\
121 ($11^{2}$) & 29             & 47 & 71 & 97 & 113 & 149 & 173 & 197 & \\
\end{tabular*}
\end{center}

The only two adjacent numbers on this list, and the solution, are
\fcolorbox{red}{white}{\bf 127 and 128}\,.

There isn't a particular answer to the ``rather large'' whole number about which the speaker is thinking.
However, it must have all of the other factors above at least, in addition to the remaining factors of 2 that shows up in 64.
Plugging those into Wolfram Alpha gives me \numprint{1327927515090260884407345538562367745796828278681721394601759928808007945120777126248000}\,, a number with 88 digits.
Any multiple of this number could be the speaker's number, as long as that multiple itself isn't divisible by 2 or 127.




\end{document}