\documentclass{article}

\usepackage{amsmath} % math stuff
\usepackage{amssymb} % math stuff
\usepackage{array} % equations and stuff
\usepackage{bm} % bold math
%\usepackage{booktabs} % extra table rule options
%\usepackage{caption} % suppressed table numbering; incompatible with revtex, and longtable, I think
\usepackage{comment} % comment environment
%\usepackage{enumitem} % customization of enumeration, itemize, and description
\usepackage[T1]{fontenc} % font encoding for special characters, must also use scalable font package
\usepackage[margin=0.8in]{geometry} % paper sizes and margins (but be careful not to mess up pre-defined pages)
\usepackage{graphicx} % for graphics
%\usepackage{helvet} % default font is the helvetica postscript font
\usepackage[utf8]{inputenc} % special characters in tex input
\usepackage{layouts} % print units like widths
\usepackage{lipsum} % lorem ipsum filler text
\usepackage{lmodern} % scalable font?
\usepackage{longtable} % multi-page tables
\usepackage{makecell} % specify line-breaks in table cells
\usepackage{mathrsfs} % math script font
\usepackage{mhchem} % easier chemical formula
\usepackage{microtype} % allows disabling of ligatures
\usepackage{multicol} % multicolumns
%\usepackage{newcent} % new century schoolbook font
\usepackage{nicefrac}
\usepackage{numprint} % print and format (large) numbers
\usepackage{parskip} % removes paragraph indentation, and adjusts paragraph skip, as well as list items
\usepackage{pdfpages} % add pdf files as pages
%\usepackage{setspace} % adjust text spacing and indents
\usepackage{siunitx} % decimal alignment
\usepackage{subfigure} % divided figures
%\usepackage{tabu} % extra table options
\usepackage{textcomp} % symbols
\usepackage{threeparttablex} % better footnotes with longtable
\usepackage{titling} % title placement
\usepackage{ulem} % strikethrough text
%\usepackage{url} % superceded by hyperref
\usepackage{verbatim} % verbatim environment
\usepackage{xcolor} % colors and color boxes
\usepackage{xspace} % commands that don't eat up white space
\usepackage{hyperref} % links and page setup; should always come last

\hypersetup{
 bookmarks=true,
 colorlinks=true,
 citecolor=blue,
 linkcolor=blue,
 urlcolor=blue,
 pdfstartview={XYZ null null 1.0} % default open view is 100%
}

\DisableLigatures[f,t]{encoding = T1} % disable ff, fi, fl, tt ligatures; without options, it also disables -- = endash
\renewcommand{\arraystretch}{1.0} % extra vertical (and horizontal?) space in tables

% define centered, left- and right-aligned columns with specified widths
\newcommand{\PreserveBackslash}[1]{\let\temp=\\#1\let\\=\temp}
\newcolumntype{C}[1]{>{\PreserveBackslash\centering}p{#1}}
\newcolumntype{L}[1]{>{\PreserveBackslash\raggedright}p{#1}}
\newcolumntype{R}[1]{>{\PreserveBackslash\raggedleft}p{#1}}

\begin{document}

\pagestyle{empty} % don't number pages

% custom title
\begin{center}
{\LARGE Express Riddler}

\vspace{0.15in}

{\Large 29 October 2021}
\end{center}


\section*{Riddle:}

I have a spherical pumpkin.
I carefully calculate its volume in cubic inches, as well as its surface area in square inches.
Next, I got up to have a piece of Halloween candy (which, naturally, was a Reese's Peanut Butter Cup).

But when I came back to my calculations, I saw that my units---the square inches and the cubic inches---had mysteriously disappeared from my calculations.
But it didn't matter, because both numerical values were the same!

What is the radius of my spherical pumpkin?

\textit{Extra credit:} Let's dispense with 3D thinking.
Instead, suppose I have an $n$-hyperspherical pumpkin.
Once again, I calculate its volume (with units $\mathrm{in}^{n}$) and surface area (with units $\mathrm{in}^{n-1}$).
Miraculously, the numerical values are once again the same!
What is the radius of my $n$-hyperspherical pumpkin?


\section*{Solution:}

Solving this riddle simply requires setting the volume and surface area of a sphere with $r$ equal to each other:

\[
\frac{4}{3}\pi r^{3} = 4\pi r^{2}
\]

This has a solution of $r=3$ in whatever units are used.
In the current case, the solution is \fcolorbox{red}{white}{\textbf{3~in}}\,.

For the case of an $n$-hypersphere, the strategy is the same: set the $n$-dimensional ``volume'' and $n-1$-dimensional ``surface area'' equal to each other.
I don't know the general formula for the volume of an $n$-dimensional sphere, but I do know that the surface area is the derivative of the volume.
The volume is some constant term $k$ multiplied by the $r^{n}$ term.
Setting this equal to its derivative yields:

\begin{align*}
kr^{n} &= \frac{d}{dr}kr^{n} \\
       &= nkr^{n-1}
\end{align*}

This has the relatively simple solution of \fcolorbox{red}{white}{$\bm{r=n}$}\,.


\end{document}