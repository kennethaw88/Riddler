\documentclass{article}

\usepackage{amsmath} % math stuff
\usepackage{amssymb} % math stuff
\usepackage{array} % equations and stuff
\usepackage{bm} % bold math
%\usepackage{caption} % suppressed table numbering; incompatible with revtex, and longtable, I think
\usepackage{comment} % comment environment
%\usepackage{enumitem} % customization of enumeration, itemize, and description
\usepackage[T1]{fontenc} % font encoding for special characters, must also use scalable font package
\usepackage[margin=0.8in]{geometry} % paper sizes and margins (but be careful not to mess up pre-defined pages)
\usepackage{graphicx} % for graphics
%\usepackage{helvet} % default font is the helvetica postscript font
\usepackage{layouts} % print units like widths
\usepackage{lipsum} % lorem ipsum filler text
\usepackage{lmodern} % scalable font?
\usepackage{longtable} % multi-page tables
\usepackage{makecell} % specify line-breaks in table cells
\usepackage{mathrsfs} % math script font
\usepackage{mhchem} % easier chemical formula
\usepackage{microtype} % allows disabling of ligatures
%\usepackage{newcent} % new century schoolbook font
\usepackage{nicefrac}
\usepackage{parskip} % removes paragraph indentation, and adjusts paragraph skip, as well as list items
\usepackage{pdfpages} % add pdf files as pages
%\usepackage{setspace} % adjust text spacing and indents
\usepackage{siunitx} % decimal alignment
\usepackage{subfigure} % divided figures
%\usepackage{tabu} % extra table options
\usepackage{textcomp} % symbols
\usepackage{threeparttablex} % better footnotes with longtable
\usepackage{titling} % title placement
\usepackage{ulem} % strikethrough text
%\usepackage{url} % superceded by hyperref
\usepackage{verbatim} % verbatim environment
\usepackage{xcolor} % colors and color boxes
\usepackage{xspace} % commands that don't eat up white space
\usepackage{hyperref} % links and page setup; should always come last

\hypersetup{
 bookmarks=true,
 colorlinks=true,
 citecolor=blue,
 linkcolor=blue,
 urlcolor=blue,
 pdfstartview={XYZ null null 1.0} % default open view is 100%
}

\DisableLigatures[f,t]{encoding = T1} % disable ff, fi, fl, tt ligatures, without f option, it also disables -- = endash
\renewcommand{\arraystretch}{1.1} % extra vertical space in tables

\begin{document}

\pagestyle{empty} % don't number pages

% custom title
\begin{center}
{\LARGE Express Riddler}

\vspace{0.15in}

{\Large 6 November 2020}
\end{center}


\section*{Riddle:}

Last weekend's New York City Marathon was canceled.
But runners from Des Linden---one of the top American marathoners---to FiveThirtyEight's own Santul Nerkar---my number one editor---still went out there and braved the course.
Santul finished in a time of 3:41:43 (3 hours, 41 minutes, 43 seconds), which averaged to 8 minutes, 27 seconds per mile.

Suppose, while training, Santul completed two 20-mile runs on a treadmill.
For the first run, he set the treadmill to a constant speed so that he ran every mile in 9 minutes.

The second run was a little different.
He started at a 10 minute-per-mile pace and accelerated continuously until he was running at an 8-minute-per mile pace at the end.
Moreover, Santul's minutes-per-mile pace (i.e., \textit{not} his speed) changed linearly over time.
So a quarter of the way through the duration (in time, not distance) of the run, his pace was 9 minutes, 30 seconds per mile, halfway through it was 9 minutes per mile, etc.

Which training run was faster (i.e., took less time) overall?
And what were Santul's times for the two runs?


\section*{Solution:}

I will label the constant-speed time $t_{1}$ and the accelerating-speed time $t_{2}$.
The first time is easy to calculate:

\begin{align*}
t_{1} &= 9~\rm{min/mile}*20~\rm{mile} \\
      &= 180~\rm{min}
\end{align*}

The second time is more complicated.
The pace is given as an inverse rate $\nicefrac{1}{v}(t)$.
Based on the parameters given, the inverse rate can be written as

\[
\frac{1}{v}(t)=10-\frac{2}{t_{2}}t
\]

so that the pace is 10~min/mile at $t=0$ and 8~min/mile at $t=t_{2}$.
The total distance of 20~miles is the integral of the velocity with respect to time, the pace must be inverted and integrated.

\begin{align*}
20 &= \int v(t)\,dt \\
   &= \int_{0}^{t_{2}}\frac{1}{10-\frac{2}{t_{2}}t}\,dt \\
   &= -\frac{t_{2}}{2}\left.\ln\left(10-\frac{2}{t_{2}}\right)\right|_{0}^{t_{2}} \\
   &= -\frac{t_{2}}{2}\left[\ln(8)-\ln(10)\right] \\
   &= \frac{t_{2}}{2}\ln\left(\frac{5}{4}\right)
\end{align*}
\begin{align*}
t_{2} &= \frac{40}{\ln\left(\frac{5}{4}\right)} \\
      &= 179.2568\dots~\rm{min}
\end{align*}

So the solution is that
\fcolorbox{red}{white}{\bf the second run is faster by about 45~seconds}\,.



\end{document}