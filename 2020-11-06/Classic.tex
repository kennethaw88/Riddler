\documentclass{article}

\usepackage{amsmath} % math stuff
\usepackage{amssymb} % math stuff
\usepackage{array} % equations and stuff
\usepackage{bm} % bold math
%\usepackage{caption} % suppressed table numbering; incompatible with revtex, and longtable, I think
\usepackage{comment} % comment environment
%\usepackage{enumitem} % customization of enumeration, itemize, and description
\usepackage[T1]{fontenc} % font encoding for special characters, must also use scalable font package
\usepackage[margin=0.8in]{geometry} % paper sizes and margins (but be careful not to mess up pre-defined pages)
\usepackage{graphicx} % for graphics
%\usepackage{helvet} % default font is the helvetica postscript font
\usepackage{layouts} % print units like widths
\usepackage{lipsum} % lorem ipsum filler text
\usepackage{lmodern} % scalable font?
\usepackage{longtable} % multi-page tables
\usepackage{makecell} % specify line-breaks in table cells
\usepackage{mathrsfs} % math script font
\usepackage{mhchem} % easier chemical formula
\usepackage{microtype} % allows disabling of ligatures
%\usepackage{newcent} % new century schoolbook font
\usepackage{nicefrac}
\usepackage{parskip} % removes paragraph indentation, and adjusts paragraph skip, as well as list items
\usepackage{pdfpages} % add pdf files as pages
%\usepackage{setspace} % adjust text spacing and indents
\usepackage{siunitx} % decimal alignment
\usepackage{subfigure} % divided figures
%\usepackage{tabu} % extra table options
\usepackage{textcomp} % symbols
\usepackage{threeparttablex} % better footnotes with longtable
\usepackage{titling} % title placement
\usepackage{ulem} % strikethrough text
%\usepackage{url} % superceded by hyperref
\usepackage{verbatim} % verbatim environment
\usepackage{xcolor} % colors and color boxes
\usepackage{xspace} % commands that don't eat up white space
\usepackage{hyperref} % links and page setup; should always come last

\hypersetup{
 bookmarks=true,
 colorlinks=true,
 citecolor=blue,
 linkcolor=blue,
 urlcolor=blue,
 pdfstartview={XYZ null null 1.0} % default open view is 100%
}

\DisableLigatures[f,t]{encoding = T1} % disable ff, fi, fl, tt ligatures, without f option, it also disables -- = endash
\renewcommand{\arraystretch}{1.1} % extra vertical space in tables

\begin{document}

\pagestyle{empty} % don't number pages

% custom title
\begin{center}
{\LARGE Classic Riddler}

\vspace{0.15in}

{\Large 6 November 2020}
\end{center}


\section*{Riddle:}

Mathematician John von Neumann is credited with figuring out how to take a biased coin (whose probability of coming up heads is $p$, not necessarily equal to 0.5) and ``simulate'' a fair coin.
Simply flip the coin twice.
If it comes up heads both times or tails both times, then flip it twice again.
Eventually, you'll get two different flips---either a heads and then a tails, or a tails and then a heads, with each of these two cases equally likely.
Once you get two different flips, you can call the second of those flips the outcome of your ``simulation.''

For any value of $p$ between zero and one, this procedure will always return heads half the time and tails half the time.
This is pretty remarkable!
But there's a downside to von Neumann's approach---you don't know how long (i.e., how many flips) the simulation will last.

Suppose I want to simulate a fair coin in at most \textit{three} flips.
For which values of $p$ is this possible?

\textit{Extra credit}: Suppose I want to simulate a fair coin in at most $N$ flips. For how many values of $p$ is this possible?


\section*{Solution:}

We can imagine each coin flip as dividing a unit cube along one axis, with each slice separating the cube into regions with volumes $p$ and $1-p$.
After three flips there are 8 sections of the cube, with 4 distinct volumes; each section is a rectangular prism.

Letting section A represent the sequence HHH, it has volume $p^{3}$.
Sections B, C, and D represent HHT, HTH, THH (in any order) and have volume $p^{2}(1-p)$.
Sections E, F, and G represent HTT, THT, TTH (in any order) and have volume $p(1-p)^{2}$.
Finally, section H represents TTT and has volume $(1-p)^{3}$.

Solving the riddle involves finding some combination of these sections whose total volume is \nicefrac{1}{2}, meaning that after three flips, there is some specific combination of flip sequences whose total probability is \nicefrac{1}{2}.
Each combination creates a (potentially) cubic equation which gives a solution to the riddle if there is a root for $0<p<1$.
Specifically, the equation is 

\[
\sum V_{i}=\frac{1}{2}
\]

where the sum is over the volumes $V_{i}$ of the specific combination of sections.
In principle, there are up to $256=2^{8}$ possible combinations to check.
However, the empty combination and the combination with every section cannot be solutions.
Further, only half of the combinations need to be checked anyway, since the reverse combination will necessarily have the same area of \nicefrac{1}{2}.
Additionally, many distinct combinations represent the same volume.
I list all of the combinations below and the numerical solutions below.
Because of the symmetry of the problem, I have only searched for solutions involving section A.
Most solutions have closed forms when solved in Wolfram Alpha, but not all of them; I have not bothered to manually find all closed form solutions, so I only list the numerical solutions.
Combinations without valid solutions are marked with ``--''.

\vspace{0.2in}
\begin{center}
\begin{tabular*}{0.9\textwidth}{@{\extracolsep{\fill}}cccccc}
Combination & $p$ & Combination & $p$ & Combination & $p$ \\
\cline{1-2} \cline{3-4} \cline{5-6}
A & 0.7937 & \makecell{AB\\AC\\AD} & 0.7071 & \makecell{AE\\AF\\AG} & 0.7718 \\
\cline{1-2} \cline{3-4} \cline{5-6}
AH & \makecell{0.7887\\0.2113} & \makecell{ABC\\ABD\\ACD} & 0.5970 & \makecell{ABE\\ABF\\ABG\\ACE\\ACF\\ACG\\ADE\\ADF\\ADG} & 0.6478 \\
\cline{1-2} \cline{3-4} \cline{5-6}
\makecell{ABH\\ACH\\ADH} & \makecell{0.6846\\0.2373} & \makecell{AEF\\AEG\\AFG} & 0.7347 & \makecell{AEH\\AFH\\AGH} & \makecell{0.7627\\0.3154} \\
\cline{1-2} \cline{3-4} \cline{5-6}
ABCD & 0.5 & \makecell{ABCE\\ABCF\\ABCG\\ABDE\\ABDF\\ABDG\\ACDE\\ACDF\\ACDG} & 0.5 & \makecell{ABCH\\ABDH\\ACDH} & \makecell{0.5\\0.2929} \\
\cline{1-2} \cline{3-4} \cline{5-6}
\makecell{ABEF\\ABEG\\ABFG\\ACEF\\ACEG\\ACFG\\ADEF\\ADEG\\ADFG} & 0.5 & \makecell{ABEH\\ABFH\\ABGH\\ACEH\\ACFH\\ACGH\\ADEH\\ADFH\\ADGH} & 0.5 & AEFG & 0.5 \\
\cline{1-2} \cline{3-4} \cline{5-6}
\makecell{AEFH\\AEGH\\AFGH} & \makecell{0.5\\0.7071} & \makecell{ABCDE\\ABCDF\\ABCDG} & 0.4030 & ABCDH & -- \\
\cline{1-2} \cline{3-4} \cline{5-6}
\makecell{ABCEF\\ABCEG\\ABCFG\\ABDEF\\ABDEG\\ABDFG\\ACDEF\\ACDEG\\ACDFG} & 0.3522 & \makecell{ABCEH\\ABCFH\\ABCGH\\ABDEH\\ABDFH\\ABDGH\\ACDEH\\ACDFH\\ACDGH} & -- & \makecell{ABEFG\\ACEFG\\ADEFG} & 0.2653 \\
\cline{1-2} \cline{3-4} \cline{5-6}
\makecell{ABEFH\\ABEGH\\ABFGH\\ACEFH\\ACEGH\\ACFGH\\ADEFH\\ADEGH\\ADFGH} & -- & AEFGH & -- & \makecell{ABCDEF\\ABCDEG\\ABCDFG} & 0.2929 \\
\cline{1-2} \cline{3-4} \cline{5-6}
\end{tabular*}
\end{center}

\vspace{0.2in}
\begin{center}
\begin{tabular*}{0.9\textwidth}{@{\extracolsep{\fill}}cccccc}
Combination & $p$ & Combination & $p$ & Combination & $p$ \\
\cline{1-2} \cline{3-4} \cline{5-6}
\makecell{ABCDEH\\ABCDFH\\ABCDGH} & -- & \makecell{ABCEFG\\ABDEFG\\ACDEFG} & 0.2282 &
\makecell{ABCEFH\\ABCEGH\\ABCFGH\\ABDEFH\\ABDEGH\\ABDFGH\\ACDEFH\\ACDEGH\\ACDFGH} & -- \\
\cline{1-2} \cline{3-4} \cline{5-6}
\makecell{ABEFGH\\ACEFGH\\ADEFGH} & -- & ABCDEFG & 0.2063 & \makecell{ABCDEFH\\ABCDEGH\\ABCDFGH} & -- \\
\cline{1-2} \cline{3-4} \cline{5-6}
\makecell{ABCEFGH\\ABDEFGH\\ACDEFGH} & --
\end{tabular*}
\end{center}

In total, there are
\fcolorbox{red}{white}{\bf 19} possibilities for $p$.
In an easier-to-read list, they are:

\vspace{0.2in}
\begin{center}
\begin{tabular*}{0.55\textwidth}{@{\extracolsep{\fill}}ccccc}
0.2063 & 0.2113 & 0.2282 & 0.2373 & 0.2653 \\
0.2929 & 0.3154 & 0.3522 & 0.4030 & 0.5 \\
0.5970 & 0.6478 & 0.6846 & 0.7071 & 0.7347 \\
0.7627 & 0.7718 & 0.7887 & 0.7937
\end{tabular*}
\end{center}




\end{document}