\documentclass{article}

\usepackage{amsmath} % math stuff
\usepackage{amssymb} % math stuff
\usepackage{array} % equations and stuff
\usepackage{bm} % bold math
%\usepackage{booktabs} % extra table rule options
%\usepackage{caption} % suppressed table numbering; incompatible with revtex, and longtable, I think
\usepackage{comment} % comment environment
%\usepackage{enumitem} % customization of enumeration, itemize, and description
\usepackage[T1]{fontenc} % font encoding for special characters, must also use scalable font package
\usepackage[margin=0.8in]{geometry} % paper sizes and margins (but be careful not to mess up pre-defined pages)
\usepackage{graphicx} % for graphics
%\usepackage{helvet} % default font is the helvetica postscript font
\usepackage[utf8]{inputenc} % special characters in tex input
\usepackage{layouts} % print units like widths
\usepackage{lipsum} % lorem ipsum filler text
\usepackage{lmodern} % scalable font?
\usepackage{longtable} % multi-page tables
\usepackage{makecell} % specify line-breaks in table cells
\usepackage{mathrsfs} % math script font
\usepackage{mhchem} % easier chemical formula
\usepackage{microtype} % allows disabling of ligatures
\usepackage{multicol} % multicolumns
%\usepackage{newcent} % new century schoolbook font
\usepackage{nicefrac}
\usepackage{numprint} % print and format (large) numbers
\usepackage{parskip} % removes paragraph indentation, and adjusts paragraph skip, as well as list items
\usepackage{pdfpages} % add pdf files as pages
%\usepackage{setspace} % adjust text spacing and indents
\usepackage{siunitx} % decimal alignment
\usepackage{subfigure} % divided figures
%\usepackage{tabu} % extra table options
\usepackage{textcomp} % symbols
\usepackage{threeparttablex} % better footnotes with longtable
\usepackage{titling} % title placement
\usepackage{ulem} % strikethrough text
%\usepackage{url} % superceded by hyperref
\usepackage{verbatim} % verbatim environment
\usepackage{xcolor} % colors and color boxes
\usepackage{xspace} % commands that don't eat up white space
\usepackage{hyperref} % links and page setup; should always come last

\hypersetup{
 bookmarks=true,
 colorlinks=true,
 citecolor=blue,
 linkcolor=blue,
 urlcolor=blue,
 pdfstartview={XYZ null null 1.0} % default open view is 100%
}

\DisableLigatures[f,t]{encoding = T1} % disable ff, fi, fl, tt ligatures; without options, it also disables -- = endash
\renewcommand{\arraystretch}{1.0} % extra vertical (and horizontal?) space in tables

% define centered, left- and right-aligned columns with specified widths
\newcommand{\PreserveBackslash}[1]{\let\temp=\\#1\let\\=\temp}
\newcolumntype{C}[1]{>{\PreserveBackslash\centering}p{#1}}
\newcolumntype{L}[1]{>{\PreserveBackslash\raggedright}p{#1}}
\newcolumntype{R}[1]{>{\PreserveBackslash\raggedleft}p{#1}}

\begin{document}

\pagestyle{empty} % don't number pages

% custom title
\begin{center}
{\LARGE Express Riddler}

\vspace{0.15in}

{\Large 12 November 2021}
\end{center}


\section*{Riddle:}

I have three dice (d4, d6, d8) on my desk that I fiddle with while working, much to the chagrin of my co-workers.
For the uninitiated, the d4 is a tetrahedron that is equally likely to land on any of its four faces (numbered 1 through 4), the d6 is a cube that is equally likely to land on any of its six faces (numbered 1 through 6), and the d8 is an octahedron that is equally likely to land on any of its eight faces (numbered 1 through 8).

I like to play a game in which I roll all three dice in ``numerical'' order: d4, then d6 and then d8.
I win this game when the three rolls form a strictly increasing sequence (such as 2-4-7, but \textit{not} 2-4-4).
What is my probability of winning?

\textit{Extra credit}: Instead of three dice, I now have six dice: d4, d6, d8, d10, d12 and d20.
If I roll all six dice in ``numerical'' order, what is the probability I’ll get a strictly increasing sequence?


\section*{Solution:}

There are $4\times6\times8=192$ possible ways to roll the three dice in order.
That is in the range where it is possible to list out every winning combination.
I've listed out those combinations here:

\vspace{0.1in}
\begin{center}
\begin{tabular*}{3.5in}{@{\extracolsep{\fill}}cccccc}
1-2-3 & 1-3-6 & 1-5-7 & 2-3-8 & 2-6-7 & 3-5-8 \\
1-2-4 & 1-3-7 & 1-5-8 & 2-4-5 & 2-6-8 & 3-6-7 \\
1-2-5 & 1-3-8 & 1-6-7 & 2-4-6 & 3-4-5 & 3-6-8 \\
1-2-6 & 1-4-5 & 1-6-8 & 2-4-7 & 3-4-6 & 4-5-6 \\
1-2-7 & 1-4-6 & 2-3-4 & 2-4-8 & 3-4-7 & 4-5-7 \\
1-2-8 & 1-4-7 & 2-3-5 & 2-5-6 & 3-4-8 & 4-5-8 \\
1-3-4 & 1-4-8 & 2-3-6 & 2-5-7 & 3-5-6 & 4-6-7 \\
1-3-5 & 1-5-6 & 2-3-7 & 2-5-8 & 3-5-7 & 4-6-8 \\
\end{tabular*}
\end{center}
\vspace{0.1in}

That is a total of 48 combinations, which means the solution is \nicefrac{48}{192}, or
\fcolorbox{red}{white}{$\bm{{\nicefrac{1}{4}}}$}\,.

Another way of looking at the first solution is that there are six combinations that begin 1-2, five that begin 1-3, four that begin 1-4, three that begin 1-5, and two that begin 1-6.
Similarly, there are five combinations that begin 2-3, four that begin 2-4, three that begin 2-5, and two that begin 2-6.
Then there are four that begin 3-4, three that begin 3-5, and two that begin 3-6.
Finally there are three that begin 4-5 and two that begin 4-6.
These can be written as a sum:

\[
\sum_{j=3}^{6}\ \sum_{i=2}^{j}\ i
\]

which indeed gives the solution of 48.

I starting writing out the winning combinations for the six-dice problem.
It yields similar (though much larger) patterns of winning combinations.
For example, there are 15 combinations that begin 1-2-3-4-5, 14 that begin 1-2-3-4-6, etc., down to eight solutions that begin 1-2-3-4-12.
Extending this pattern ultimately gives the sum:

\[
\sum_{m=12}^{15}\ \sum_{l=11}^{m}\ \sum_{k=10}^{l}\ \sum_{j=9}^{k}\ \sum_{i=8}^{j}\ i
\]

This has a total sum of 5,434.
I don't have an explanation for each of the indices in the sum, but it seems to work.
Because there are $4\times6\times8\times10\times12\times20=460{,}800$ possible rolls, this means the solution is \nicefrac{5,434}{460,800}, or
\fcolorbox{red}{white}{$\bm{{\nicefrac{2,717}{230,400}\approx0.01179}}$}\,.



\end{document}