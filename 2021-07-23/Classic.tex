\documentclass{article}

\usepackage{amsmath} % math stuff
\usepackage{amssymb} % math stuff
\usepackage{array} % equations and stuff
\usepackage{bm} % bold math
%\usepackage{booktabs} % extra table rule options
%\usepackage{caption} % suppressed table numbering; incompatible with revtex, and longtable, I think
\usepackage{comment} % comment environment
%\usepackage{enumitem} % customization of enumeration, itemize, and description
\usepackage[T1]{fontenc} % font encoding for special characters, must also use scalable font package
\usepackage[margin=0.8in]{geometry} % paper sizes and margins (but be careful not to mess up pre-defined pages)
\usepackage{graphicx} % for graphics
%\usepackage{helvet} % default font is the helvetica postscript font
\usepackage[utf8]{inputenc} % special characters in tex input
\usepackage{layouts} % print units like widths
\usepackage{lipsum} % lorem ipsum filler text
\usepackage{lmodern} % scalable font?
\usepackage{longtable} % multi-page tables
\usepackage{makecell} % specify line-breaks in table cells
\usepackage{mathrsfs} % math script font
\usepackage{mhchem} % easier chemical formula
\usepackage{microtype} % allows disabling of ligatures
\usepackage{multicol} % multicolumns
%\usepackage{newcent} % new century schoolbook font
\usepackage{nicefrac}
\usepackage{numprint} % print and format (large) numbers
\usepackage{parskip} % removes paragraph indentation, and adjusts paragraph skip, as well as list items
\usepackage{pdfpages} % add pdf files as pages
%\usepackage{setspace} % adjust text spacing and indents
\usepackage{siunitx} % decimal alignment
\usepackage{subfigure} % divided figures
%\usepackage{tabu} % extra table options
\usepackage{textcomp} % symbols
\usepackage{threeparttablex} % better footnotes with longtable
\usepackage{titling} % title placement
\usepackage{ulem} % strikethrough text
%\usepackage{url} % superceded by hyperref
\usepackage{verbatim} % verbatim environment
\usepackage{xcolor} % colors and color boxes
\usepackage{xspace} % commands that don't eat up white space
\usepackage{hyperref} % links and page setup; should always come last

\hypersetup{
 bookmarks=true,
 colorlinks=true,
 citecolor=blue,
 linkcolor=blue,
 urlcolor=blue,
 pdfstartview={XYZ null null 1.0} % default open view is 100%
}

\DisableLigatures[f,t]{encoding = T1} % disable ff, fi, fl, tt ligatures; without options, it also disables -- = endash
\renewcommand{\arraystretch}{1.0} % extra vertical (and horizontal?) space in tables

% define centered, left- and right-aligned columns with specified widths
\newcommand{\PreserveBackslash}[1]{\let\temp=\\#1\let\\=\temp}
\newcolumntype{C}[1]{>{\PreserveBackslash\centering}p{#1}}
\newcolumntype{L}[1]{>{\PreserveBackslash\raggedright}p{#1}}
\newcolumntype{R}[1]{>{\PreserveBackslash\raggedleft}p{#1}}

\begin{document}

\pagestyle{empty} % don't number pages

% custom title
\begin{center}
{\LARGE Classic Riddler}

\vspace{0.15in}

{\Large 23 July 2021}
\end{center}


\section*{Riddle:}

Today marks the beginning of the Summer Olympics!
One of the brand-new events this year is sport climbing, in which competitors try their hands (and feet) at lead climbing, speed climbing and bouldering.

Suppose the event's organizers accidentally forgot to place all the climbing holds on and had to do it last-minute for their 10-meter wall (the regulation height for the purposes of this riddle).
Climbers won't have any trouble moving horizontally along the wall.
However, climbers can't move between holds that are more than 1 meter apart vertically.

In a rush, the organizers place climbing holds randomly until there are no vertical gaps between climbing holds (including the bottom and top of the wall).
Once they are done placing the holds, how many will there be on average (not including the bottom and top of the wall)?

\textit{Extra credit}: Now suppose climbers find it just as difficult to move horizontally as vertically, meaning they can't move between any two holds that are more than 1 meter apart in any direction.
Suppose also that the climbing wall is a 10-by-10 meter square.
If the organizers again place the holds randomly, how many have to be placed on average until it's possible to climb the wall?



\section*{Solution:}

I wrote a script to tackle this problem, which I saved as \texttt{rock\_climb.C}\,.
The script generates random numbers one at a time, between 0 and 1.
At every point, it sorts the full list of numbers and calculates all of the differences between adjacent numbers.
The list of numbers also includes 0 and 1 in order to calculate the top and bottom gaps.
It doesn't bother calculating the differences if there are fewer than 11 numbers.
Once the largest difference between adjacent numbers is below 0.1, it counts the numbers in the list.
After one million trials, the average is calculated.
This could be counted for more trials, but based on one run of the program, my solution is
\fcolorbox{red}{white}{\bf 43.05 climbing holds}\,.




\end{document}