\documentclass{article}

\usepackage{amsmath} % math stuff
\usepackage{amssymb} % math stuff
\usepackage{array} % equations and stuff
\usepackage{bm} % bold math
%\usepackage{caption} % suppressed table numbering; incompatible with revtex, and longtable, I think
\usepackage{comment} % comment environment
%\usepackage{enumitem} % customization of enumeration, itemize, and description
\usepackage[T1]{fontenc} % font encoding for special characters, must also use scalable font package
\usepackage[margin=0.8in]{geometry} % paper sizes and margins (but be careful not to mess up pre-defined pages)
\usepackage{graphicx} % for graphics
%\usepackage{helvet} % default font is the helvetica postscript font
\usepackage{lipsum} % lorem ipsum filler text
\usepackage{lmodern} % scalable font?
\usepackage{longtable} % multi-page tables
\usepackage{mathrsfs} % math script font
\usepackage{mhchem} % easier chemical formula
\usepackage{microtype} % allows disabling of ligatures
%\usepackage{newcent} % new century schoolbook font
\usepackage{nicefrac}
\usepackage{parskip} % removes paragraph indentation, and adjusts paragraph skip, as well as list items
%\usepackage{setspace} % adjust text spacing and indents
\usepackage{siunitx} % decimal alignment
\usepackage{subfigure} % divided figures
%\usepackage{tabu} % extra table options
\usepackage{textcomp} % symbols
\usepackage{threeparttablex} % better footnotes with longtable
\usepackage{titling} % title placement
\usepackage{ulem} % strikethrough text
%\usepackage{url} % superceded by hyperref
\usepackage{verbatim} % verbatim environment
\usepackage{xcolor} % colors and color boxes
\usepackage{xspace} % commands that don't eat up white space
\usepackage{hyperref} % links and page setup; should always come last

\hypersetup{
	bookmarks=true,
	colorlinks=true,
	citecolor=blue,
	linkcolor=blue,
	urlcolor=blue,
	pdfstartview={XYZ null null 1.0} % default open view is 100%
}

\DisableLigatures[f]{encoding = *, family = * } % disable ff, fi, fl ligatures, without f option, it also disables -- = endash
\renewcommand{\arraystretch}{1.1} % extra vertical space in tables

\begin{document}

\pagestyle{empty} % don't number pages

% custom title
\begin{center}
{\LARGE Express Riddler}

\vspace{0.15in}

{\Large 25 September 2020}
\end{center}


\section*{Riddle:}

The U.S. Open concluded last weekend, with physics major Bryson DeChambeau emerging victorious.
Seeing his favorite golfer win his first major got Dan thinking about the precision needed to be a professional at the sport.

A typical hole is about 400 yards long, while the cup measures a mere 4.25 inches in diameter.
Suppose that, with every swing, you hit the ball X percent closer to the center of the hole.
For example, if $X$ were 75 percent, then with every swing the ball would be four times closer to the hole than it was previously.

For a 400-yard hole, assuming there are no hazards (water, sand or otherwise) in the way, what is the minimum value of $X$ so that you'll shoot par, meaning you'll hit the ball into the cup in exactly four strokes?

\section*{Solution:}

This is a pretty simple problem with exponents.
Besides the explicit assumptions in the puzzle, I am also assuming that the golf ball is point-like (or, alternately, is perfectly spherical, and touches the perfectly flat ground at a single point).
If the starting and ending distances from the center of the hole are $d_{tee}$ and $d_{edge}$, the desired number of strokes is $n$, and the $X$ value is used as a fraction (instead of a percent), then $X$ is the solution to the equation

\[
(1-X)^{n}=\frac{d_{edge}}{d_{tee}}
\]

In this case, $n=4$, $d_{edge}=2.125\ \rm{in}$, $d_{tee}=400\ \rm{yd}$, giving a ratio $\nicefrac{d_{edge}}{d_{tee}}\approx0.0001476$.
Solving for $X$ gives the (approximate) solution of 0.8898, or
\fcolorbox{red}{white}{\bf 88.98\%}

\end{document}