\documentclass{article}

\usepackage{amsmath} % math stuff
\usepackage{amssymb} % math stuff
\usepackage{array} % equations and stuff
\usepackage{bm} % bold math
%\usepackage{booktabs} % extra table rule options
%\usepackage{caption} % suppressed table numbering; incompatible with revtex, and longtable, I think
\usepackage{comment} % comment environment
%\usepackage{enumitem} % customization of enumeration, itemize, and description
\usepackage[T1]{fontenc} % font encoding for special characters, must also use scalable font package
\usepackage[margin=0.8in]{geometry} % paper sizes and margins (but be careful not to mess up pre-defined pages)
\usepackage{graphicx} % for graphics
%\usepackage{helvet} % default font is the helvetica postscript font
\usepackage[utf8]{inputenc} % special characters in tex input
\usepackage{layouts} % print units like widths
\usepackage{lipsum} % lorem ipsum filler text
\usepackage{lmodern} % scalable font?
\usepackage{longtable} % multi-page tables
\usepackage{makecell} % specify line-breaks in table cells
\usepackage{mathrsfs} % math script font
\usepackage{mhchem} % easier chemical formula
\usepackage{microtype} % allows disabling of ligatures
\usepackage{multicol} % multicolumns
%\usepackage{newcent} % new century schoolbook font
\usepackage{nicefrac}
\usepackage{numprint} % print and format (large) numbers
\usepackage{parskip} % removes paragraph indentation, and adjusts paragraph skip, as well as list items
\usepackage{pdfpages} % add pdf files as pages
%\usepackage{setspace} % adjust text spacing and indents
\usepackage{siunitx} % decimal alignment
\usepackage{subfigure} % divided figures
%\usepackage{tabu} % extra table options
\usepackage{textcomp} % symbols
\usepackage{threeparttablex} % better footnotes with longtable
\usepackage{titling} % title placement
\usepackage{ulem} % strikethrough text
%\usepackage{url} % superceded by hyperref
\usepackage{verbatim} % verbatim environment
\usepackage{xcolor} % colors and color boxes
\usepackage{xspace} % commands that don't eat up white space
\usepackage{hyperref} % links and page setup; should always come last

\hypersetup{
 bookmarks=true,
 colorlinks=true,
 citecolor=blue,
 linkcolor=blue,
 urlcolor=blue,
 pdfstartview={XYZ null null 1.0} % default open view is 100%
}

\DisableLigatures[f,t]{encoding = T1} % disable ff, fi, fl, tt ligatures; without options, it also disables -- = endash
\renewcommand{\arraystretch}{1.0} % extra vertical (and horizontal?) space in tables

% define centered, left- and right-aligned columns with specified widths
\newcommand{\PreserveBackslash}[1]{\let\temp=\\#1\let\\=\temp}
\newcolumntype{C}[1]{>{\PreserveBackslash\centering}p{#1}}
\newcolumntype{L}[1]{>{\PreserveBackslash\raggedright}p{#1}}
\newcolumntype{R}[1]{>{\PreserveBackslash\raggedleft}p{#1}}

\begin{document}

\pagestyle{empty} % don't number pages

% custom title
\begin{center}
{\LARGE Classic Riddler}

\vspace{0.15in}

{\Large 24 September 2021}
\end{center}


\section*{Riddle:}

You may recall a previous riddle about the new Olympic event, sport climbing.
This week, puzzle submitter Andy Esposito takes us back to this exciting event:

The finals of the sport climbing competition has eight climbers, each of whom compete in three different events: speed climbing, bouldering and lead climbing.
Based on their time and performance, each of the eight climbers is given a ranking (first through eighth, with no ties allowed) in each event, as well as a corresponding score (1 through 8, respectively).

The three scores each climber earns are then multiplied together to give a final score.
For example, a climber who placed second in speed climbing, fifth in bouldering and sixth in lead climbing would receive a score of $2\times5\times6$, or 60 points.
The gold medalist is whoever achieves the lowest final score among the eight finalists.

What is the highest (i.e., worst) score one could achieve in this event and still have a chance of winning (or at least tying for first place overall)?


\section*{Solution:}

The product of all scores in each event is $8!$, so the total product across all events is $8!^{3}$.
If each player were to have the same score at the end (an eight-way tie), that score would be $8!^{3/8}\approx53.3$.
Of course, this is impossible, but it sets an upper limit on the winning score; If a ``winning'' score were 54 or higher, another score would need to be 53 or lower, which would then become the winning score.

A score of 53 is not possible, since it is prime.
Similarly, 52 and 51 are not possible, since they have factors of 13 and 17, respectively.

The first possible score is 50, which can come from event scores of 2, 5, and 5.
For this to be a winning score, each other climber must have scores at or above 50.
Using the same logic as earlier, the average score of the remaining seven climbers must be $(8!^{3}/50)^{1/7}\approx53.8$.
So at least one other climber must have a score below 54.
Scores of 53, 52, 51, as well as scores below 50 are excluded.
So the next-best score must also be 50.
However, two (or more) climbers cannot both score 50, because the total product ($8!^{3}$) only has three factors of 5.
So 50 is not the solution.

The next possible score is 49 (from 1, 7, and 7).
After a score of 49, the remaining average is $(8!^{3}/49)^{1/7}\approx53.99$.
Again, the next-best score must be below 54.
It also cannot be 49, because the total product only has three factors of 7.
If the second-best score is 50, then the remaining average is $(8!^{3}/49/50)^{1/6}\approx54.7$.
The third-best score again cannot be 50, but it could be 54 (from 3, 3, and 6).
The remaining average is $(8!^{3}/49/50/54)^{1/5}\approx54.8$.
At this point, the fourth-best score would have to be 54.
However, this is not possible, since it \textit{must} be factored as 3, 3, and 6.
But a third-place event finish can only happen three times, so this is not possible, and 49 is also not the solution.

The next possible score is 48.
For this to be the solution, it only needs to be shown that some combination of event scores results in a lowest, winning score of 48.
I show this below, using A--H as the players, ranked in order of total score.

\vspace{0.1in}
\begin{center}
\begin{tabular}{ccccc}
Player & Score 1 & Score 2 & Score 3 & Total Score \\
\hline
A & 1 & 6 & 8 & 48 \\
B & 6 & 8 & 1 & 48 \\
C & 8 & 1 & 6 & 48 \\
D & 3 & 4 & 4 & 48 \\
E & 2 & 5 & 5 & 50 \\
F & 4 & 2 & 7 & 56 \\
G & 7 & 3 & 3 & 63 \\
H & 5 & 7 & 2 & 70 \\
\end{tabular}
\end{center}
\vspace{0.1in}

So in this example, there is a four-way tie for first place.
But it is sufficient to show that the highest-possible winning score is \fcolorbox{red}{white}{\textbf{48}}\,.
I don't know if this example (along with permutations) is unique.


\end{document}