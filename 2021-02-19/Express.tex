\documentclass{article}

\usepackage{amsmath} % math stuff
\usepackage{amssymb} % math stuff
\usepackage{array} % equations and stuff
\usepackage{bm} % bold math
%\usepackage{booktabs} % extra table rule options
%\usepackage{caption} % suppressed table numbering; incompatible with revtex, and longtable, I think
\usepackage{comment} % comment environment
%\usepackage{enumitem} % customization of enumeration, itemize, and description
\usepackage[T1]{fontenc} % font encoding for special characters, must also use scalable font package
\usepackage[margin=0.8in]{geometry} % paper sizes and margins (but be careful not to mess up pre-defined pages)
\usepackage{graphicx} % for graphics
%\usepackage{helvet} % default font is the helvetica postscript font
\usepackage{layouts} % print units like widths
\usepackage{lipsum} % lorem ipsum filler text
\usepackage{lmodern} % scalable font?
\usepackage{longtable} % multi-page tables
\usepackage{makecell} % specify line-breaks in table cells
\usepackage{mathrsfs} % math script font
\usepackage{mhchem} % easier chemical formula
\usepackage{microtype} % allows disabling of ligatures
%\usepackage{newcent} % new century schoolbook font
\usepackage{nicefrac}
\usepackage{numprint} % print and format (large) numbers
\usepackage{parskip} % removes paragraph indentation, and adjusts paragraph skip, as well as list items
\usepackage{pdfpages} % add pdf files as pages
%\usepackage{setspace} % adjust text spacing and indents
\usepackage{siunitx} % decimal alignment
\usepackage{subfigure} % divided figures
%\usepackage{tabu} % extra table options
\usepackage{textcomp} % symbols
\usepackage{threeparttablex} % better footnotes with longtable
\usepackage{titling} % title placement
\usepackage{ulem} % strikethrough text
%\usepackage{url} % superceded by hyperref
\usepackage{verbatim} % verbatim environment
\usepackage{xcolor} % colors and color boxes
\usepackage{xspace} % commands that don't eat up white space
\usepackage{hyperref} % links and page setup; should always come last

\hypersetup{
 bookmarks=true,
 colorlinks=true,
 citecolor=blue,
 linkcolor=blue,
 urlcolor=blue,
 pdfstartview={XYZ null null 1.0} % default open view is 100%
}

\DisableLigatures[f,t]{encoding = T1} % disable ff, fi, fl, tt ligatures, without f option, it also disables -- = endash
\renewcommand{\arraystretch}{2.0} % extra vertical space in tables

% define centered, left- and right-aligned columns with specified widths
\newcommand{\PreserveBackslash}[1]{\let\temp=\\#1\let\\=\temp}
\newcolumntype{C}[1]{>{\PreserveBackslash\centering}p{#1}}
\newcolumntype{L}[1]{>{\PreserveBackslash\raggedright}p{#1}}
\newcolumntype{R}[1]{>{\PreserveBackslash\raggedleft}p{#1}}

\begin{document}

\pagestyle{empty} % don't number pages

% custom title
\begin{center}
{\LARGE Express Riddler}

\vspace{0.15in}

{\Large 19 February 2021}
\end{center}


\section*{Riddle:}

This week marks the third of four CrossProduct™ puzzles.
This time, there are \textit{seven} three-digit numbers---each belongs in a row of the table below, with one digit per cell.
The products of the three digits of each number are shown in the rightmost column.
Meanwhile, the products of the digits in the hundreds, tens and ones places, respectively, are shown in the bottom row.

\begin{center}
\begin{tabular}{|C{0.75in}|C{0.75in}|C{0.75in}!{\vrule width 1.5pt}R{0.75in}|}
\hline
 & & & 280 \\
\hline
 & & & 168 \\
\hline
 & & & 162 \\
\hline
 & & & 360 \\
\hline
 & & & 60 \\
\hline
 & & & 256 \\
\hline
 & & & 126 \\
\noalign{\hrule height 2pt}
183,708 & 245,760 & 117,600 & \\
\hline
\end{tabular}
\end{center}

Can you find all seven three-digit numbers and complete the table?

\section*{Solution:}

The riddle is essentially asking for 21 digits to be placed in a $7\times3$ table.
Generally, the first step is to decompose each number into either three or seven (not-necessarily-prime) single-digit factors.
For this riddle, though, I only decomposed the rows into factors.
I list all possible sets of factors for each row below:

\begin{center}
\setlength{\tabcolsep}{3pt}
\begin{tabular}{lll}
280 & $=5\cdot7\cdot8$ \\
168 & $=3\cdot7\cdot8$ & $=4\cdot6\cdot7$ \\
162 & $=2\cdot9\cdot9$ & $=3\cdot6\cdot9$ \\
360 & $=5\cdot8\cdot9$ \\
60  & $=2\cdot5\cdot6$ & $=3\cdot4\cdot5$ \\
256 & $=4\cdot8\cdot8$ \\
126 & $=2\cdot7\cdot9$ & $=3\cdot6\cdot7$
\end{tabular}
\end{center}

The 183,708 is not divisible by 8, so for the sixth row, the first column must be 4.
The remaining two spots in the row must be the 8s.
The sixth row is therefore 4,8,8.

The 183,708 is not divisible by 5 (in addition to 8), so for the fourth row, the first column must be 9.
The 117,600 already has a factor of 8 and does not have another factor of 8, so the last column must be a 5.
The fourth row is therefore 9,8,5.

Now that the first column has factors of 4 and 9, the only remaining factor in common with 280 is 7, which goes in the first column of the first row.
Once again, the last column cannot have another 8, so the last column of the first row must be 5.
The first row is therefore 7,8,5.

The remaining 5 from 60 must go in the middle column.
Since the first column has no remaining factors of 2, the factor from 60 must be a 3.
The fifth row is therefore 3,5,4.

The two remaining 7s from the 168 and 126 must go in the last column.
The only factor left in the last column is 3, which goes in the third row.
The other two factors of 162 must be 6 and 9.
Again, since the first column has no remaining factors of 2, the third row must be a 9.
The third row is therefore 9,6,3.

Using the same logic as the third row, the seventh row must have 9 in the first column, leaving 2 in the middle.
The seventh row is therefore 9,2,7.

The only remaining factors for the 168 are either 3 and 8 or 4 and 6.
Using the same logic as the previous rows, the three must be in the first column, leaving 8 in the middle.
The second row is therefore 3,8,7.

The final solution is

\begin{center}
\begin{tabular}{|C{0.75in}|C{0.75in}|C{0.75in}|}
\hline
7 & 8 & 5 \\
\hline
3 & 8 & 7 \\
\hline
9 & 6 & 3 \\
\hline
9 & 8 & 5 \\
\hline
3 & 5 & 4 \\
\hline
4 & 8 & 8 \\
\hline
9 & 2 & 7 \\
\hline
\end{tabular}
\end{center}


\end{document}