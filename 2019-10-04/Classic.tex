\documentclass{article}

\usepackage{amsmath} % math stuff
\usepackage{amssymb} % math stuff
\usepackage{array} % equations and stuff
\usepackage{bm} % bold math
%\usepackage{caption} % suppressed table numbering; incompatible with revtex, and longtable, I think
\usepackage{comment} % comment environment
%\usepackage{enumitem} % customization of enumeration, itemize, and description
\usepackage[T1]{fontenc} % font encoding for special characters, must also use scalable font package
\usepackage[margin=0.8in]{geometry} % paper sizes and margins (but be careful not to mess up pre-defined pages)
\usepackage{graphicx} % for graphics
%\usepackage{helvet} % default font is the helvetica postscript font
\usepackage{layouts} % print units like widths
\usepackage{lipsum} % lorem ipsum filler text
\usepackage{lmodern} % scalable font?
\usepackage{longtable} % multi-page tables
\usepackage{makecell} % specify line-breaks in table cells
\usepackage{mathrsfs} % math script font
\usepackage{mhchem} % easier chemical formula
\usepackage{microtype} % allows disabling of ligatures
%\usepackage{newcent} % new century schoolbook font
\usepackage{nicefrac}
\usepackage{numprint} % print and format (large) numbers
\usepackage{parskip} % removes paragraph indentation, and adjusts paragraph skip, as well as list items
\usepackage{pdfpages} % add pdf files as pages
%\usepackage{setspace} % adjust text spacing and indents
\usepackage{siunitx} % decimal alignment
\usepackage{subfigure} % divided figures
%\usepackage{tabu} % extra table options
\usepackage{textcomp} % symbols
\usepackage{threeparttablex} % better footnotes with longtable
\usepackage{titling} % title placement
\usepackage{ulem} % strikethrough text
%\usepackage{url} % superceded by hyperref
\usepackage{verbatim} % verbatim environment
\usepackage{xcolor} % colors and color boxes
\usepackage{xspace} % commands that don't eat up white space
\usepackage{hyperref} % links and page setup; should always come last

\hypersetup{
 bookmarks=true,
 colorlinks=true,
 citecolor=blue,
 linkcolor=blue,
 urlcolor=blue,
 pdfstartview={XYZ null null 1.0} % default open view is 100%
}

\DisableLigatures[f,t]{encoding = T1} % disable ff, fi, fl, tt ligatures, without f option, it also disables -- = endash
\renewcommand{\arraystretch}{2.0} % extra vertical space in tables

\begin{document}

\pagestyle{empty} % don't number pages

% custom title
\begin{center}
{\LARGE Classic Riddler}

\vspace{0.15in}

{\Large 4 October 2019}
\end{center}


\section*{Riddle:}

The classic birthday problem asks about how many people need to be in a room together before you have better-than-even odds that at least two of them have the same birthday.
Ignoring leap years, the answer is, paradoxically, only 23 people---fewer than you might intuitively think.

But Joel noticed something interesting about a well-known group of 100 people: In the U.S. Senate, three senators happen to share the same birthday of October 20: Kamala Harris, Brian Schatz and Sheldon Whitehouse.

And so Joel has thrown a new wrinkle into the classic birthday problem.
How many people do you need to have better-than-even odds that at least three of them have the same birthday?
(Again, ignore leap years.)


\section*{Solution:}

I'm sure that there is an analytical solution that could be found with an absurd number of factorials and other large numbers.
But it was rather easy to set this up in a script, which I put in \texttt{Three\_birthdays.C}\,.
The script tests a given number of people to determine the probability that three have the same birthday.
I also make the assumption that each of the 365 birthdays is equally likely (which is not actually true).
The script doesn't bother making a calculation if there are more than 730 people; by the pigeon-hole principle, there will always be at least three people who share a birthday in this situation.
It is up to the user to test individual numbers of people one-by-one until the probability goes over 50\%.

In my tests, with 87 people, the average probability was 49.95\%, and with 88 people, the average was 51.09\%.
I'm not absolutely certain that the 87-person probability is definitely below 50\%, but I can't do any better than what I have.
So my solution is
\fcolorbox{red}{white}{\bf 88}\,.




\end{document}