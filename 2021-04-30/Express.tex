\documentclass{article}

\usepackage{amsmath} % math stuff
\usepackage{amssymb} % math stuff
\usepackage{array} % equations and stuff
\usepackage{bm} % bold math
%\usepackage{booktabs} % extra table rule options
%\usepackage{caption} % suppressed table numbering; incompatible with revtex, and longtable, I think
\usepackage{comment} % comment environment
%\usepackage{enumitem} % customization of enumeration, itemize, and description
\usepackage[T1]{fontenc} % font encoding for special characters, must also use scalable font package
\usepackage[margin=0.8in]{geometry} % paper sizes and margins (but be careful not to mess up pre-defined pages)
\usepackage{graphicx} % for graphics
%\usepackage{helvet} % default font is the helvetica postscript font
\usepackage{layouts} % print units like widths
\usepackage{lipsum} % lorem ipsum filler text
\usepackage{lmodern} % scalable font?
\usepackage{longtable} % multi-page tables
\usepackage{makecell} % specify line-breaks in table cells
\usepackage{mathrsfs} % math script font
\usepackage{mhchem} % easier chemical formula
\usepackage{microtype} % allows disabling of ligatures
%\usepackage{newcent} % new century schoolbook font
\usepackage{nicefrac}
\usepackage{numprint} % print and format (large) numbers
\usepackage{parskip} % removes paragraph indentation, and adjusts paragraph skip, as well as list items
\usepackage{pdfpages} % add pdf files as pages
%\usepackage{setspace} % adjust text spacing and indents
\usepackage{siunitx} % decimal alignment
\usepackage{subfigure} % divided figures
%\usepackage{tabu} % extra table options
\usepackage{textcomp} % symbols
\usepackage{threeparttablex} % better footnotes with longtable
\usepackage{titling} % title placement
\usepackage{ulem} % strikethrough text
%\usepackage{url} % superceded by hyperref
\usepackage{verbatim} % verbatim environment
\usepackage{xcolor} % colors and color boxes
\usepackage{xspace} % commands that don't eat up white space
\usepackage{hyperref} % links and page setup; should always come last

\hypersetup{
 bookmarks=true,
 colorlinks=true,
 citecolor=blue,
 linkcolor=blue,
 urlcolor=blue,
 pdfstartview={XYZ null null 1.0} % default open view is 100%
}

\DisableLigatures[f,t]{encoding = T1} % disable ff, fi, fl, tt ligatures; without options, it also disables -- = endash
\renewcommand{\arraystretch}{1.2} % extra vertical (and horizontal?) space in tables

% define centered, left- and right-aligned columns with specified widths
\newcommand{\PreserveBackslash}[1]{\let\temp=\\#1\let\\=\temp}
\newcolumntype{C}[1]{>{\PreserveBackslash\centering}p{#1}}
\newcolumntype{L}[1]{>{\PreserveBackslash\raggedright}p{#1}}
\newcolumntype{R}[1]{>{\PreserveBackslash\raggedleft}p{#1}}

\begin{document}

\pagestyle{empty} % don't number pages

% custom title
\begin{center}
{\LARGE Express Riddler}

\vspace{0.15in}

{\Large 30 April 2021}
\end{center}


\section*{Riddle:}

You're playing a game of cornhole with your friends, and it's your turn to toss the four bean bags.
For every bean bag you toss onto your opponents' board, you get 1 point.
For every bag that goes through the hole on their board, you get 3 points.
And for any bags that don't land on the board or through the hole, you get 0 points.

Your opponents had a terrible round, missing the board with all their throws.
Meanwhile, your team currently has 18 points---just 3 points away from victory at 21.
You're also playing with a special house rule: To win, you must score \textit{exactly} 21 points, without going over.

Based on your history, you know there are three kinds of throws you can make:

\begin{itemize}
\item An \textit{aggressive} throw, which has a 40 percent chance of going in the hole, a 30 percent chance of landing on the board and a 30 percent chance of missing the board and hole.
\item A \textit{conservative} throw, which has a 10 percent chance of going in the hole, an 80 percent chance of landing on the board and a 10 percent chance of missing the board and hole.
\item A \textit{wasted} throw, which has a 100 percent chance of missing the board and hole.
\end{itemize}

For each bean bag, you can choose any of these three throws.
Your goal is to maximize your chances of scoring exactly 3 points with your four tosses. What is the probability that your team will finish the round with exactly 21 points and declare victory?



\section*{Solution:}

For this problem I define a state $(M,N)$ as having $M$ moves left ($M=0,1,2,4$) and $N$ points left to win ($N=0,1,2,3$ or $X$ for an overshoot).
Each state has an associated probability $P(M,N)$ that is the probability to win using the best strategy.
The simplest probabilities to define are for $M=0$ (after the fourth throw).
For $N=0$, $P=1$, and for all other $N$, $P=0$.
For $N=X$ and any $M$, clearly $P=0$.
It is also obvious that from any state $(M>0,0)$, the best strategy is wasted throws until $M=0$, which ensures that $P=1$.

From these base set of states, I built up the other probabilities one move at a time, considering the outcomes of either the aggressive or conservative throws.
For example, starting at $(1,1)$ and throwing aggressively ends at the winning state (0,0) 30\% of the time and loses otherwise, but throwing conservatively ends at the winning state 80\% of the time.
So from this state the best strategy is the conservative throw, and $P(1,1)=0.8$.
Conversely, from the state (1,2), the only possible throws lead to (0,1), (0,2), and $(0,X)$, which are all losing positions, so $P(1,2)=0$.

It turns out that the ideal strategy is pretty intuitive, as follows:

\begin{itemize}
\item If exactly 0 points are still needed, throw wastes.
\item If exactly 3 points are still needed, throw aggressively.
\item Otherwise, throw conservatively.
\end{itemize}

This results in the following probabilities for all the relevant $M$ and $N$:

\vspace{0.1in}
\begin{center}
\begin{tabular}{C{0.2in}C{0.2in}|C{0.32in}|C{0.32in}|C{0.32in}|C{0.32in}|C{0.32in}|}
 & \multicolumn{1}{c}{} & \multicolumn{5}{c}{$N$} \\
 & \multicolumn{1}{c}{} & \multicolumn{1}{c}{0} & \multicolumn{1}{c}{1} & \multicolumn{1}{c}{2} & \multicolumn{1}{c}{3} & \multicolumn{1}{c}{$X$} \\
\cline{3-7}
 & 0 & 1 & 0 & 0 & 0 & 0 \\
\cline{3-7}
 & 1 & 1 & 0.8 & 0 & 0.4 & 0 \\
\cline{3-7}
M & 2 & 1 & 0.88 & 0.64 & 0.52 & 0 \\
\cline{3-7}
 & 3 & 1 & 0.888 & 0.768 & 0.748 & 0 \\
\cline{3-7}
 & 4 & 1 & 0.8888 & 0.7872 & 0.8548 & 0 \\ 
\cline{3-7}
\end{tabular}
\end{center}

So the final chance of winning 3 points in 4 throws is
\fcolorbox{red}{white}{\bf 85.48\%}\,.

\end{document}