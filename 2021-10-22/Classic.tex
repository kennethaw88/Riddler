\documentclass{article}

\usepackage{amsmath} % math stuff
\usepackage{amssymb} % math stuff
\usepackage{array} % equations and stuff
\usepackage{bm} % bold math
%\usepackage{booktabs} % extra table rule options
%\usepackage{caption} % suppressed table numbering; incompatible with revtex, and longtable, I think
\usepackage{comment} % comment environment
%\usepackage{enumitem} % customization of enumeration, itemize, and description
\usepackage[T1]{fontenc} % font encoding for special characters, must also use scalable font package
\usepackage[margin=0.8in]{geometry} % paper sizes and margins (but be careful not to mess up pre-defined pages)
\usepackage{graphicx} % for graphics
%\usepackage{helvet} % default font is the helvetica postscript font
\usepackage[utf8]{inputenc} % special characters in tex input
\usepackage{layouts} % print units like widths
\usepackage{lipsum} % lorem ipsum filler text
\usepackage{lmodern} % scalable font?
\usepackage{longtable} % multi-page tables
\usepackage{makecell} % specify line-breaks in table cells
\usepackage{mathrsfs} % math script font
\usepackage{mhchem} % easier chemical formula
\usepackage{microtype} % allows disabling of ligatures
\usepackage{multicol} % multicolumns
%\usepackage{newcent} % new century schoolbook font
\usepackage{nicefrac}
\usepackage{numprint} % print and format (large) numbers
\usepackage{parskip} % removes paragraph indentation, and adjusts paragraph skip, as well as list items
\usepackage{pdfpages} % add pdf files as pages
%\usepackage{setspace} % adjust text spacing and indents
\usepackage{siunitx} % decimal alignment
\usepackage{subfigure} % divided figures
%\usepackage{tabu} % extra table options
\usepackage{textcomp} % symbols
\usepackage{threeparttablex} % better footnotes with longtable
\usepackage{titling} % title placement
\usepackage{ulem} % strikethrough text
%\usepackage{url} % superceded by hyperref
\usepackage{verbatim} % verbatim environment
\usepackage{xcolor} % colors and color boxes
\usepackage{xspace} % commands that don't eat up white space
\usepackage{hyperref} % links and page setup; should always come last

\hypersetup{
 bookmarks=true,
 colorlinks=true,
 citecolor=blue,
 linkcolor=blue,
 urlcolor=blue,
 pdfstartview={XYZ null null 1.0} % default open view is 100%
}

\DisableLigatures[f,t]{encoding = T1} % disable ff, fi, fl, tt ligatures; without options, it also disables -- = endash
\renewcommand{\arraystretch}{1.0} % extra vertical (and horizontal?) space in tables

% define centered, left- and right-aligned columns with specified widths
\newcommand{\PreserveBackslash}[1]{\let\temp=\\#1\let\\=\temp}
\newcolumntype{C}[1]{>{\PreserveBackslash\centering}p{#1}}
\newcolumntype{L}[1]{>{\PreserveBackslash\raggedright}p{#1}}
\newcolumntype{R}[1]{>{\PreserveBackslash\raggedleft}p{#1}}

\begin{document}

\pagestyle{empty} % don't number pages

% custom title
\begin{center}
{\LARGE Classic Riddler}

\vspace{0.15in}

{\Large 22 October 2021}
\end{center}


\section*{Riddle:}

Suppose you have an equilateral triangle.
You pick three random points, one along each of its three edges, uniformly along the length of each edge---that is, each point along each edge has the same probability of being selected.

With those three randomly selected points, you can form a new triangle inside the original one.
What is the probability that the center of the larger triangle also lies inside the smaller one?


\section*{Solution:}

To solve this problem, I set up the diagram below.
I set the side length of the triangle to be 2, with vertices located at $(0,0)$, $(2,0)$, and $(1,\sqrt{3})$.
The center point Q of the triangle is at $(1,\nicefrac{\sqrt{3}}{3})$.
I designate the three points along the sides as A, B, and C; $a$, $b$, and $c$ are the variables that represent the proportion along the side length where A, B, and C are located, respectively.

\vspace{0.1in}
\begin{center}
\includegraphics[width=2.5in]{triangle1.png}
\end{center}
\vspace{0.1in}

Based on this, point A is at $(a,\sqrt{3}a)$, B is at $(1+b,\sqrt{3}(1-b))$, and C is at $(2(1-c),0)$.

There are eight scenarios for the variables.
First, if each of $a$, $b$, and $c$ is greater than \nicefrac{1}{2}, then Q is guaranteed to be inside the smaller triangle.
This happens because if $a$ is greater than \nicefrac{1}{2}, the line $\overline{\mathrm{AB}}$ lies above Q no matter the value of $b$.
Similarly, if $b$ is greater than \nicefrac{1}{2}, $\overline{\mathrm{BC}}$ lies to the right of Q, and if $c$ is greater than \nicefrac{1}{2}, $\overline{\mathrm{AC}}$ lies to the left of Q.
The same reasoning applies when $a$, $b$, and $c$ are each less than \nicefrac{1}{2}.
Because $a$, $b$, and $c$ are evenly distributed between 0 and 1, these two scenarios each happen with probability $(\nicefrac{1}{2})^{3}=\nicefrac{1}{8}$.

Each of the other six scenarios involves two variables on the same side of \nicefrac{1}{2} and the remaining variable on the other side of \nicefrac{1}{2}.
Because of symmetry, each of these will have the same probability of having Q inside the smaller triangle.
So without loss of generality, I consider the case with $a$ greater than \nicefrac{1}{2} and $c$ less than \nicefrac{1}{2}.
As mentioned above, it doesn't matter what $b$ is when $a$ is greater than \nicefrac{1}{2}, so I actually consider both scenarios together for $b$ less than and greater than \nicefrac{1}{2}.

For a given value of $a$, I draw the line $\overline{\mathrm{AQ}}$ and extend it to the C side of the equilateral triangle.
I designate the point of intersection as $\mathrm{C_{0}}$.
This is shown in the second diagram below.

\vspace{0.1in}
\begin{center}
\includegraphics[width=2.5in]{triangle2.png}
\end{center}
\vspace{0.1in}

I can determine that $\mathrm{C_{0}}$ is at $(\nicefrac{2a}{(3a-1)},0)$.
Because this is an absolute point on the triangle, I can convert this to the proportion along the axis, which I designate $c_{0}$.
The value of $c_{0}$ is \nicefrac{(2a-1)}{(3a-1)}.
This becomes a lower limit for integration along $c$.
The upper limit within this scenario is \nicefrac{1}{2}.
The variable $a$ is integrated between \nicefrac{1}{2} and 1, and $b$ is ignored in the scenario.
The probability $p$ of having Q inside the smaller triangle in this scenario is

\[
p=\int_{\frac{1}{2}}^{1}\int_{\frac{2a-1}{3a-1}}^{\frac{1}{2}}dc\,da
\]

which has a result of approximately 0.0707.
Because this represents two of six equivalent scenarios, this must be tripled and added to the probabilities above.
The total probability is (approximately) \fcolorbox{red}{white}{\textbf{0.4621}}\,.


\end{document}