\documentclass{article}

\usepackage{amsmath} % math stuff
\usepackage{amssymb} % math stuff
\usepackage{array} % equations and stuff
\usepackage{bm} % bold math
%\usepackage{booktabs} % extra table rule options
%\usepackage{caption} % suppressed table numbering; incompatible with revtex, and longtable, I think
\usepackage{comment} % comment environment
%\usepackage{enumitem} % customization of enumeration, itemize, and description
\usepackage[T1]{fontenc} % font encoding for special characters, must also use scalable font package
\usepackage[margin=0.8in]{geometry} % paper sizes and margins (but be careful not to mess up pre-defined pages)
\usepackage{graphicx} % for graphics
%\usepackage{helvet} % default font is the helvetica postscript font
\usepackage[utf8]{inputenc} % special characters in tex input
\usepackage{layouts} % print units like widths
\usepackage{lipsum} % lorem ipsum filler text
\usepackage{lmodern} % scalable font?
\usepackage{longtable} % multi-page tables
\usepackage{makecell} % specify line-breaks in table cells
\usepackage{mathrsfs} % math script font
\usepackage{mhchem} % easier chemical formula
\usepackage{microtype} % allows disabling of ligatures
\usepackage{multicol} % multicolumns
%\usepackage{newcent} % new century schoolbook font
\usepackage{nicefrac}
\usepackage{numprint} % print and format (large) numbers
\usepackage{parskip} % removes paragraph indentation, and adjusts paragraph skip, as well as list items
\usepackage{pdfpages} % add pdf files as pages
%\usepackage{setspace} % adjust text spacing and indents
\usepackage{siunitx} % decimal alignment
\usepackage{subfigure} % divided figures
%\usepackage{tabu} % extra table options
\usepackage{textcomp} % symbols
\usepackage{threeparttablex} % better footnotes with longtable
\usepackage{titling} % title placement
\usepackage{ulem} % strikethrough text
%\usepackage{url} % superceded by hyperref
\usepackage{verbatim} % verbatim environment
\usepackage{xcolor} % colors and color boxes
\usepackage{xspace} % commands that don't eat up white space
\usepackage{hyperref} % links and page setup; should always come last

\hypersetup{
 bookmarks=true,
 colorlinks=true,
 citecolor=blue,
 linkcolor=blue,
 urlcolor=blue,
 pdfstartview={XYZ null null 1.0} % default open view is 100%
}

\DisableLigatures[f,t]{encoding = T1} % disable ff, fi, fl, tt ligatures; without options, it also disables -- = endash
\renewcommand{\arraystretch}{1.0} % extra vertical (and horizontal?) space in tables

% define centered, left- and right-aligned columns with specified widths
\newcommand{\PreserveBackslash}[1]{\let\temp=\\#1\let\\=\temp}
\newcolumntype{C}[1]{>{\PreserveBackslash\centering}p{#1}}
\newcolumntype{L}[1]{>{\PreserveBackslash\raggedright}p{#1}}
\newcolumntype{R}[1]{>{\PreserveBackslash\raggedleft}p{#1}}

\begin{document}

\pagestyle{empty} % don't number pages

% custom title
\begin{center}
{\LARGE Express Riddler}

\vspace{0.15in}

{\Large 22 October 2021}
\end{center}


\section*{Riddle:}

Duke Leto Atreides knows for a fact that there are not one, but two traitors within his royal household.
The suspects are Lady Jessica, Dr. Wellington Yueh, Gurney Halleck and Duncan Idaho.
Leto’s advisor, Thufir Hawat, will assist him in questioning the four suspects.
Anyone who is a traitor will tell a lie, while anyone who is not a traitor will tell the truth.

Upon interrogation, Jessica says that she is not the traitor, while Wellington similarly says that he is not the traitor.
Gurney says that Jessica or Wellington is a traitor.
Finally, Duncan says that Jessica or Gurney is a traitor.
(Thufir, being the logician that he is, notes that when someone says thing \textit{A} is true or thing \textit{B} is true, both \textit{A} and \textit{B} can technically be true.)

After playing back the interrogations in his mind, Thufir is ready to report the name of one of the traitors to the duke.
Whose name does he report?


\section*{Solution:}

There are $\binom{4}{2}=6$ ways to have two traitors among four people, whom I designate J, W, G, and D.
I list these scenarios below.
They are organized by whose statements are true or false, along with whether the resulting statements are logically consistent:

\vspace{0.1in}
\begin{center}
\begin{tabular}{ccc}
True & False & Consistent? \\
\hline
J W & G D & No \\
J G & W D & Yes \\
J D & W G & No \\
W G & J D & No \\
W D & J G & No \\
G D & J W & Yes
\end{tabular}
\end{center}
\vspace{0.1in}

There are two scenarios which are logically consistent, so the identity of both traitors cannot be determined.
But in both of those scenarios, Wellington is one of the traitors.
So the solution is \fcolorbox{red}{white}{\textbf{Dr. Wellington Yueh}}\,.


\end{document}