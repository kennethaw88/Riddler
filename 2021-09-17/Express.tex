\documentclass{article}

\usepackage{amsmath} % math stuff
\usepackage{amssymb} % math stuff
\usepackage{array} % equations and stuff
\usepackage{bm} % bold math
%\usepackage{booktabs} % extra table rule options
%\usepackage{caption} % suppressed table numbering; incompatible with revtex, and longtable, I think
\usepackage{comment} % comment environment
%\usepackage{enumitem} % customization of enumeration, itemize, and description
\usepackage[T1]{fontenc} % font encoding for special characters, must also use scalable font package
\usepackage[margin=0.8in]{geometry} % paper sizes and margins (but be careful not to mess up pre-defined pages)
\usepackage{graphicx} % for graphics
%\usepackage{helvet} % default font is the helvetica postscript font
\usepackage[utf8]{inputenc} % special characters in tex input
\usepackage{layouts} % print units like widths
\usepackage{lipsum} % lorem ipsum filler text
\usepackage{lmodern} % scalable font?
\usepackage{longtable} % multi-page tables
\usepackage{makecell} % specify line-breaks in table cells
\usepackage{mathrsfs} % math script font
\usepackage{mhchem} % easier chemical formula
\usepackage{microtype} % allows disabling of ligatures
\usepackage{multicol} % multicolumns
%\usepackage{newcent} % new century schoolbook font
\usepackage{nicefrac}
\usepackage{numprint} % print and format (large) numbers
\usepackage{parskip} % removes paragraph indentation, and adjusts paragraph skip, as well as list items
\usepackage{pdfpages} % add pdf files as pages
%\usepackage{setspace} % adjust text spacing and indents
\usepackage{siunitx} % decimal alignment
\usepackage{subfigure} % divided figures
%\usepackage{tabu} % extra table options
\usepackage{textcomp} % symbols
\usepackage{threeparttablex} % better footnotes with longtable
\usepackage{titling} % title placement
\usepackage{ulem} % strikethrough text
%\usepackage{url} % superceded by hyperref
\usepackage{verbatim} % verbatim environment
\usepackage{xcolor} % colors and color boxes
\usepackage{xspace} % commands that don't eat up white space
\usepackage{hyperref} % links and page setup; should always come last

\hypersetup{
 bookmarks=true,
 colorlinks=true,
 citecolor=blue,
 linkcolor=blue,
 urlcolor=blue,
 pdfstartview={XYZ null null 1.0} % default open view is 100%
}

\DisableLigatures[f,t]{encoding = T1} % disable ff, fi, fl, tt ligatures; without options, it also disables -- = endash
\renewcommand{\arraystretch}{1.0} % extra vertical (and horizontal?) space in tables

% define centered, left- and right-aligned columns with specified widths
\newcommand{\PreserveBackslash}[1]{\let\temp=\\#1\let\\=\temp}
\newcolumntype{C}[1]{>{\PreserveBackslash\centering}p{#1}}
\newcolumntype{L}[1]{>{\PreserveBackslash\raggedright}p{#1}}
\newcolumntype{R}[1]{>{\PreserveBackslash\raggedleft}p{#1}}

\begin{document}

\pagestyle{empty} % don't number pages

% custom title
\begin{center}
{\LARGE Express Riddler}

\vspace{0.15in}

{\Large 17 September 2021}
\end{center}


\section*{Riddle:}

I recently came across a rather peculiar recipe for something called Babylonian radish pie.
Intrigued, I began to follow the directions, which said I could start with any number of cups of flour.

\textit{Any} number?
I mean, I had to start with \textit{some} flour, so zero cups wasn't an option.
But according to the recipe, any positive value was fair game.
Next, I needed a second amount of flour that was 3 divided by my original number.
For example, if I had started with two cups of flour, then the recipe told me I now needed 3 divided by 2, or 1.5, cups at this point.

I was then instructed to combine these amounts of flour and discard half.
Apparently, this was my new starting amount of flour.
I was to repeat the process, combining this amount with 3 divided by it and then discarding half.

The recipe told me to keep doing this, over and over.
Eventually, I'd have the proper number of cups of flour for my radish pie.

How many cups of flour does the recipe ultimately call for?

\section*{Solution:}

If the amount of flour does indeed converge, then the starting and ending amount of flour in subsequent iterations is the same. If that amount of flour is $x$, then this can be expressed as

\[
\frac{\left(x+\frac{3}{x}\right)}{2}=x
\]

The positive solution to this equation is \fcolorbox{red}{white}{$\bm{\sqrt{3}}$}\,.
I do not know how to prove that the solution converges.


\end{document}