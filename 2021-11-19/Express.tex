\documentclass{article}

\usepackage{amsmath} % math stuff
\usepackage{amssymb} % math stuff
\usepackage{array} % equations and stuff
\usepackage{bm} % bold math
%\usepackage{booktabs} % extra table rule options
%\usepackage{caption} % suppressed table numbering; incompatible with revtex, and longtable, I think
\usepackage{comment} % comment environment
%\usepackage{enumitem} % customization of enumeration, itemize, and description
\usepackage[T1]{fontenc} % font encoding for special characters, must also use scalable font package
\usepackage[margin=0.8in]{geometry} % paper sizes and margins (but be careful not to mess up pre-defined pages)
\usepackage{graphicx} % for graphics
%\usepackage{helvet} % default font is the helvetica postscript font
\usepackage[utf8]{inputenc} % special characters in tex input
\usepackage{layouts} % print units like widths
\usepackage{lipsum} % lorem ipsum filler text
\usepackage{lmodern} % scalable font?
\usepackage{longtable} % multi-page tables
\usepackage{makecell} % specify line-breaks in table cells
\usepackage{mathrsfs} % math script font
\usepackage{mhchem} % easier chemical formula
\usepackage{microtype} % allows disabling of ligatures
\usepackage{multicol} % multicolumns
%\usepackage{newcent} % new century schoolbook font
\usepackage{nicefrac}
\usepackage{numprint} % print and format (large) numbers
\usepackage{parskip} % removes paragraph indentation, and adjusts paragraph skip, as well as list items
\usepackage{pdfpages} % add pdf files as pages
%\usepackage{setspace} % adjust text spacing and indents
\usepackage{siunitx} % decimal alignment
\usepackage{subfigure} % divided figures
%\usepackage{tabu} % extra table options
\usepackage{textcomp} % symbols
\usepackage{threeparttablex} % better footnotes with longtable
\usepackage{titling} % title placement
\usepackage{ulem} % strikethrough text
\usepackage{upgreek} % upright Greek letters
%\usepackage{url} % superceded by hyperref
\usepackage{verbatim} % verbatim environment
\usepackage{xcolor} % colors and color boxes
\usepackage{xspace} % commands that don't eat up white space
\usepackage{hyperref} % links and page setup; should always come last

\hypersetup{
 bookmarks=true,
 colorlinks=true,
 citecolor=blue,
 linkcolor=blue,
 urlcolor=blue,
 pdfstartview={XYZ null null 1.0} % default open view is 100%
}

\DisableLigatures[f,t]{encoding = T1} % disable ff, fi, fl, tt ligatures; without options, it also disables -- = endash
\renewcommand{\arraystretch}{1.0} % extra vertical (and horizontal?) space in tables

% define centered, left- and right-aligned columns with specified widths
\newcommand{\PreserveBackslash}[1]{\let\temp=\\#1\let\\=\temp}
\newcolumntype{C}[1]{>{\PreserveBackslash\centering}p{#1}}
\newcolumntype{L}[1]{>{\PreserveBackslash\raggedright}p{#1}}
\newcolumntype{R}[1]{>{\PreserveBackslash\raggedleft}p{#1}}

\begin{document}

\pagestyle{empty} % don't number pages

% custom title
\begin{center}
{\LARGE Express Riddler}

\vspace{0.15in}

{\Large 19 November 2021}
\end{center}


\section*{Riddle:}

After realizing that your friends on $\upmu\upepsilon\uptau\upalpha$ (The Riddler Social Network) are more popular than you, you decide to make some new friends at your local gym, which is open daily from 5~p.m. to~8 p.m.

Some gym members attend very often. Others barely show up at all.
As a matter of fact, there's a uniform distribution for how often the different members are in the gym---from 0 percent of the time that the gym is open to 100 percent of the time.

As a new member, you plan to be in the gym 50 percent of the time that it's open.
While working out, you decide to make friends with the first person you see.
(Aw!)

What is the probability that this person visits the gym more often than you?


\section*{Solution:}

If the percent of time a member spends at the gym is $x$, then the probability that that person is at the gym at any given time is of course $x$.
Thus, the probability distribution is simply a triangle bounded by a positively-sloped line (with a particular normalization factor).
Randomly meeting a person at the gym is equivalent to randomly selecting a point from inside the triangle.
To determine if that person visits the gym more often than 50\%\ of the time, it is necessary to find the proportion of the triangle to the right of the 50\%\ mark.
This is represented by the graph below:

\vspace{0.1in}
\begin{center}
\includegraphics[width=1.5in]{diagram}
\end{center}
\vspace{0.1in}

This can be solved geometrically or with calculus.
Using calculus, the probability is simply the ratio:

\[
\frac{\int_{0.5}^{1}x\,dx}{\int_{0}^{1}x\,dx}=0.75
\]

So the solution is \fcolorbox{red}{white}{\textbf{0.75}}\,.



\end{document}